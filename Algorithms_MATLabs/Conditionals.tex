\lab{Algorithms}{Conditionals and Loops}{Conditionals and Loops}

\objective{Introduce Loops and Conditionals}

Direct use of relational operators through masking is often the optimal way to do operations. However, there is a more user-friendly way to utilize logical operations: \li{if} statements.

An \li{if} statement simply executes a specific operation if a condition holds true. For example, suppose that we want to add one to a number if it is odd. We can do so with the following:

\begin{lstlisting}[style=matlab]
if mod(x,2) == 1
  x = x+1;
\end{lstlisting}

\li{If} statements also have a built-in way to operate if the statement is false. We can do that using the else command.

\begin{lstlisting}[style=matlab]
if mod(x,2) == 1
  x = x+1;
else
  x = x/2;
end
\end{lstlisting}

What does this code do? It adds one to a number if it is odd and divides by two if it is even. Note that the end command is necessary if the ``scope'' of the \li{if} statement spreads more than one line. Also, although the indentation is not required, it is good practice to indent \li{if} statements.

You can also nest \li{if} statemtents. For example:

\begin{lstlisting}[style=matlab]
if x < 5
    x = x+5;
else if x < 10
        x = 10;
    end
end
\end{lstlisting}

This code will add 5 if $x$ is less than 5, and set $x$ to 10 if $x$ is between 5 and 10.

Another statement that can be useful is the \li{switch} statement. It essentially allows us to select from among a number of choices, while a single \li{if} statement only allows us to select one. Here's an example of a switch statement

\begin{lstlisting}[style=matlab]
car = 'camry';

switch car
   case {'camry','corolla'}
      disp('Manufacturer is Toyota')
   case 'mustang'
      disp('Manufacturer is Ford')
   case 'viper'
      disp('Manufacturer is Dodge')
   otherwise
      disp('manufacturer unknown')
end

Manufacturer is Toyota
\end{lstlisting}

These statements are useful if we can pick among several choices (for example if our number is an integer and we need to treat it a number of different ways depending on what it is). These statements, however, are not as useful when we are dealing with decimal numbers.

Loops are another important programming tool. There are two type of loops in MATLAB: \li{for} and \li{while}.

A \li{for} loop is designed to iterate a specified number of times. The syntax is as follows:
\begin{lstlisting}[style=matlab]
for i = vector
   statement
end
\end{lstlisting}

The loop will iterate as many times as there are entries in the vector, with i being equal to a different entry of the vector each time. Suppose we want to add the prime numbers from one to a thousand. We can do so with a \li{for} loop as follows:

\begin{lstlisting}[style=matlab]
x = 0
for i = 1:1000
  if isprime(i)
    x = x+i;
  end
end
\end{lstlisting}

There are two important tools to point out now that you have the power to create meaningful functions. First, in MATLAB files you can add comments anywhere by putting the \% sign. This allows you to explain your code, or remove pieces while testing. This can be done to a complete line (or several lines) by hitting CTRL-R. Lines can be uncommented by using CTRL-T. You should get in the practice of writing comments into your code to explain what you are doing. Also, if you wish to halt whatever MATLAB is working on you can hit CTRL-C.

Additionally you can add help files to your functions. This is done by adding lines of comments directly after the function declaration (the first line of your function). When you type \li{help myFunction} at the command line it will then display however many contiguous comment lines are after the first line of the function.

\begin{problem}
Write a function that adds all of the prime numbers that are less than the input value n. This will simply require copying the code above into a function and modifying it to handle input. How long does it take if $n = 10^6$? 
\end{problem}

\begin{problem}
Now write a second function that does the same thing without using loops (use the \li{sum} and \li{find} function). How long does this function take if $n = 10^6$? The first is probably a lot easier to code and understand, but the second runs significantly faster.
Also, add a help file to your function. Test this out by typing \li{help myfunctionName}.
\end{problem}

The other type of loop is a \li{while} loop. The syntax is:

\begin{lstlisting}[style=matlab]
while condition
  statement
end
\end{lstlisting}

As long as the condition holds true the statement is executed. For example, the following will divide the $x$ by 2 as many times as it takes to make $x$ less than one.

\begin{lstlisting}[style=matlab]
while x > 1
    x = x/2;
end
\end{lstlisting}

\begin{problem}
One way to calculate the value of $e^x$ is to use the taylor expansion:
\[
e^x = \sum_{i=0}^\infty{\frac{x^i}{i!}}
\]
This method, however, is problematic when x becomes large (it requires a large number of terms to be accurate). We can utilize what is known as a scaling and squaring method to overcome this problem. This method exploits the property that $e^{2x} = (e^x)^2$. We can thus divide by exponent by two until we get a small enough x to be accurate, and then square the appropriate number of times.
Using while loops and a scaling and squaring method write a function expSS(x) that calculates e(x) using the Taylor Series Expansion. Test your function against MATLAB's built-in function using values like 3 and 30.
\end{problem}

\begin{problem}
Now modify your function so that it can accept a vector of inputs and calculate $e^x$ for each input. Make this function as fast as possible (investigate eliminating loops, etc).
\end{problem}

One last note about loops is that you can use the \li{break} command to step out of a loop. This can be useful in many cases. For example we could use the \li{break} command in the following code:

\begin{lstlisting}[style=matlab]
for i = 1:10
    x = x/2;
    if x < 1
        break;
    end
end
\end{lstlisting}

This code will divide $x$ by two, up to ten times. If $x$ ever goes below one then MATLAB steps out of the loop.
