\lab{Public Key Cryptography}{RSA}
\objective{Understand the fundamentals of RSA public key cryptography.}

How do you exchange information over unsecure, public channels fo communcations?
Historically, the answer has been to use some sort of cipher to obscure the meaning of a message.
Many ciphers use a secret key which was used to encrypt and decrypt any sent messages.
However, the problem then became how to keep the key secret and known only to the authorized recipients
If the key was obtained by any attackers, then the whole cipher was compromised.

Public key cryptography provides a way for each person to have their own keys.
There is no need to transmit any keys.
There are several public key crypto systems, but one of the most popular is RSA.
In the RSA system, each person has a public key and a private key.
The security of the system is founded in the difficulty of factoring large numbers.

To make RSA feasible, we need the following two results.
\section*{Fast Modular Exponentiation}
How would we compute $8^{3039} \pmod{38}$?
We understand that one possible way of computing it would be
\[
8*8*8*8*\dots (3034 \text{ more times}) *8 \pmod{38}
\]
Calculating $8^8 = 16777216$!
Multiplying by $8$ each time leads to an enormous number which will finally be modded by $38$.
This will yield the answer $8^{3039} \equiv 18 \pmod{38}$.
However, we end up taking the mod of a number that is $2746$ digits!
Imagine if our exponent was hundreds of digits!
We could quickly run out of memory.

We can avoid dealing such huge numbers by doing the mod after each multiplication.
Doing the mod after each multiplication means that the largest number we have to deal with in this problem is 240.
In general, we wouldn't have to work with anything larger than $37^{2}$.
However, if we still have large exponent, the loop doing the multiplication would still have to run $3039$ iterations.

\begin{problem}
Write a function that will naively calculate $b^{e} \pmod{m}$.
Do not use Python's built-in power function.
\label{prob:naivepower}
\end{problem}

We need a faster better way.
The most common method of modular exponentiation is the \emph{right-to-left binary method}.
The essence of this method is to use the binary representation of the exponent.
We can quickly calculate the power up to the nearest power of two of the exponent.
This we simply calculate the remaining part the naieve way.

Calculating a modular exponent in this way leads to a very desirable $O(\log_2 e)$ runtime performance, where $e$ is the exponent.

\begin{problem}
Write a function that will perform modular exponentiation using the right-to-left binary method.
Compare the performance of \ref{prob:naivepower} and your new implementation.
Note that Python's built-in function, \li{pow()}, performs this type of fast modular exponentiation.
We expect you to use it instead of your own implementation in the future.
\label{prob:rlbpower}
\end{problem}

\section*{Fermat's Little Theorem}
An integral part of RSA relies on working with large prime numbers.
Prime numbers are numbers that are only divisible by 1 and the number itself.
In the RSA algorithms, we need to start with two prime numbers, $p$ and $q$.
We multiply these numbers to get $n=pq$ which is our modulus.
However, if $n$ is not that big, we can easily factor it, defeating the whole purpose of RSA.
The security of RSA largely reduces down to how well we choose starting primes.
If we pick primes that are too small, we sacrifice security.
How do we pick large primes?
We can use sieves to generate primes, but doing so would take lots of time and lots of memory.
It would be better if we could choose a number and determine if it is prime or not.

Fortunately, mathematics has developed a few tools that can help us out.
One of the first steps was given by Fermat.
\begin{theorem}[Fermat's Little Theorem]
If $p$ is prime, then for a natural number $a$, $a^{p-1} - 1 \equiv 0 \pmod{p}$
\end{theorem}
But, to use this we have start with a prime number.
However, if we have a candidate for $p$ and it satisifies the condition can we say anything about $p$?
The answer is yes, but we cannot say that $p$ is prime.
We can only say that $p$ might be prime.
\begin{example}
Suppose $p=37$.
We want to know if $p$ is prime.  We choose $a=8$.
\[
8^{36} - 1 \equiv 0 \pmod{37} 
\]
We call $8$ the witness number.  Think of as, \emph{$8$ says that $37$ is prime}.
This is the probabilistic part of Fermat's theorem.
We try enough witness numbers until we are convinced that our candidate number is prime.
Trying enough prime numbers, we will eventually be convinced that $37$ is prime, which it is.
\end{example}

\begin{example}
Let's check the primality of $1729$.
We will use $a=145$ as a witness number.
\[
145^{1728} \equiv 1 \pmod{1729}
\]
It appears that $1729$ is prime.
Let's try another witness number, $a=583$.
\[
583^{1728} \equiv 1 \pmod{1729}
\]
Let's try one last witness number, $a=945$.
\[
945^{1728} \equiv 742 \not\equiv 1 \pmod{1729}
\]
We have a problem.  
Our candidate is not behaving like we would expect a prime to behave.
But several witnesses verified that our number was prime.  What happened?
Our number, $1729$ is not really prime.  Its prime factors are $7$, $13$, and $19$. 
It is a special number called a Carmichael number.
Because of these numbers, we have to be careful with Fermat's Little Theorem.
More stringent primality tests exist, but are beyond the scope of this lab.
\end{example}
We have to remember that the Fermat test is really a test for compositeness, not primality.

\begin{problem}
For any candidate prime number, $n$, all the integers $2 \geq a < n$ can serve as a witness number.
Each of these possible witness numbers will either prove the compositeness of $n$ or give no information.
If we randomly choose a witness number, we will select a number in one of these two sets.
Supposing we chose a good random number generator, we can become convinced to a very high degree that $n$ is likely prime with just a few witness numbers.

Write a function that will return the number of witnesses that prove the compositeness of $n$.
What is the ratio of these witness numbers to all the numbers $2 \geq a < n$?
Suppose we were to choose witness numbers at random.  
How many random witness numbers must we choose to be $99$\% sure that $n$ is prime?
Remember that each new witness number increases the confidence that $n$ is probably prime.
\label{prob:prime_confidence}
\end{problem}

\section*{RSA Algorithm}
Bob wants to send a secret message to Alice.
Eve is trying to intercept this message.
Bob and Alice agree to use RSA to exchange the secret message.
Bob and Alice need to generate their keys.
Each generate a pair of keys, a private and a public one.
Both Bob and Alice need to keep their private keys secret.
Anyone with Alice's public key can send her a message that only she can decrypt.
Bob obtains a copy of Alice's pulic key and encrypt his message using it.
Alice decrypts Bob's message using her private key, and reads the message.

RSA involves generating a private key and a public key.

We begin by finding two large prime numbers.
By large, we mean several hundred digits.

RSA can be succinctly summarised as the following operations for encryption and decryption respectively
\[
c \equiv m^e \pmod{n}
\]
\[
m \equiv c^d \pmod{n}
\]

\begin{problem}
Generate a public/private key pair.

We do this by first multiplying the two primes, $p$ and $q$, together to get $n=pq$.
This will act as the modulus for future operations.
Next we need to choose an encryption exponent, $e$, such that $1 < e < \phi(n)$ and $\gcd(e, \phi(n)) = 1$.
Euler's totient function is 
Then our decryption exponent, $d$, must naturally be an inverse of $e$ mod $\phi(n)$.
In other words, we need $de \equiv 1 \pmod{\phi(n)}$.
You will need to write your own function that finds the inverse of $e$ mod $\phi(n)$ using the Extended Euclidean Algorithm.
We distribute $(e, n)$ as the public key and need to make sure that $(d, n)$ remains private.
After the keys are created, we have no further need of $p$ or $q$.
\end{problem}

Before we can encrypt and decrypt a text message, we need to convert it to a number first.
It is useful to understand the code below.
\begin{lstlisting}
# Recipe from itertools module in standard library
def grouper(iterable, n, fillvalue=None):
    "Collect data into fixed-length chunks or blocks"
    # grouper('ABCDEFG', 3, 'x') --> ABC DEF Gxx
    args = [iter(iterable)] * n
    return izip_longest(fillvalue=fillvalue, *args)
    
def a2i(msg):
    '''Convert an ASCII message to an integer.'''
    # bytearray will give us the ASCII values for each character
    if not isinstance(msg, bytearray):
        msg = bytearray(msg)
    binmsg = []
    # convert each character to binary
    for c in msg:
        binmsg.append(bin(c)[2:].zfill(8))
    return int(''.join(binmsg), 2)

def i2a(msg):
    '''Convert an integer to an ASCII message'''
    # convert to binary first
    binmsg = bin(msg)[2:]
    # we need to pad the message so length is divisible by 8
    binmsg = "0"*(8-(len(binmsg)%8)) + binmsg
    msg = bytearray()
    for block in grouper(binmsg, 8):
        # convert block of 8 bits back to ASCII
        msg.append(int(''.join(block), 2))
    return msg
\end{lstlisting}

\begin{problem}
Encrypt the message: \emph{Simon sells seashells by the seashore.}
Let $p=83285677$ and $q=2848968679$.
Choose $e=65593$.
Verify your encryption is correct by decrypting the message and checking the result.
\end{problem}

\begin{warn}
The security of RSA depends on keeping the private key secret.  
If an attacker is able to obtain the private key, then the security of the entire algorithm is compromised.
When sharing keys, be absolutely sure that you are making only the public keys available and securely storing the private keys.
This is especially important with digital signatures since an attacker with a private key can impersonate the owner of that private key.
If a private key is compromised, generating a new RSA key is the only remedy.
\end{warn}


\subsection*{Digital Signatures}
Suppose Alice wants to confirm that Bob was really the sender of a message.
How would she do this?
Using the RSA algorithm, you can easily sign a message to verify authenticity.
This is done by encrypting the message using Alice's public key and then decrypting the result with Bob's private key.
Order is very important.  
Switching the two steps will result in the same message, but there are security vulnerablilities doing that way.

Another, more commonly used, method for signing messages is to sign a hash of the message.
Using a known good hash function, we hash the message (before encrypting), sign the hash, and send it along with the encrypted message.
The exchange would be as follows for the message \emph{Alice, the password is 123password.}
Suppose Bob's private and public keys respectively are (11407237936868069647229007915206939, 91340897874775971350115591460548127) and (54323, 91340897874775971350115591460548127).
Alice's private and public keys respectively are (200302109105861235857586283559067187, 274592902010736258475323322324306807) and (68923, 274592902010736258475323322324306807L).
First Bob would hash the message.  We will use SHA256 hash algorithm which is considered cryptographically secure.
The hash of the message (including the period symbol) is: b113b491a7ada55a888cf112005988ce5475d5fc909cafae770a4e08b8b3980c.
Bob would encrypt the message using Alice's public key and decrypt the hash.
The result would be 


\section{PyCrypto}
RSA encryption in Python can be accomplished very easily with a library called PyCrypto.
This library contains many secure hash functions, random number generators, and encryption classes.
Many programs use PyCrypto for their security needs.
\begin{warn}
Make certain that you are using the latest version of PyCrypto.
Security software is updated often to fix security vulnerablilities and bugs.
The current version of PyCrypto at the time of writing is version 2.6.1.
\end{warn}

The RSA module in PyCrypto is located in the PublicKey module.
Let's try a small example using PyCrypto.
The library allows us to explicity construct a key, or generate a key automatically.
\begin{lstlisting}
>>> from Crypto.PublicKey import RSA
>>> from Crypto import Random
>>> keypair = RSA.generate(2048) #generate a 2048-bit RSA key
>>> publickey = keypair.publickey()
>>> share_this = publickey.exportkey()
\end{lstlisting}

The RSA encryption and decryption methods on these keys are textbook approaches.
However, to increase security, we will want to pad the messages so every message encrypted with a particular will become exactly as large (in bits) as the key itself.
A commonly used padding algorithm is implemented in PyCrypto in the \li{Crypto.Cipher.PKCS1_OAEP} module.
\begin{lstlisting}
>>> from Crypto.Cipher import PKCS1_OAEP as oaep

# generate a new key from the original RSA key.
# This key can encrypt and decrypt
>>> paddedkey = oaep.new(keypair)
>>> encrypted = paddedkey.encrypt('hello world')
>>> paddedkey.decrypt(encrypted)
'hello world'
\end{lstlisting}

To sign the message we use PyCrypto's \li{Crypto.Signature.PKCS1_PSS} module.
We first need to hash the message using a cryptographically secure hash, such as SHA.
Just as only private keys can decrypt, only private keys can sign.
Public keys can decrypt and verify.
\begin{lstlisting}
>>> from Crypto.Signautre import PKCS1_PSS as pss
>>> from Crypto.Hash import SHA
>>> sigkey = pss.new(keypair)
>>> mhash = SHA.new("Hello World")
>>> sigkey.sign
\end{lstlisting}


\begin{problem}[Group Project]
Split into several groups of at least three individuals.
Each group member should generate a private and public keypair using PyCrypto.
Share your public keys with everyone in the group.
\begin{enumerate}
\item Encrypt and sign a message for someone in the group and send it to the entire group.
Attach the sender's name to the encrypted message (a claim that the message came from a particular sender).
In this exercise we will verify that the actual origin of the message is indeed the claimed origin.
Each group member should do the following:
\begin{enumerate}
\item Decrypt and verify the message addressed for you.
\item Attempt to decrypt and verify message addressed for someone else in the group.
\item Report the message and verified signature.
\end{enumerate}

\item Encrypt and sign a different message for someone in the group and send it, anonymously, to the entire group.
For example, if Bob encrypts a message and sends it to Alice, he should claim the message came from another individual.
Each group member should do the following:
\begin{enumerate}
\item Decrypt and verify the message address to you.
\item Report the message and verified signature.
\end{enumerate}
\end{enumerate}
\end{problem}
