\lab{Inverse Problems}{Inverse Problems}
\label{lab:inverse_problems}

% \objective{ }
An important concept in mathematics is the idea of a well-posed problem. The concept initially came from Jacques Hadamard. A mathematical problem is \textit{well posed} if 
\begin{enumerate}
	\item a solution exists, 
	\item that solution is unique, and 
	\item the solution is continuously dependent on the data in the problem. \label{inverse_problems:continuous_dependence}
\end{enumerate}
A problem that is not well posed is \textit{ill posed}. Notice that a problem may be well posed, and yet still possess the property that small changes in the data result in larger changes in the solution; in this case the problem is said to be \text{ill conditioned}, and has a large \text{condition number}.

Note that for a physical phenomena, a well-posed mathematical model would seem to be a necessary requirement! However, there are important examples of mathematical problems that are ill-posed. For example, consider the process of differentiation. Given an function $u$ together with its derivative $u'$, let $\tilde{u}(t) = u(t) +  \epsilon \sin(\epsilon^{-2}t)$ for some small $\epsilon > 0$. Then note that 
\begin{align*}
	\|u-\tilde{u}\|_{\infty} &= \epsilon,
\end{align*}
while
\begin{align*}
	\|u'-\tilde{u}'\|_{\infty} &= \epsilon^{-1}.
\end{align*}
Since a small change in the data leads to an arbitrarily large change in the output, differentiation is an ill-posed problem. And we haven't even mentioned numerically approximating a derivative!


\section*{Another look at heat flow through a rod}
Consider the ordinary differential equation, together with natural boundary conditions at the ends of the interval: 
\begin{align}
\begin{cases}
	-(au')' = f, & x \in (0,1),\\
	a(0)u'(0) = c_0, & a(1)u'(1) = c_1.
\end{cases} \label{inverse_problems:heat_flow}
\end{align}
This BVP can, for example, be used to describe the flow of heat through a rod. The boundary conditions would correspond the specifying the heat flux through the ends of the rod. $f(x)$ would then represent external heat sources along the rod, and $a(x)$ the density of the road at each point. 

Typically, the density $a(x)$ would be specified, along with any heat sources $f(x)$, and the (direct) problem is to solve for the steady-state heat distribution $u(x)$. Here we shake things up a bit: suppose the heat sources $f$ are given, and we can measure the heat distribution $u(x)$. Can we find the density of the rod? This is an example of a \textit{parameter estimation problem}.

Let us solve problem \eqref{inverse_problems:heat_flow} for the density $a(x)$ where 
$c_0 = 3/8,$ $c_1 = 5/4$, $u(x) = x^2 + x/2 + 5/16$, and 
\begin{align*}
	f &= \begin{cases}
		-6x^2 + 6x - 5/2 & x \leq 1/2,\\
		-1 & 1/2 < x \leq 1.
	\end{cases}
\end{align*}

\begin{lstlisting}
import numpy as np
import matplotlib.pyplot as plt

def f(x):
	out = -np.ones(x.shape)
	m = np.where(x<.5)
	out[m] = -6*x[m]**2. + 3.*x[m] - 1.
	return out

def u(x):
	return (x+1./4)**2. + 1./4

\end{lstlisting}

Subdivide $[0,1]$ into $N$ equal subintervals, and let $x_j = jh$, $j = 0, \ldots,N$, where $h = 1/N$.
Let $\phi_j(x)$ be the tent functions (used earlier in the finite element lab), given by 
\begin{align*}
	\phi_j(x) = \begin{cases}
(x - x_{j-1})/h  &  x \in [x_{j-1},x_j],\\
 (x_{j+1} - x)/h  &  x \in [x_{j},x_{j+1}],\\
0 & \text{ otherwise.}
\end{cases}
\end{align*}
We look for an approximation $a^h(x)$ of the form 
\begin{align*}
	a^h &= \sum_{j=0}^N \alpha_j \phi_j.
\end{align*}

Integrating \eqref{inverse_problems:heat_flow} from $0$ to $x$, we obtain
\begin{align}
\begin{split}
&{} \int_0^x -(au')'\, ds = \int_0^x f(s)\, ds,\\
&{} -[a(x)u'(x) - a(0)u'(0)] = \int_0^x f(s)\, ds,\\
&{} u'(x) = \frac{3/8 - \int_0^x f(s)\, ds}{a(x)}.
\end{split}
\end{align}
Thus for each $x_j$ we obtain
\begin{align*}
	u'(x_j) &= \frac{3/8 - \int_0^{x_j} f(s)\, ds}{a(x_j)},\\
	&= \frac{3/8 - \int_0^{x_j} f(s)\, ds}{\alpha_j}
\end{align*}

The coefficients $\alpha_j$ can be approximated by minimizing 
\begin{align*}
	\sum_{j=0}^N \left( \frac{3/8 - \int_0^{x_j} f(s)\, ds}{\alpha_j} - u'(x_j)  \right)^2.
\end{align*}

\begin{figure}
\centering
\includegraphics[width=\textwidth]{density_a.pdf}
\caption{The solution $a(x)$ computed by the example code.}
\label{fig:inverse_problems:exercise1}
\end{figure}

\begin{lstlisting}
from scipy.optimize import minimize

def integral_of_f(x):
	# out =  \int_0^x f(s) ds
	return out

def derivative_of_u(x):
	# out = derivative_of_u
	return out

x = np.linspace(0,1,11)
F, u_p = integral_of_f(x), derivative_of_u(x)

def least_squares(c):
	pass

guess = (1./4)*(3-x)
sol = minimize(least_squares,guess)

plt.plot(x,sol.x,'-ob',linewidth=2)
plt.show()
\end{lstlisting}

\begin{problem}
	Finish the previous code block to solve \eqref{inverse_problems:heat_flow} for $a(x)$ where $c_0 = 3/8,$ $c_1 = 5/4$, $u(x) = x^2 + x/2 + 5/16$, and 
	\begin{align*}
		f &= \begin{cases}
			-6x^2 + 6x - 5/2 & x \leq 1/2,\\
			-1 & 1/2 < x \leq 1.
		\end{cases}
	\end{align*}
	Produce the plot shown in Figure \ref{fig:inverse_problems:exercise1}.
\end{problem}

\begin{problem}
	Find the density function $a(x)$ satisfying 
	\begin{align}
	\begin{cases}
		-(au')' = -1, & x \in (0,1),\\
		a(0)u'(0) = 1, & a(1)u'(1) = 2.
	\end{cases} \label{inverse_problems:ill_posed}
	\end{align}
	where $u(x) = x + 1 + \epsilon \sin(\epsilon^{-2}x)$.  Using several values of $\epsilon  > 0.66049142$, plot the corresponding density $a(x)$ to demonstrate that the problem is ill-posed.
\end{problem}

\begin{figure}
\centering
\includegraphics[width=\textwidth]{ill_posed_density_a.pdf}
\caption{The density function $a(x)$ satisfying \eqref{inverse_problems:ill_posed} for $\epsilon = .8$.}
\label{fig:inverse_problems:exercise1}
\end{figure}


% \begin{align*}
% 	a(x) &=
% 	\begin{cases}
% 		x^2 - x + 3/4 & x \leq 1/2,\\
% 		\frac{1}{2} & 1/2 < x \leq 1.
% 	\end{cases}
% \end{align*}




% \begin{align*}
% 	au'(x) &=
% 	\begin{cases}
% 		(x^2 - x + 3/4)(2x - 1) & x \leq 1/2,\\
% 		\frac{1}{2}(2x - 1) & 1/2 < x \leq 1.
% 	\end{cases}
% \end{align*}

% \begin{align*}
% 	au'(x) &=
% 	\begin{cases}
% 		(x^2 - x + 3/4)(2x - 1) & x \leq 1/2,\\
% 		\frac{1}{2}(2x - 1) & 1/2 < x \leq 1.
% 	\end{cases}\\
% 	(au'(x))' &= 
% 	\begin{cases}
% 		6x^2 - 6x + 5/2 & x \leq 1/2,\\
% 		1 & 1/2 < x \leq 1.
% 	\end{cases}\\
% 	&= -f
% \end{align*}


% \begin{align*}
% 	u(x) &= (x+1/4)^2 + 1/4 = x^2 + x/2 + 5/16,\\
% 	f &= \begin{cases}
% 		-6x^2 + 6x - 5/2 & x \leq 1/2,\\
% 		-1 & 1/2 < x \leq 1.
% 	\end{cases}\\
% 	a(x) &=
% 	\begin{cases}
% 		x^2 - x + 3/4 & x \leq 1/2,\\
% 		\frac{1}{2} & 1/2 < x \leq 1.
% 	\end{cases}
% \end{align*}

% \begin{align}
% 	
% \end{align}
 
% \textit{}
% \footnote{}
% \label{inverse_problems:}
% \eqref{inverse_problems:}

% \begin{lstlisting}
% \end{lstlisting}

% \begin{figure}
% \centering
% \includegraphics[width=\textwidth]{.pdf}
% \caption{}
% \label{fig:inverse_problems}
% \end{figure}





