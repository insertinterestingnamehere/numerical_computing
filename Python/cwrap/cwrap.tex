\lab{Python}{Wrapping C and  C++}{Wrapping C and C++}
\label{lab:cwrap}

\objective{Learn to wrap C, and C++ code as well as other sorts of object files.}

There are several ways to interface with other languages from Python.
Python, since it is such a high-level language, for performance reasons, often needs to call functions from C and Fortran.
There is also an easy built in interface for R, which is a language that is used mostly for statistics.

\section*{Object Files and DLL's}

For C, C++, and Fortran, the desired functions or classes are compiled into an object file containing instructions in assembly that can be used by compilers of a variety of languages.
Object files use the \li{.o} extension.
Object files that can be used and modified by multiple programs are called "shared objects" and use the \li{.so} extension.
When C code has been compiled with the proper headers, etc. to interface with Python, it uses the \li{.pyd} extension on Windows and the \li{.so} extension on Unix-based operating systems.
The two extensions are essentially the same, except that Python object files are distinguished from normal object files on Windows.
When you compile anything that uses functions from object files, you must tell your compiler to include the contents of the object file.
The part of the compiler that forms these links between the code is called the linker.
If you do not give your compiler proper instructions for linking, it will raise a linking error.
If you are using GCC or MinGW, you can tell your compiler to include a file for linking by adding the flag 

DLL's are object files that are loaded by the operating system instead of by the compiler.
Th

