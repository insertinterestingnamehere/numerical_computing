\lab{Python}{Intro to pandas I}{Intro to pandas I}

In volumes 1 and 2, we solved data problems primarily using NumPy and SciPy.
We now turn our attention to a Python library that is more specifically built
for data analysis. Welcome to \emph{pandas}.

\section*{Data Structures in pandas}
Just as NumPy is built on the \li{ndarray} data structure suited for efficient scientific
and numerical computation, pandas is centered around a handful of core data structures
custom built for data analysis. These data structures include the \li{Series}, \li{DataFrame},
and \li{Panel}, which correspond roughly to one-, two-, and three-dimensional arrays.
We explore each in turn.

\subsection*{Series}
The \li{Series} is a one-dimensional array whose entries are labeled. The values of the array may be
any data type, including integers, strings, or general Python objects. Further, the array
need not be homogeneous. That is, it can hold values of different data types. Together,
the array values are referred to as the data of the \li{Series}.

The labels must consist of hashable types. They are commonly integers or strings.
Together, the labels are referred to as the index of the \li{Series}.

Thus, a \li{Series} consists of data and an index. The most basic way to initialize such an object
is as follows:
\begin{lstlisting}
>>> import pandas as pd
>>> s = pd.Series(data, index=index)
\end{lstlisting}
We needn't explicitly define the index. The default index is simply \li{np.arange(len(data))}.

For example, we can create a Series containing the integers from 9 down to 0:
\begin{lstlisting}
>>> s1 = pd.Series(range(9, -1, -1))
>>> s1.values    #the data
array([9, 8, 7, 6, 5, 4, 3, 2, 1, 0], dtype=int64)
>>> s1.index     #the labels
Int64Index([0, 1, 2, 3, 4, 5, 6, 7, 8, 9], dtype='int64')
>>> s1           #left column is index, right column is data
0    9
1    8
2    7
3    6
4    5
5    4
6    3
7    2
8    1
9    0
dtype: int64
\end{lstlisting}

Here is an example where we create customized labels:
\begin{lstlisting}
>>> data = np.random.randn(3)
>>> index = ['first', 'second', 'third']
>>> s2 = pd.Series(data, index=index)
>>> s2
first     1.661255
second   -0.033570
third    -2.185991
dtype: float64
\end{lstlisting}

We can create a \li{Series} having constant values in the following manner:
\begin{lstlisting}
>>> val = 4     #desired constant value of Series
>>> n = 6       #desired length of Series
>>> s3 = pd.Series(val, index=range(n))
>>> s3
0    4
1    4
2    4
3    4
4    4
5    4
dtype: int64
\end{lstlisting}

It is also possible to use a Python dict when creating a \li{Series}:
\begin{lstlisting}
>>> d = {'e1':93, 'e2':95, 'e3':87, 'e4':82, 'e5':94}
>>> s4 = pd.Series(d)
>>> s4
e1    93
e2    95
e3    87
e4    82
e5    94
dtype: int64
\end{lstlisting}
Note that we didn't need to specify the index: the keys of the dict are used as the index for the \li{Series}.
There are many more ways to create Series objects, and the reader is referred to the pandas
documentation.
\begin{problem}
Create the following pandas \li{Series}.

\begin{itemize}
\item Constant array with value -3, length 5. Labels should be the first five positive even integers.

\item Data is given by the dict \{`Bill':31, `Sarah':28, `Jane':34, `Joe':26\}.
\end{itemize}
\end{problem}

As with the \li{ndarray}, we can slice a \li{Series} using the usual syntax:
\begin{lstlisting}
>>> s4[:3]
e1    93
e2    95
e3    87
dtype: int64
\end{lstlisting}
Notice that both the data and the index are sliced in this manner.
We can also input \li{Series} with numerical data into many NumPy functions.
\begin{lstlisting}
>>> np.log(s4)
e1    4.532599
e2    4.553877
e3    4.465908
e4    4.406719
e5    4.543295
dtype: float64
\end{lstlisting}

A \li{Series} also has similarities with the Python dict.
We can access and alter the data using the index.
\begin{lstlisting}
>>> s4['e3']
87
>>> s4['e3'] = 99
>>> s4['e3']
99
\end{lstlisting}

We finish off this section by considering elementary vectorized operations with \li{Series}.
\begin{lstlisting}
>>> x = pd.Series(np.random.randn(4), index=['a', 'b', 'c', 'd'])
>>> x
a   -0.924259
b    0.767422
c    0.399212
d    0.130365
dtype: float64
>>> y = pd.Series(np.random.randn(5), index=['a', 'b', 'd', 'e', 'f'])
>>> y
a   -0.708301
b   -2.214516
d   -2.352364
e    0.789419
f   -0.859482
dtype: float64
\end{lstlisting}
Much as with the NumPy array, we can perform basic arithmetic operations on the entries of
a \li{Series} without the use of a for-loop. For example, we square the elements of \li{x} as
follows:
\begin{lstlisting}
>>> x**2
a    0.854254
b    0.588937
c    0.159370
d    0.016995
dtype: float64
\end{lstlisting}
In some cases, the \li{Series} allows for even greater flexibility than the NumPy array. For example,
we are able to add \li{Series} \li{x} and \li{y}, even though they have different lengths and labels:
\begin{lstlisting}
>>> z = x+y
>>> z
a   -1.632559
b   -1.447093
c         NaN
d   -2.221999
e         NaN
f         NaN
dtype: float64
\end{lstlisting}
Notice that the index of \li{z} is the \emph{union} of the index of \li{x} and the index of \li{y}.
For the labels shared by both \li{x} and \li{y} (namely \li{a}, \li{b}, and \li{d}), the corresponding
value of \li{z} is just the sum of the entries of \li{x} and \li{y}. In all other cases, the value of
\li{z} is \li{NaN}, which is the pandas type indicating a missing value. It is simple to omit missing
values from a series:
\begin{lstlisting}
>>> z.dropna()
a   -1.632559
b   -1.447093
d   -2.221999
dtype: float64
\end{lstlisting}




\subsection*{DataFrame}
The \li{DataFrame} data structure is a two-dimensional generalization of the \li{Series}. It can be viewed
as a tabular structure with labeled rows and columns. The row labels are collectively called the
index, and the column labels are collectively called the columns. An individual column in a
\li{DataFrame} object is a \li{Series}.

There are many ways to initialize a \li{DataFrame}. In the following, we build a \li{DataFrame} out of a
dict of \li{Series}.
\begin{lstlisting}
>>> d = {1:x, 2:y}
>>> df1 = pd.DataFrame(d)
>>> df1
	        1	        2
a	-0.924259	-0.708301
b	 0.767422	-2.214516
c	 0.399212	      NaN
d	 0.130365	-2.352364
e	      NaN	 0.789419
f	      NaN	-0.859482
\end{lstlisting}
Note that the index of this \li{DataFrame} is the union of the index of \li{Series} \li{x} and that of \li{Series} \li{y}.
The columns are given by the keys of the dict \li{d}. Since \li{x} doesn't have a label \li{e}, the
value in row \li{e}, column \li{1} is \li{NaN}. This same reasoning explains the other missing values as well.
Note that if we take the first column of the \li{DataFrame} and drop the missing values, we recover the \li{Series} \li{x}:
\begin{lstlisting}
>>> x == df1[1].dropna()
a    True
b    True
c    True
d    True
dtype: bool
\end{lstlisting}

We can also initialize a \li{DataFrame} using a NumPy array, creating custom row and column labels:
\begin{lstlisting}
>>> data = np.random.random((3,4))
>>> pd.DataFrame(data, index=['A','B','C'], columns=range(1,5))

            1	        2	        3	        4
A	 0.065646	 0.968593	 0.593394	 0.750110
B	 0.803829	 0.662237	 0.200592	 0.137713
C	 0.288801	 0.956662	 0.817915	 0.951016
3 rows     4 columns
\end{lstlisting}
As with Series, if we don't specify the index or columns, the default is \li{range(n)}, where \li{n}
is either the number of rows or columns.

It is also possible to create multi-indexed arrays, for example
\begin{lstlisting}
>>> grade=['eighth','ninth','tenth']
>>> subject=['math','science','english']
>>> myindex = pd.MultiIndex.from_product([grade,subject], names=['grade','subject'])
>>> myseries = pd.Series(np.random.randn(9),index=myindex)
>>> myseries
grade   subject
eighth  math       1.706644
        science   -0.899587
        english   -1.009832
ninth   math       2.096838
        science    1.884932
        english    0.413266
tenth   math      -0.924962
        science   -0.851689
        english    1.053329
dtype: float64
\end{lstlisting}

Multi-indexing is visually convenient, but not strictly necessary for most applications.
The interested reader is invited to explore the documentation to learn more.

A \li{DataFrame} behaves in certain respects like a dict of \li{Series}, where each column label and
the corresponding column form the key-value pairs.
We can access a desired column via its column label:
\begin{lstlisting}
>>> df1[2]
a   -0.708301
b   -2.214516
c         NaN
d   -2.352364
e    0.789419
f   -0.859482
Name: 2, dtype: float64
\end{lstlisting}
We can insert a new column or delete a column much as we would with a dict:
\begin{lstlisting}
>>> df1['product'] = df1[1] * df1[2]  #insert column containing the product of columns 1 and 2
>>> df1['constant'] = 'const'         #insert column containing constant data
>>> df1
            1	        2	  product	 constant
a	-0.924259	-0.708301	 0.654653	    const
b	 0.767422	-2.214516	-1.699469	    const
c	 0.399212	      NaN	      NaN	    const
d	 0.130365	-2.352364	-0.306666	    const
e	      NaN	 0.789419	      NaN	    const
f	      NaN	-0.859482	      NaN	    const
6 rows      4 columns
>>> del df1['constant']               #delete the constant column
            1	        2	  product	
a	-0.924259	-0.708301	 0.654653	
b	 0.767422	-2.214516	-1.699469	
c	 0.399212	      NaN	      NaN	
d	 0.130365	-2.352364	-0.306666	
e	      NaN	 0.789419	      NaN	
f	      NaN	-0.859482	      NaN	
6 rows      4 columns
\end{lstlisting}

We can also select specified rows, either using the row label, or its integer position:
\begin{lstlisting}
>>> df1.loc['b']        #select 2nd row via label
1          0.767422
2         -2.214516
product   -1.699469
Name: b, dtype: float64
>>> df1.iloc[1]         #select 2nd row via integer position
1          0.767422
2         -2.214516
product   -1.699469
Name: b, dtype: float64
\end{lstlisting}
We can slice rows much as we do with NumPy arrays:
\begin{lstlisting}
>>> df1[1:4]
	        1	        2	  product
b	 0.767422	-2.214516	-1.699469
c	 0.399212	      NaN	      NaN
d	 0.130365	-2.352364	-0.306666
3 rows     3 columns
\end{lstlisting}

A particularly slick way to slice rows is by boolean indexing, in which a criterion for selecting rows is given.

\begin{lstlisting}
>>> df #Here's a data frame
           A         B         C strcol
0   0.628435  1.910799  0.194874   asdf
1   0.785914  0.412298 -1.813564   asdf
2   1.124251  0.929658 -0.580788   asdf
3   0.978128  1.126996 -0.063130   asdf
4   0.529262  1.063394  0.304336   asdf
5  -0.310556 -0.012804 -1.042040   asdf
6  -0.561364  0.382025 -1.103540   asdf
7   1.335368  0.676128  0.101528   asdf
8  -0.315294 -0.273289 -0.009816   asdf
9   0.179279  0.743868  0.294675   asdf
10  0.694140 -0.808863  1.243764   asdf
11 -0.811642  2.499150  2.100622   asdf
12  0.834592 -0.197040 -0.133127   asdf

[212 rows x 4 columns]
>>> df[df['C']>0] #Get all the rows having positive entry in column C
           A         B         C strcol
0   0.628435  1.910799  0.194874   asdf
4   0.529262  1.063394  0.304336   asdf
7   1.335368  0.676128  0.101528   asdf
9   0.179279  0.743868  0.294675   asdf
10  0.694140 -0.808863  1.243764   asdf
11 -0.811642  2.499150  2.100622   asdf
[6 rows x 4 columns]
\end{lstlisting}

Boolean indexing is particularly useful for selecting data that satisfy certain criteria,
including when it comes time for data cleaning. One caution: if multiple statements are used
to select rows, they must be separated by parentheses:

\begin{lstlisting}
>>> df[(df['C']>0) & (df['A']> 0.3)] #Note the parentheses
           A         B         C strcol
0   0.628435  1.910799  0.194874   asdf
4   0.529262  1.063394  0.304336   asdf
7   1.335368  0.676128  0.101528   asdf
10  0.694140 -0.808863  1.243764   asdf
15  0.594757 -0.659265  0.023030   asdf

[5 rows x 4 columns]
>>> df[df['C']>0 & df['A']> 0.3] #Forgot parentheses. This will cause an error.
\end{lstlisting}

\subsection*{Panel}
The third fundamental object in pandas is the \li{Panel}. It is analogous to a three-dimensional
array and was designed for dealing with panel data, which is common in economic research, among
other fields. In economics, for instance, one might want to track information for several
countries over multiple years. \li{Panel} is the least-commonly used of the three fundamental pandas
datatypes, but it is worth knowing about.

The three axes of a \li{Panel} are \li{items}, \li{major_axis}, and \li{minor_axis}, so a panel can
be initialized from a 3-dimensional array as follows:

\begin{lstlisting}
>>> pan = pd.Panel(np.random.randn(3,2,3), items=['1','2','3'], major_axis=['A','B'], minor_axis=['a','b','c'])
\end{lstlisting}

To see what this looks like, use the command \li{to_frame}, which lists out the data as a
multi-indexed \li{DataFrame}.

\begin{lstlisting}
>>> pan
<class 'pandas.core.panel.Panel'>
Dimensions: 3 (items) x 2 (major_axis) x 3 (minor_axis)
Items axis: 1 to 3
Major_axis axis: A to B
Minor_axis axis: a to c
>>> pan.to_frame()
                    1         2         3
major minor
A     a     -0.252654 -0.913452 -0.901322
      b      1.933009 -0.262968 -0.467978
      c      0.484946 -0.214478 -0.061904
B     a     -0.959360  2.475628  0.900854
      b      0.257859 -0.879048 -1.243879
      c     -0.103787 -1.141230  0.222686

[6 rows x 3 columns]
\end{lstlisting}
Note that the \li{items} field in the \li{Panel} constructor is analogous to the columns in a \li{DataFrame},
while the two axes arguments behave like a multi-dimensional index.

Alternatively, a panel can be constructed from a dict of \li{DataFrame} objects, as follows:

\begin{lstlisting}
>>> df1 = pd.DataFrame(np.random.randn(2, 3))
>>> df2 = pd.DataFrame(np.random.randn(3, 2))
>>> paneldata = pd.Panel({'2001' : df1, '2011' : df2})
>>> paneldata.to_frame(filter_observations=False)
                        2001	     2011
major	minor		
0	    0	        1.190182	-0.578198
        1	        0.728126	-0.848281
        2	       -0.103827	 NaN
1	    0	       -0.278637	-0.454951
        1	        0.128356	-0.828686
        2	       -0.439030	 NaN
2	    0	        NaN	        -2.167289
        1	        NaN	         0.428068
        2	        NaN	         NaN

[9 rows � 2 columns]
\end{lstlisting}

The operation above deals nicely with the asymmetric data, inserting \li{NaN} where needed.
If you want to view the \li{Panel} using \li{to_frame}, however, the default is to drop the asymmetric parts and
reduce to a $2\times 2\times 2$ structure. The argument \li{filter_observations=False} prevents this.
Another way to access the data in a particular item is simply to use \li{paneldata['2011']}.

Many of the functions of the \li{DataFrame} type perform similarly on the \li{Panel} type. These include transposition,
inserting and deleting items, and accessing data. Another way to use the \li{Panel} is to use \li{squeeze} to

\section*{Viewing Data}

\subsection*{Plotting}
Plotting is often a much more effective way to view and gain understanding of a dataset than simply
viewing the raw numbers. Fortunately, pandas interfaces well with matplotlib, enabling relatively
painless data visualization functionality.

We start by plotting a \li{Series}. Doing so is easy, as the \li{Series} object is equipped with its own plot
function.
Let's start with visualizing a simple random walk. By way of background, a \emph{random walk} is a
stochastic process used to model a non-deterministic path through some space. It is a useful construct in
many fields, from explaining the motion of a molecule as it travels through a liquid to modeling the fluctuations
of stock prices. Here we will simulate a one-dimensional symmetric random walk on the integers, which can be
described as follows.
\begin{itemize}
  \item[Step 1] Start at 0.
  \item[Step 2] Flip a fair coin.
  \item[Step 3] If heads, move one unit to the right. Otherwise, move one unit to the left.
  \item[Step 4] Go to Step 2.
\end{itemize}
How can we simulate this random walk efficiently? Note that the walk is really characterized by the outcomes
of the coin flip. If we represent heads by the number $1$ and tails by $-1$, then our position at a given moment
is just the cumulative sum of all previous outcomes. Below, we simulate a sequence of coin flips, build the
resulting random walk, and plot the outcome.
\begin{lstlisting}
>>> import matplotlib.pyplot as plt
>>> N = 1000        #length of random walk
>>> s = np.zeros(N)
>>> s[1:] = np.random.binomial(1, .5, size=(N-1,))*2-1 #coin flips
>>> s = pd.Series(s)
>>> s = s.cumsum()  #random walk
>>> s.plot()
>>> plt.ylim([-50,50])
>>> plt.show()
\end{lstlisting}

The random walk is shown in Figure \ref{pandas:randomWalk}.

\begin{figure}
\centering
\includegraphics[width=.7 \textwidth]{randomWalk.pdf}
\caption{Random walk of length 1000.}
\label{pandas:randomWalk}
\end{figure}

\begin{problem}
Create five random walks of length 100, and plot them together.

Next, create a ``biased" random walk by changing the coin flip probability of head from 0.5 to 0.51.
Plot this biased walk with lengths 100, 10000, and then 100000. Notice the definite trend that emerges.
Your results should be comparable to those in Figure \ref{pandas:biasedRandomWalk}.
\end{problem}

\begin{figure}
\centering
\includegraphics[width=.7 \textwidth]{biasedRandomWalk.pdf}
\caption{Biased random walk of length 100 (above) and 10000 (below).}
\label{pandas:biasedRandomWalk}
\end{figure}

Using DataFrames, one can also plot one column against another.

\begin{lstlisting}
>>> xvals = pd.Series(np.sqrt(np.arange(1000)))
>>> yvals = pd.Series(np.random.randn(1000).cumsum())
>>> df = pd.DataFrame({'xvals':xvals,'yvals':yvals}) #Put in a Dataframe
>>> df.plot(x='xvals',y='yvals') #Plot, specifying which column is to be used to x and y values.
>>> plt.show()
\end{lstlisting}

The result is displayed in Figure \ref{pandas:dfPlot}.

\begin{figure}
\centering
\includegraphics[width=.7 \textwidth]{dfPlot.pdf}
\caption{ Graph generated when one coordinate is taken from the xvals column and the other from the yvals column.}
\label{pandas:dfPlot}
\end{figure}

A variety of other types of plots are possible. One of the more useful plots when trying to estimate or
visualize the distribution of data is a histogram. The code listed below demonstrates how to generate
a histogram for each column in a DataFrame, with the result shown in Figure \ref{pandas:histogram}.

\begin{lstlisting}
>>> col1 = pd.Series(np.random.randn(1000))         #normal distribution
>>> col2 = pd.Series(np.random.gamma(5, size=1000)) #gamma distribution
>>> df = pd.DataFrame({'normal':col1, 'gamma':col2})
>>> df.hist()
>>> plt.plot()
\end{lstlisting}

\begin{figure}
\centering
\includegraphics[width=.7 \textwidth]{histogram.pdf}
\caption{Histogram of two columns of a DataFrame.}
\label{pandas:histogram}
\end{figure}

\subsection*{SQL Operations in pandas}
The DataFrame, being a tabular data structure, bears an obvious resemblance to a standard relational
database table. SQL is the standard for working with relational databases, and in this section we will
explore how pandas accomplishes some of the same tasks as SQL. The SQL-like functionality of pandas is
one of its big advantages, since it can eliminate the need to switch between programming languages
for different tasks. Within pandas we can handle both the querying \emph{and} data analysis.

For the following examples, we will use the following data:
\begin{lstlisting}
>>> #build toy data for SQL operations
>>> name = ['Bill', 'Alice', 'Joe', 'Jenny', 'Ted', 'Taylor', 'Tracy', 'Morgan', 'Liz']
>>> sex = ['M', 'F', 'M', 'F', 'M', 'F', 'M', 'M', 'F']
>>> age = [20,21,18,22,19,20,20,19,20]
>>> rank = ['Sp', 'Se', 'Fr', 'Se', 'Sp', 'J', 'J', 'J', 'Se']
>>> ID = range(9)
>>> aid = ['y', 'n', 'n', 'y', 'n', 'n', 'n', 'y', 'n']
>>> GPA = [3.8, 3.5, 3.0, 3.9, 2.8, 2.9, 3.8, 3.4, 3.7]
>>> mathID = [0, 1, 5, 6, 3]
>>> mathGd = [4.0, 3.0, 3.5, 3.0, 4.0]
>>> major = ['y', 'n', 'y', 'n', 'n']
>>> studentInfo = pd.DataFrame({'ID':ID,'Name':name, 'Sex':sex, 'Age':age, 'Class':rank})
>>> otherInfo = pd.DataFrame({'ID':ID, 'GPA':GPA, 'Financial_Aid':aid})
>>> mathInfo = pd.DataFrame({'ID':mathID, 'Grade':mathGd, 'Math_Major':major})
\end{lstlisting}

Let's first look at the pandas equivalent of some SQL SELECT statements.
\begin{lstlisting}
>>> # SELECT ID, Age FROM studentInfo
>>> studentInfo[['ID', 'Age']]

>>> # SELECT ID, GPA FROM otherInfo WHERE Financial_Aid = 'y'
>>> otherInfo[otherInfo['Financial_Aid']=='y'][['ID', 'GPA']]

>>> # SELECT Math_Major, COUNT(*) FROM mathInfo GROUP BY Math_Major
>>> print mathInfo.groupby('Math_Major').size()
\end{lstlisting}

\begin{problem}
The example above shows how to implement a simple WHERE condition, and it is easy
to have a more complex expression. Simply enclose each simple condition by parentheses,
and use the standard boolean operators \li{\&} (AND), \li{\|} (OR), and \li{\~} (NOT) to
connect the conditions appropriately. Use pandas to execute the following query:
\begin{lstlisting}[language=SQL]
SELECT ID, Name from studentInfo WHERE Age > 19 AND Sex = 'M'
\end{lstlisting}
\begin{lstlisting}
A: studentInfo[(studentInfo['Age']>19)&(studentInfo['Sex']=='M')][['ID', 'Name']]
\end{lstlisting}
\end{problem}

Next, let's look at JOIN statements. In pandas, this is done with the \li{merge} function,
which takes as arguments the two DataFrame objects to join, as well as keyword arguments specifying
the column on which to join, along with the type (left, right, inner, outer).

\begin{lstlisting}
>>> # SELECT * FROM studentInfo INNER JOIN mathInfo ON studentInfo.ID = mathInfo.ID
>>> pd.merge(studentInfo, mathInfo, on='ID') # INNER JOIN is the default
   Age Class  ID    Name Sex  Grade Math_Major
0   20    Sp   0    Bill   M    4.0          y
1   21    Se   1   Alice   F    3.0          n
2   22    Se   3   Jenny   F    4.0          n
3   20     J   5  Taylor   F    3.5          y
4   20     J   6   Tracy   M    3.0          n
[5 rows x 7 columns]

>>> # SELECT GPA, Grade FROM otherInfo FULL OUTER JOIN mathInfo on otherInfo.ID = mathInfo.ID
>>> pd.merge(otherInfo, mathInfo, on='ID', how='outer')[['GPA', 'Grade']]
   GPA  Grade
0  3.8    4.0
1  3.5    3.0
2  3.0    NaN
3  3.9    4.0
4  2.8    NaN
5  2.9    3.5
6  3.8    3.0
7  3.4    NaN
8  3.7    NaN
[9 rows x 2 columns]
\end{lstlisting}

It is sometimes desirable to join to DataFrames on their indexes rather than on a particular column.
The \li{join} method makes this operation convenient.
\begin{lstlisting}
>>> #create new DataFrame
>>> sibs = [0, 1, 0, 5, 2, 9]
>>> sibInfo = pd.DataFrame({'Siblings':sibs})
>>> sibInfo
   Siblings
0         0
1         1
2         0
3         5
4         2
5         9
[6 rows x 1 columns]

>>> #now join studentInfo with sibInfo
>>> studentInfo.join(sibInfo)
   Age Class  ID    Name Sex  Siblings
0   20    Sp   0    Bill   M         0
1   21    Se   1   Alice   F         1
2   18    Fr   2     Joe   M         0
3   22    Se   3   Jenny   F         5
4   19    Sp   4     Ted   M         2
5   20     J   5  Taylor   F         9
6   20     J   6   Tracy   M       NaN
7   19     J   7  Morgan   M       NaN
8   20    Se   8     Liz   F       NaN
[9 rows x 6 columns]
\end{lstlisting}
The default is a left join, but this can be altered.
When attempting to join to DataFrames that share common columns on their indexes,
we must rename the columns to avoid contradictions. We do this by simply appending
characters to the existing column names, specified by the \li{lsuffix} and \li{rsuffix}
parameters (for the left and right DataFrames, respectively).
\begin{lstlisting}
>>> #join studentInfo and mathInfo on indexes
>>> studentInfo.join(mathInfo, lsuffix='_left', rsuffix='_right', how='inner')
   Age Class  ID_left   Name Sex  Grade  ID_right Math_Major
0   20    Sp        0   Bill   M    4.0         0          y
1   21    Se        1  Alice   F    3.0         1          n
2   18    Fr        2    Joe   M    3.5         5          y
3   22    Se        3  Jenny   F    3.0         6          n
4   19    Sp        4    Ted   M    4.0         3          n
[5 rows x 8 columns]
\end{lstlisting}

The final SQL-like operation we will discuss is the UNION, or concatenation of DataFrames.
In the simplest setting, this operation is useful when we have two DataFrames that have the
same columns but possibly different rows.
\begin{lstlisting}
>>> #create another df holding more math info
>>> mathID2 = [0, 5, 2, 4]
>>> mathGd2 = [4.0, 3.5, 2.0, 3.0]
>>> major2 = ['y', 'y', 'n', 'y']
>>> mathInfo2 = pd.DataFrame({'Grade':mathGd2, 'ID':mathID2, 'Math_Major':major2})
>>> pd.concat([mathInfo, mathInfo2], ignore_index=True).drop_duplicates()
   Grade  ID Math_Major
0    4.0   0          y
1    3.0   1          n
2    3.5   5          y
3    3.0   6          n
4    4.0   3          n
7    2.0   2          n
8    3.0   4          y
[7 rows x 3 columns]
\end{lstlisting}
The \li{ignore_index=True} argument means we discard the indexes of the two input DataFrames and create
a new one for the concatenated DataFrame. The \li{drop_duplicates} method simply drop duplicate rows.

When you find yourself unsure of how to carry out a more involved SQL-like operation, the online
pandas documentation will be of great service.

ADD IN ABOUT SAVING TO SQL DATABASE


\begin{lstlisting}
>>> df1 #First data frame
      DATA1     DATA2  EMPLOYEEID
0  0.763628  0.671691           1
1  0.635869  1.486546           2
2  1.917649 -0.432400           3
3 -0.679496  0.794938           4
4 -0.201446  1.936673           5

[5 rows x 3 columns]
>>> df2 #Second data frame
      DATA3     DATA4  EMPLOYEEID
0 -1.194058  0.814376           2
1  1.068904 -0.289241           3
2  0.099272  0.595524           4
3  0.688357 -0.207746           5
4  0.292399  0.504268           6
5 -0.259372  1.159809           7

[6 rows x 3 columns]
>>> pd.merge(df1,df2,on='EMPLOYEEID') #Use the merge function to do SQL joins. The default is inner join.
      DATA1     DATA2  EMPLOYEEID     DATA3     DATA4
0  0.635869  1.486546           2 -1.194058  0.814376
1  1.917649 -0.432400           3  1.068904 -0.289241
2 -0.679496  0.794938           4  0.099272  0.595524
3 -0.201446  1.936673           5  0.688357 -0.207746

[4 rows x 5 columns]
>>> pd.merge(df1,df2,on='EMPLOYEEID',how='outer') #The how keyword allows for the other major join types. Notice how NaN is used to fill missing spots.
      DATA1     DATA2  EMPLOYEEID     DATA3     DATA4
0  0.763628  0.671691           1       NaN       NaN
1  0.635869  1.486546           2 -1.194058  0.814376
2  1.917649 -0.432400           3  1.068904 -0.289241
3 -0.679496  0.794938           4  0.099272  0.595524
4 -0.201446  1.936673           5  0.688357 -0.207746
5       NaN       NaN           6  0.292399  0.504268
6       NaN       NaN           7 -0.259372  1.159809

[7 rows x 5 columns]
>>> pd.merge(df1,df2,on='EMPLOYEEID',how='left') #Left outer join
      DATA1     DATA2  EMPLOYEEID     DATA3     DATA4
0  0.763628  0.671691           1       NaN       NaN
1  0.635869  1.486546           2 -1.194058  0.814376
2  1.917649 -0.432400           3  1.068904 -0.289241
3 -0.679496  0.794938           4  0.099272  0.595524
4 -0.201446  1.936673           5  0.688357 -0.207746

[5 rows x 5 columns]
>>> pd.merge(df1,df2,on='EMPLOYEEID',how='right') #Right outer join
      DATA1     DATA2  EMPLOYEEID     DATA3     DATA4
0  0.635869  1.486546           2 -1.194058  0.814376
1  1.917649 -0.432400           3  1.068904 -0.289241
2 -0.679496  0.794938           4  0.099272  0.595524
3 -0.201446  1.936673           5  0.688357 -0.207746
4       NaN       NaN           6  0.292399  0.504268
5       NaN       NaN           7 -0.259372  1.159809

[6 rows x 5 columns]
\end{lstlisting}

The SQL command \li{select} also has an easy substitute.

\begin{lstlisting}
>>> df1[['EMPLOYEEID','DATA1']] #Select only two of the columns
   EMPLOYEEID     DATA1
0           1  0.763628
1           2  0.635869
2           3  1.917649
3           4 -0.679496
4           5 -0.201446

[5 rows x 2 columns]
>>> df1[['EMPLOYEEID','DATA1']].head(2) #Take only the first two rows
   EMPLOYEEID     DATA1
0           1  0.763628
1           2  0.635869

[2 rows x 2 columns]
>>> df1[['EMPLOYEEID','DATA1']].tail(3) #Take only the last three rows
   EMPLOYEEID     DATA1
2           3  1.917649
3           4 -0.679496
4           5 -0.201446

[3 rows x 2 columns]
\end{lstlisting}

\section*{Manipulating Data}

Pandas is particularly well-suited to handling missing and anomalous data. The pandas default for a missing value is \li{NaN}. In basic arithmetic operations, missing values are dealt with in a conservative manner:

\begin{lstlisting}
>>> myseries2 = pd.Series(np.random.randn(9),index=myindex)
>>> myseries2['eighth','english']='NaN'
>>> myseries2
grade   subject
eighth  math       0.191243
        science    0.554761
        english         NaN
ninth   math      -0.526643
        science    2.398025
        english    1.082043
tenth   math       1.396143
        science   -0.850942
        english   -1.378125
dtype: float64
>>> myseries
grade   subject
eighth  math       1.706644
        science   -0.899587
        english   -1.009832
ninth   math       2.096838
        science    1.884932
        english    0.413266
tenth   math      -0.924962
        science   -0.851689
        english    1.053329
dtype: float64
>>> myseries+myseries2
grade   subject
eighth  math       1.897887
        science   -0.344826
        english         NaN
ninth   math       1.570195
        science    4.282957
        english    1.495309
tenth   math       0.471181
        science   -1.702631
        english   -0.324795
dtype: float64
\end{lstlisting}

Other functions, such as $.sum()$ and $.mean()$ treat NaN as zero.

\begin{lstlisting}
>>> myseries2.mean()
0.35831305133218083
\end{lstlisting}

There are several ways to deal with NaN entries. One is to simply drop them.

\begin{lstlisting}
>>> myseries2.dropna()
grade   subject
eighth  math       0.191243
        science    0.554761
ninth   math      -0.526643
        science    2.398025
        english    1.082043
tenth   math       1.396143
        science   -0.850942
        english   -1.378125
dtype: float64
\end{lstlisting}

Notice that the above operation did not chance myseries2, but rather created a new series without the NaN entries. To change myseries2 directly, use the inplace option

\begin{lstlisting}
>>> myseries3.dropna(inplace=True)
\end{lstlisting}

Alternately, rather than dropping NaN entries, it might be suitable to replace them with the mean or some other value.

\begin{lstlisting}
>>> myseries2.fillna(myseries2.mean())
grade   subject
eighth  math       0.191243
        science    0.554761
        english    0.358313
ninth   math      -0.526643
        science    2.398025
        english    1.082043
tenth   math       1.396143
        science   -0.850942
        english   -1.378125
dtype: float64
>>> myseries2.mean()
0.35831305133218083
\end{lstlisting}

Other ``cleaning'' operations include transposing and sorting. ``Transposing'' means flipping the rows and columns (see example below), and sorting is simply ordering the data in some meaningful way.

\begin{lstlisting}
>>> df #Original dataframe
            0         1         2         3         4
a 0 -1.749534  0.763194  1.039742 -1.179616  2.128509
b 1 -0.956593 -0.032001  1.249905  0.418494 -0.029238
c 2  0.485885 -1.539459 -0.602270  1.939523  0.291494
d 3 -1.328571       NaN  0.481167  1.623820  0.228658
e 4  1.731380  1.395385 -0.207489  0.253467  0.348348

[5 rows x 5 columns]
>>> df.transpose() #Transposed dataframe. See how the rows and columns are switched?
          a         b         c         d         e
          0         1         2         3         4
0 -1.749534 -0.956593  0.485885 -1.328571  1.731380
1  0.763194 -0.032001 -1.539459       NaN  1.395385
2  1.039742  1.249905 -0.602270  0.481167 -0.207489
3 -1.179616  0.418494  1.939523  1.623820  0.253467
4  2.128509 -0.029238  0.291494  0.228658  0.348348

[5 rows x 5 columns]

#Now let's sort the data
>>> mydata #Original DataFrame
          0         1         2         3
a -1.150876  0.201994 -0.447484 -0.204217
b -0.171968 -0.303143 -0.605779  0.499296
c -0.647877 -0.010989  0.881816 -0.783891
d  1.500841  0.698213  0.963573  0.964841
e -0.274085 -0.124850  0.947559  0.697708

[5 rows x 4 columns]
>>> mydata.sort([1]) #Sorted by the entries of column 1
          0         1         2         3
b -0.171968 -0.303143 -0.605779  0.499296
e -0.274085 -0.124850  0.947559  0.697708
c -0.647877 -0.010989  0.881816 -0.783891
a -1.150876  0.201994 -0.447484 -0.204217
d  1.500841  0.698213  0.963573  0.964841

[5 rows x 4 columns]
\end{lstlisting}

With these basic ideas and operations in hand, you'll be prepared to deal with the missing values and formatting indiscrepancies that occur in most real-life data sets.

\section*{Analyzing Data}

Of course, the whole point of \li{pandas} is to help you \emph{analyze} the data once it's inputted. Several upcoming labs will involve practicing different ways of using pandas for data analysis. We'll just cover the very basics right now.

A good first step is just to determine the datatypes of each column in a data frame:

\begin{lstlisting}
>>> df['a'][0]='this' #Changing this entry to make the DataFrame more interesting.
>>> df
           a         b         c         d         e
0       this -0.171968 -0.647877  1.500841 -0.274085
1  0.2019935 -0.303143 -0.010989  0.698213 -0.124850
2 -0.4474842 -0.605779  0.881816  0.963573  0.947559
3 -0.2042165  0.499296 -0.783891  0.964841  0.697708

[4 rows x 5 columns]
>>> df.info() #Asking for the contents of each column.
<class 'pandas.core.frame.DataFrame'>
Int64Index: 4 entries, 0 to 3
Data columns (total 5 columns):
a    4 non-null object #Notice this one.
b    4 non-null float64
c    4 non-null float64
d    4 non-null float64
e    4 non-null float64
dtypes: float64(4), object(1)
\end{lstlisting}

It is worth noting that the datatypes in each column need not be homogenous, in which case, the classification is given simply as ``object.''

Another quick, easy analysis is given by the \li{describe} method.

\begin{lstlisting}
>>> df.describe()
              b         c         d         e
count  4.000000  4.000000  4.000000  4.000000 #Number of entries per column
mean  -0.145398 -0.140235  1.031867  0.311583 #Mean
std    0.466608  0.760107  0.336857  0.601952 #Standard deviation
min   -0.605779 -0.783891  0.698213 -0.274085 #The five number summary
25%   -0.378802 -0.681881  0.897233 -0.162158
50%   -0.237555 -0.329433  0.964207  0.286429
75%   -0.004152  0.212212  1.098841  0.760171
max    0.499296  0.881816  1.500841  0.947559
\end{lstlisting}

There are also easy commands to get covariance matrices (.cov()), correlation matices (.corr()), and almost any other basic statistical info.



\section*{Data I/O}

Before we can use \li{pandas} for anything, we need to have some data, and when we're done using the data, we need to store it in a convenient place and format. In other words, we need to do data input/output, or I/O. The process of input is called \textit{reading}, and the process of output is called \textit{writing}. Reading usually takes data from another file, such as a CSV file, and turns it into a \li{python} data type. Writing then creates or updates another file with the most recent configuration of the data.\\
\indent The \li{pandas} package supports reading many different file types, using functions such as \li{pd.read_csv(),pd.read_stata(), pd.read_excel()}, etc. The output is a DataFrame object, although there is an option, \li{squeeze}, to make the output a Series object. For instance, to read a \li{.csv} file, one could use the following code:
\begin{lstlisting}
pd.read_csv('FakeData.csv', index_col=0)
\end{lstlisting}
Here, the first argument gives the file name, and the second specifies that the first column be used to index the table. If the second argument is left off, python will generate its own index column. Here's the output:

\begin{lstlisting}
           date  time  room instructor
Psych  20120101  1200    12    Carlson
Math   20120102   100    14      Smith
Bio    20120378   400    17   Hamilton

[3 rows x 4 columns]
\end{lstlisting}

Compare this with the output we get without \li{pandas}
\begin{lstlisting}
>>> print(open('C:\...\FakeData.csv').read())
,date,time,room,instructor
Psych,20120101,1200,12,Carlson
Math,20120102,100,14,Smith
Bio,20120378,400,17,Hamilton
\end{lstlisting}

Not nearly as structured or easy to interpret!

Many real-world data sets contain imperfections that you will need to ``clean,'' and some of these issues will need to be cleared up at the time that the data is read in. For instance, while \li{pandas} has some ability to read in data sets with missing values, it will generally need your help (possibly in the form of the \li{error_bad_lines} option) to handle a line of data with extra values. Because there are so many possible problems with reading in real data, the reader is encouraged to look through the documentation when specific issues arise.\\
\indent Writing tends to be much less messy since the data is already going to be in a format suitable for computations. The only details that typically come up are in specifying what formatting should be put on the output file. There are many options attached to the write commands, such as whether to include headers, and which options are specified depends on the future use that the data will have. Most of the time, a write command will just look like this:
\begin{lstlisting}
mydatatable.to_csv('FakeData1.csv')
\end{lstlisting}

\begin{problem}
Download the dataset at \url{http://research.stlouisfed.org/fred2/series/ACILACB#}, which is some loan data put together by the St. Louis Fed. It will be an Excel file. Use the command \li{pd.read_excel} to read in the file. Notice that this command takes two arguments, the path to the file and the name of the sheet to be read. Look at the resulting DataFrame. It probably won't look very good due to the presence of extra information in the file. One easy way to fix this is just to open the Excel file, manually remove the extra material, and read the file again. Do this, and look at the results. The DataFrame should look much neater now. Use the \li{describe} command to get the mean of the data.\\

Now, the data presented here is quarterly. Reduce the data by removing all except for the January data. Write this to a file, ``fradgraph2.xls,'' and turn the file in, along with the mean calculated, as above.
\end{problem}













































