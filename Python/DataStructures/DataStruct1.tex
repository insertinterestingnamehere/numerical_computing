\lab{Python}{Data Structures I}{Data Structures I}
\label{lab:Python_DataStructures}
\objective{Learn about data structures}

We can store data in computer in a variety of ways.
The most basic way we have to store data is via primitive datatypes.
These datatypes are booleans, strings, floats, and integers.
Most information that we care to store is in one of these forms.
However, for storing large amounts information, these primitive types can quickly become unwieldy.
We can create more natural ways to store data.  We use these primitive datatypes as building blocks along with arrays (or lists in Python).  The more complex data structures are called \emph{abstract data types}.
Python has a couple of the more common abstract data types such as dictionaries, sets, and lists.

\section{Dictionaries, Sets, and Lists}
Dictionaries store (key, value) elements.  The keys must be unique and immutable.  Order is not important with dictionaries.  Any semblance of order is a result of traversing the underlying data structure.

Sets store unique elements.  Like dictionaries, order is not preserved.  Sets are used for very fast membership testing.

Lists are an ordered collection of elements.  Lists are a little more flexible than arrays.  Lists can be heterogeneous and have no gaps.  


\begin{problem}
Write a stack.  Do not use Python lists in your implementation.
You will need to create your own node class and stack class.
\end{problem}

\begin{problem}
Write a graph data structure using adjacency lists.
\end{problem}