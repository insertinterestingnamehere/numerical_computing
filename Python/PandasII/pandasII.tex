\lab{Python}{Intro to pandas II}{Intro to pandas II}

In this lab, we explore in further detail two specific areas where pandas can be a very useful tool:
analyzing sequential data, and working with large datasets that can't be stored entirely in memory. 
\section*{Time Series Analysis}
A \emph{time series} is a particular type of data set that consists of a sequence of measurements or observations 
generated at successive points in time. Examples include the yearly average temperate of a city, or
the price of a given stock measured daily. 
\section*{Working With Large Datasets}
In the real world, a data scientist is often confronted with large datasets that can't be held in memory all at once.
There are various solutions to this problem; in this section, we will explore how pandas uses the HDF5 file format
to allow us to work with datasets on disk. 

HDF5, which stands for "Hierarchical Data Format", is a data storage system especially suited for large numerical datasets.
Rich and efficient software libraries have been developed over the years to enable fast read and write operations,
which make HDF5 a competitive option for working with large datasets in many applications. The Python library pytables
is one such library, and the HDF5 capabilities in pandas are built directly on top of pytables. 

We have two primary learning goals: how to get our data into the proper HDF5 format, and how to intelligently work with
the data once it's tidied up. Let's dive in.
\subsection*{Writing HDF5 Data}

\subsection*{Working With On-Disk Arrays}

