\lab{Applications}{Surfaces in 3D and Tesselation}{Tesselation}

\objective{Understand the basics of tesselation in 3D}

{\bf Outline:}
\begin{itemize}
\item Intro: In a graphics application, how do you calculate how surfaces look? There are a few options: parametrize the surface (which can be costly) or tesselate the surface (which may reduce quality).
\item Toy Problem: How do we best represent a sphere?
\item Mesh Refinement. Basically we start with an approximation and refine our approximation one step at a time. This allows a natural structure for the information about refinements (they can represent tesselation data in wavelets this way, Tony DeRose did this).
\item In the case of the sphere we do this using a dodecahedron, and then splitting each face into four faces.
\end{itemize}

\begin{problem}
Tesselate the sphere. Observe error associated with each face. In this case the errors will be uniform. Note that the error can be posed as an itegration problem (we can use different quadrature rules)
\end{problem}

\begin{problem}
Tesselate an ellipsoid. Generate an appropriate error estimator, and refine the tesselation adaptively in appropriate areas.
\end{problem}

\begin{problem}
Specular highlights: there is a bright spot corresponding to where the light reflects directly off of of a surface towards a viewer.The amount of specular light that reaches the viewer is based on different types of distributions (``phong'' is the name of the most common). For a tesselated surface this will be easy to calculate (since the surface normal is easy to determine, and there are a finite number of different surface normals), and for a fine tesselation on a surface this could look cool. Ideas: 1)have a moving light, and have the specular light calculated each time the light is moved 2) Give them a more complex tesselated surface, and have them compute the specular highlights. NOTE: this would take a fair bit of additional explanation, it would be cool, but it might be too much for here. We just have to decide a direction.
\end{problem}
