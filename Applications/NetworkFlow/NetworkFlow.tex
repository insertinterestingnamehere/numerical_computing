\lab{Applications}{Optimization}{Transportation}
\label{lab:Transportation}
\objective{.}


\section*{Transportation}

The following data is from the Seattle Transportation Traffic Management Division.

The graph represents a section of one way roads in Seattle, with the exception of Third street going both east and west.
 
We can model the flow of traffic using a system of equations. For example, if the rate of flow into an intersection is $x$ and $20$ and the flow out of the same intersection is $y$ and $45$, then the flow through the intersection can be represented by $x-y = 25$.

 
Create the system of equations that models these streets.
 
Now solve the system using linalg.lsts(). %Part 1a
\begin{lstlisting}[style=python]
\end{lstlisting}

Was this the kind of answer you expected? How can you have a negative number of cars? Since this is a closed system, the number of cars leaving is the same entering, we'll get infinitely many answers.
This explains the negative entries of x.
We can get further information by measuring the number of cars passing through each independent street in this system. Find the rank of the matrix to know how many streets we should be looking for and identify which streets we need to place counters on. 


 
The counters revealed the following data.

\begin{table}
 $x_4$ =& 2303\\
 $x_9$ =& 5302\\
 $x_{10}$ =& 13412\\
 $x_{15}$ =& 3052\\
 $x_{16}$ =& 4270\\
 $x_{17}$ =& 1243\\
\end{table}

Solve the system with these parameters.

So we've determined the amount of traffic on each road. Interpreting, we have that $9969$ cars travel east from Pike to Union along Fifth Street.

We can now identify the intersections with significant traffic in one direction and less traffic in the other.
 
Intersection A has lots of cars travelling east. $18144$ cars approach from the west and and $9972$ leave from the east. Comparatively few cars travel north, with $2303$ cars approaching the intersection from the south. This indicates intersection A would a good candidate for a light. Intersection E would also be a good choice. 

Which intersections would you recommend putting lights in first?
 
%Part 2
Now we want to find the fastest path through from A to L. This can be represented by a a Minimum Cost Linear Program for this network flow.
We'll define $c_i$ to be the cost of travelling along road $x_i$. The cost will represent the amount of traffic, the numbers you found in Part 1.

$A$, an $m\times n$ matrix, is called the arc-node incidence matrix where $m$ is the number of nodes and $n$ is the number of arcs. In our case, there are $12$ intersections and $17$ streets. $A$ is determined by the formula
\begin{displaymath}
   f(x) = \left\{
     \begin{array}{lr}
       1 & : arc j starts at node i\\
       -1 & : arc j ends at node i\\
       0 & : else
     \end{array}
   \right.
\end{displaymath}
 
Notice that each column must sum to $0$.
Arc 1, or $x_1$, starts at node, intersection, $A$ and ends at node $B$. So the first column of $A$ is $[1,-1,0,0,0,0,0,0,0,0,0,0,0,0,0,0,0,0]^T$.

$b$ is the external supply to the system. So $b$ is the same as in part 1, but be sure to remove the independent numbers we found in part 1. Unlike other optimization problems, $Ax=b$.  



look at network flow python packages?