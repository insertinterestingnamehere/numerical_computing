\lab{Algorithms}{Poisson's equation}{Poisson's equation}
\label{lab:finitedifference2}

 The basic idea behind finite difference methods is to replace a 
 differential operator, defined on some space of continuous functions, with a 
 difference operator defined on a finite vector space (i.e., a space of grid functions).
 
 To do this, we replace derivative terms in the differential equation with 
 appropriate difference expressions. 
 
 \begin{align*}
 	u_{xx}(t,x) &\approx \frac{(u(t,x+h)- 2u(t,x) + u(t,x-h))}{h^2},\\
 	u_x(t,x) & \approx \frac{u(t,x+h)-u(t,x-h)}{2h}.
 \end{align*}

Recall that the heat equation is given by $u_t = \nu u_{xx} + f(x,t,u)$. Suppose that we want to describe the flow of heat throughout a region $\Omega$. If the temperature on the boundary of $\Omega$ is fixed at $g$, and the $\Omega$ has an initial heat profile of $f$, then the flow of heat can be described by the boundary value problem 
\begin{align}
	\begin{split}
		& { } u_t = \nu \triangle u + f(x,t,u), \quad x \in \Omega, \quad t >0,\\
		& { }u(x,t) = h(x), \quad x \in \partial \Omega, \\
		& { }u(x,0) = g(x).
	\end{split}
\end{align}
When the source term $f$ does not depend on time, there is often a steady-state heat profile $u_{\infty}$ that is achieved as $t \to \infty$. This steady state $u_{\infty}$ is a solution of the boundary value problem 
\begin{align}
	\begin{split}
		& { }  \triangle u + f(x)/\nu = 0, \quad x \in \Omega,\\
		& { }u(x,t) = h(x), \quad x \in \partial \Omega.
	\end{split}
\end{align}
The partial differential equation $\triangle u = -f$ is commonly known as Poisson's equation. This equation is satisfied by the steady-state solutions of certain evolutionary processes. Poisson's equation is often used in electrostatics, image processing, surface reconstruction, computational fluid dynamics, and other areas. 

\section{Poisson's equation in one dimension}
Consider the boundary value problem defined by
\begin{align}
	\begin{split}
u'' &= f(x), \quad a \leq x \leq b,\\
	u(a) &= \alpha,\quad u(b) = \beta,
	\end{split}
\end{align}
where $f$ is continuous. 
The basic idea behind most numerical methods for differential equations is to 
approximate the exact solution $u(x)$ at some finite collection of points in the 
domain of the problem. Instead of analytically solving an infinite-dimensional
problem, we look for a simple finite collection of algebraic equations that approximate the original problem.

Note that this equation can easily be solved by integrating twice 
and using the boundary conditions to determine the constants of integration. 
Solving this boundary
value problem is equivalent to solving the equation $Du = f,$
where $D\frac{d^2}{dx^2}$ is a differential operator defined on the infinite-dimensional space 
of functions $u:[a,b] \to \mathbb{R}$ that are twice continuously differentiable and 
satisfy the boundary conditions $u(a) = \alpha$, $u(b) = \beta$. The finite difference method in general attempts to replace infinite-dimensional differential operators with finite difference operators. For example, recalling that $\frac{d^2}{dx^2}u(x) = \frac{1}{h^2}\left(u(x-h) -2u(x) + u(x+h)\right) + \mathcal{O}(h^2),$ we could  approximate $Du$ with the finite difference $\frac{1}{h^2}\left(u(x-h) -2u(x) + u(x+h)\right).$


We consider an approximate solution $\{u_i\}_{i=0}^N$ on an evenly spaced grid $a = x_0, x_1, \ldots, x_N = b$ with $h = x_{i+1}-x_i$ for each $i.$
We may discretize the boundary value problem as 
\begin{align*}
	\frac{1}{h^2} (u_{i+1}- 2u_i + u_{i-1})  &= f(x_i), \quad i = 1, \ldots, N-1,
\end{align*}
with the boundary conditions $u_0 = \alpha,$ $u_N = \beta.$ Note that this gives $N+1$ equations and $N+1$ unknowns. We can write this in matrix form as 
\[
\frac{1}{h^2} \begin{bmatrix}h^2 & 0 &0&\hdots &0 \\ 1 &-2 & 1 &\hdots &0\\ \vdots &  & \ddots & &\vdots \\
0 & \hdots & 1 & -2 & 1 \\ 0 & \hdots & & 0 & h^2
\end{bmatrix} \cdot \begin{bmatrix}y_0\\u_1\\ \vdots \\u_N\end{bmatrix} = \begin{bmatrix}f(x_0)\\f(x_1)\\ \vdots \\ f(x_N) \end{bmatrix}.
\]
We can further modify the system to obtain an $(N-1)\times (N-1)$ tridiagonal matrix on the left: 
\[
\frac{1}{h^2} \begin{bmatrix}-2 & 1 &0 & \hdots &0\\ 1 &-2 & 1 &\hdots &0\\ \vdots &  & \ddots & &\vdots \\ 0 & \hdots & 1 & -2 & 1 \\
0 & \hdots & 0 & 1 & -2 
\end{bmatrix} \cdot \begin{bmatrix}y_1\\y_2\\ \vdots \\y_{N-2}\\y_{N-1}\end{bmatrix} = \begin{bmatrix}f(x_1) -\alpha/h^2 \\f(x_2)\\ \vdots \\ f(x_{N-2})\\ f(x_{N-1})-\beta/h^2 \end{bmatrix}.
\]

The following code uses the finite difference method just described to solve the boundary value problem
\begin{align*}
u'' &= -3 \sin{x}, \quad 0 \leq x \leq 2,\\
	u(0) &= -2,\quad u(2) = 1.
\end{align*}
Note the use of matrix functions from \li{scipy.sparse}. %Several variations of this matrix will be built throughout this lab. 

\begin{lstlisting}
from __future__ import division
import numpy as np
from scipy.sparse import spdiags
from scipy.sparse.linalg import spsolve

def bvp(func,a=0.,b=2.,alpha=-1.,beta=3.,N = 5):
	h = (b-a)/N 				# The length of each subinterval
	
	# Initialize and define the vector F on the right
	F = np.empty(N-1.)			
	F[0] = func(a+1.*h)-alpha*h**(-2.)
	F[N-2] = func(a+(N-1)*h)-beta*h**(-2.)
	for j in xrange(1,N-2): 
		F[j] = func(a + (j+1)*h)
	
	# Here we define the arrays that will go on the diagonals of A
	D0, D1 = -2.*np.ones((1,N-1)), np.ones((1,N-1))  
	# Next we concatenate the arrays, and specify on which diagonals they will be placed
	diags = np.array([0,-1,1])
	data = np.concatenate((D0,D1,D1),axis=0) 
	A=h**(-2.)*spdiags(data,diags,N-1,N-1).asformat('csr')
	
	# We create and return the numerical approximation
	U = spsolve(A,F)
	U = np.concatenate( ( np.array([alpha]), U, np.array([beta]) ) )
	return np.linspace(a,b,N+1), U

x, y = bvp(lambda x:(-3.*np.sin(x)), a=0., b=2., alpha=-2., beta=1, N=30)
\end{lstlisting}

% How do we know if a numerical approximation is reasonable?  One way to determine this is to compute solutions for various step sizes $h$ and see if the solutions are converging to something.  To be more specific, suppose our finite difference method is $\mathcal{O}(h^p)$ accurate.  This means that the error $E(h) \approx Ch^p$ for some constant $C$ as $h \to 0$ (i.e., for $h>0$ small enough).
% 
% So compute the approximation $y_k$ for each stepsize $h_k$, $h_1 > h_2> \ldots>h_m.$  We will think of $y_m$ as the true solution.  Then the error of the approximation for 
% stepsize $h_k, k < m,$ is 
% \begin{align*}
% 	E(h_k) &= \max( \abs{ y_k - y_m}) \approx C h_k^p ,\\
% 	\log(E(h_k)) &= \log(C) + p \log(h_k).
% \end{align*}
% Thus on a log-log plot of $E(h)$ vs. $h,$ these values should be on a straight line with slope $p$ when $h$ is small enough to start getting convergence. 

% The following code generates the log-log plot in \ref{figure1}, and demonstrates second-order convergence for this finite difference approximation. 
% \begin{lstlisting}
% import matplotlib.pyplot as plt
% a, b = 0., 1.
% num_approx = 10 # Number of Approximations
% N = np.array([5*2**j for j in range(num_approx)])
% h, max_error = (b-a)/N[:-1], np.ones(num_approx-1)
% 
% mesh_best, num_sol_best = bvp(lambda x:-3.*np.sin(x), a, b, alpha=-2., beta=1, N=N[-1])
% for j in range(len(N)-1): 
%     mesh, num_sol = bvp(lambda x:-3.*np.sin(x), a, b, alpha=-2., beta=1, N=N[j])
%     max_error[j] = np.max(np.abs( num_sol- num_sol_best[::2**(num_approx-j-1)] ) )
% plt.loglog(h,max_error,'.-r',label="$E(h)$")
% plt.loglog(h,h**(2.),'-k',label="$h^{\, 2}$")
% plt.xlabel("$h$")
% plt.legend(loc='best')
% plt.show()
% print "The order of the finite difference approximation is about ", ( (np.log(max_error[0]) - 
%     np.log(max_error[-1]) )/( np.log(h[0]) - np.log(h[-1]) ) ), "."
% \end{lstlisting}

\begin{figure}[ht]
\centering
\includegraphics[width=12cm]{example_convergence.pdf}
\caption{TODO}
\label{figure1}
\end{figure}


\begin{problem}
Extend the given finite difference code to the case of a general second order linear boundary value problem with Dirichlet conditions:
\begin{align*}
	&{ } a_1(x)y'' +a_2(x)y'+ a_3(x) y = f(x), \quad x \in (0,1),\\
	&{ } y(0) = \alpha, \quad y(1) = \beta.
\end{align*}
Use your code to solve the singularly perturbed boundary value problem
\begin{align*}
	&{ } \epsilon y''(x)-y'= f(x), \quad x \in (0,1), \\
	&{ } y(0) = 1, \quad y(1) = 3,
\end{align*}
with $\epsilon = 1/10$. How many subintervals are needed to obtain 4 digits of accuracy? 

% If $\alpha = 1,$  $\beta = 3,$ and $f(x) = -1$, there is an exact solution: 
% \[y(x) = \alpha + x+ (\beta - \alpha -1)\frac{e^{x/\epsilon -1}}{e^{1/\epsilon -1}}
% .\]
\end{problem}

\begin{figure}[ht]
\centering
\includegraphics[width=12cm]{figure2.pdf}
\caption{TODO}
\label{figure2}
\end{figure}



\section{Poisson's equation in two dimensions}

 Consider Poisson's equation together with Dirichlet boundary conditions on a square domain:
 \begin{align*}
 	u_{xx} + u_{yy} &= f,\quad x \text{ in } [0,1]\times[0,1] \subset \mathbb{R}^2,\\
 	u &= g, \quad x \text{ on } \partial \left( [0,1]\times[0,1]\right).
 \end{align*}
 We will use the finite difference approximation
 \begin{align*}
 u_{xx}(x,y) + u_{yy}(x,y) &= \frac{u(x+h,y) - 2u(x,y)+ u(x-h,y)}{h^2} \\
 & \qquad{}+ 
 \frac{u(x,y+h) - 2u(x,y)+ u(x,y-h)}{h^2} + \mathcal{O}(h^2).
 \end{align*}
 Define the difference operator $\nabla^2_h$ by 
 \[
 \nabla^2_h U_{ij} = \frac{1}{h^2}(U_{i-1,\,j} + U_{i+1,\,j} + U_{i,\,j-1} + U_{i,\,j+1}-4U_{i,\,j}).
 \]
 Then the set of equations  $\nabla^2_h U_{ij} = f_{ij}$, $i,j = 1,\ldots,m$ can be written in matrix form as
 $$AU + q  = f$$
 Here $A$ is the block tridiagonal matrix 
 \[
 \frac{1}{h^2} \begin{bmatrix}T & I & &  &\\ I &T & I & &\\  &\ddots  & \ddots & \ddots & \\  &  & I & T & I \\
  &  &  & I & T
 \end{bmatrix}
 \]
 where $I$ is the $m\times m$ identity matrix, and $T$ is the tridiagonal matrix
 \[
  \begin{bmatrix}-4 & 1 & &  &\\ 1 &-4 & 1 & &\\  &\ddots  & \ddots & \ddots & \\  &  & 1 & -4 & 1 \\
  &  &  & 1 & -4
 \end{bmatrix}.
 \]

 The vector $U$ is given by 
 \[
 U = \begin{bmatrix} U^1 \\ U^2 \\ \\ U^m \end{bmatrix} \text{ where } U^j = 
 \begin{bmatrix} U_{1,\,j} \\ U_{2,\,j} \\ \\ U_{m,\,j} \end{bmatrix} \text{ for each } j, 1\leq j \leq m.
 \]
 So $U^j$ represents the $j$th row of interior points in our grid, where $y_j = jh.$
 
 
 The vector $q$ is given by $u = [q^1 \ldots q^m]^T$, where 
 \[
 q^j = \frac{1}{h^2}
 \begin{bmatrix} g_{0,\,j} \\ 0 \\ \vdots \\0\\ g_{m+1,\,j} \end{bmatrix} , \,\,\, 2 \leq j \leq m-1,
 \]
 and 
 \[
 q^1 = \frac{1}{h^2}\begin{bmatrix} g_{1,0} + g_{0,1} \\ g_{2,0} \\ \vdots \\ g_{m-1,0}\\ g_{m,0} + g_{m+1,1}\end{bmatrix}, \quad q^m = \frac{1}{h^2}\begin{bmatrix} g_{1,m+1} + g_{0,m}\\ g_{2,m+1} \\ \vdots \\ g_{m-1,m+1}\\ g_{m,m+1} + g_{m+1,m}\end{bmatrix}.
 \]

\begin{problem}
	Find the solution $u$ of the 2D Laplace equation $\Delta u = 0$ on the unit 
	square $[0,1]\times [0,1] \subset \mathbb{R}^2,$ subject to the (Dirichlet) condition that 
	$u(x,y) = x^3$ on the boundary. 
	 
	 Graph your solution, and demonstrate convergence of the numerical approximation by 
	 creating a log-log plotof the error $E(h).$
\end{problem}

\begin{problem}
	Find the solution $u$ of the 2D Poisson equation with the given Dirichlet boundary conditions:
	\begin{align*}
		\Delta u &= -\pi^2 \sin(\pi x)\sin(\pi y), \quad (x,y) \in [0,1]\times [0,1], \\
		u(x,0) &= 1-x, \\
		u(x,1) &= 1-2x, \\
		u(0,y) &= 1, \\
		u(1,y) &= -y. 
	\end{align*}
	
	Graph your solution, and demonstrate convergence of the numerical approximation by 
	creating a log-log plot of the error $E(h).$
\end{problem}

 The matrix $A$ is sparse, and so we can use several functions from the package \texttt{scipy.sparse.linalg}. In particular, we use the functions \texttt{spdiags} and \texttt{spsolve}.
 
 
 
 
 \begin{verbatim}
 D1,D2,D3 = -4*np.ones((1,m**2)), np.ones((1,m**2)), np.ones((1,m**2)) 
 Dm = np.ones((1,m**2))
 for j in range(0,D2.shape[1]):
     if (j%m)==m-1:
         D2[0,j]=0
         if (j%m)==0:
             D3[0,j]=a0
 diags = np.array([0,-1,1,-m,m])
 data = np.concatenate((D1,D2,D3,Dm,Dm),axis=0)
 
 A = 1./h**2.*spdiags(data, diags, m**2,m**2).asformat('csr') 
 \end{verbatim}
 
\begin{comment}
 \subsection{2D Heat Equation}
 Recall that the collection of finite difference equations
 \[
 \nabla^2_h U_{ij} = 0, \quad 1 \leq i,j\leq m,
 \]
 can be written in matrix form as
 $$AU + q  = 0$$
 
 
 The Crank-Nicolson method for the 2D heat equation is given by 
 $$U_{i,\,j}^{n+1}- U_{i,\,j}^{n} = \frac{\Delta t}{2}(\nabla_h^2 U_{i,\,j}^{n} + \nabla_h^2 U_{i,\,j}^{n+1}) \text{ for each } 1 \leq i,j \leq m, $$
 is a second order accurate in both space and time. Basically we're using a midpoint scheme in time, 
 and a trapezoidal scheme in space. The resulting method is implicit, and can be written in matrix form as 
 \begin{align*}
 	IU^{n+1} &= IU^n + \frac{\Delta t}{2}(AU^n + q + AU^{n+1} + q),\\
 	(I - \frac{\Delta t}{2}A)U^{n+1}&= (I + \frac{\Delta t}{2}A)U^n + \Delta t q.	
 \end{align*}
 
 
 
% TODO: What size must the time step be to ensure stability? 
 
 We will need to take many time steps, where many equations must be solved with the matrix $(I - \frac{\Delta t}{2}A)$. The function \texttt{factorized} from \texttt{scipy.sparse.linalg} computes the LU decomposition of the matrix. This decomposition reduces the time required for solving consecutive time steps. 
 \end{comment}
\section{Poisson's Equation and Conservative Forces}
In physics Poisson's equation is used to describe the scalar potential of a consersative force (we'll explain what each of these terms mean). A conservative force $\bold{F}(\bold{r})$ is a vector function that obeys three equivalent conditions
\begin{enumerate}
	\item The curl is identically zero
			\[ \nabla \times \bold{F}(\bold{r}) = 0\]
	\item It can be written as the negative of a gradient, called the scalar potential
			\[ \bold{F}(\bold{r}) = - \nabla U(\bold{r}) \]
	\item The line integral over a closed path is zero
			\[ \oint \bold{F}(\bold{r}) \cdot d\bold{r} = 0 \]

\end{enumerate}
\textit{Note: Letters in bold are vector quantities}

The scalar potential defined over the space is a measure of the potential energy a particle would have if placed at that point.
It could be electric potential energy, gravitational potential energy, or anything else governed by a conservative force.

Whenever we have a conservative force we can use a few identies from vector calculus to obtain Poisson's equation for it's scalar potential. 
Consider for example the electrostatic force, the force that two electric charges exert on each other. 
It can be summarized in two of Maxwell's four equations.
\[
\nabla \times \bold{E} = -\frac{\delta \bold{B} } {\delta t} \qquad
\nabla \cdot \bold{E} = \frac{\rho}{\epsilon_0}
\]
Where $\bold{E}$ is the electric field, $\bold{B}$ is the magnetic field, $\rho$ the charge density, and $\epsilon_0$ the permisivity of free space, a constant. $\nabla \times$ and $\nabla \cdot$ represent respectively the curl and the divergence operators. In the absence of a changing magnetic field the curl is zero and so we have a conservative force (although $\bold{E}$ is not actually a force but a force per charge we can still treat it as a conservative field, proportionaly to a related conservative force). Because the curl is zero we can immediately assume the other properties of a conservative force and write $\bold{E}$ as a gradient
\[
\bold{E}  = - \nabla V = 0
\]
$V$ is called the electrostatic scalar potential, more commonly known as the voltage. 
If we insert this into the equation for the divergence of $E$ we find after the application of the identity $\nabla \cdot (\nabla\psi) = \Delta \psi$
\[
\nabla \cdot ( - \nabla V) = - \Delta V = \frac {\rho}{\epsilon_0}
\]
With this we have Poisson's equation for electrostatics. 
If there are no charges ( or that the divergence of $E$ is zero), then this reduces to Laplace's equation. 
A similar analysis can be applied using Gauss's law of gravity (a reformulation of Newton's law of gravity in terms of a divergence) for a description of the gravitational scalar potential in terms of mass density.

Unfortunately, because in most physical systems charges are free to roam, this electric potential feeds back into the system and changes the charge distribution. 
This means that in reality $V$ is present on both sides of the equation.
In fact it is usually non-linear. 
However, in the following analysis, we'll assume that the charges are fixed in space, or "frozen in".
We'll also ignore all constants (specifically $\epsilon_0$) and units so that we are really dealing with the relative charge.

One simple application of the electric potential is to calculate basic properties of simple molecules, starting from a charge distibution and from that calculating the electric potential field.
We'll assume that the electrons produce a clouds of charge centered around the atoms, appoximated by a Boltzmann Distribution $\rho = q e^(-r)$, where $q$ is the relative charge of the atom and $r$ is the distance from the atom. 
To make this easier, we'll make a function (\li{rho1}) to calculate the $\rho$ value at a certain point in space based the position of an atom and it's relative charge. 
We'll make another function (\li{rhoSum}) to give the total charge density at a point based on a list of atoms.

\begin{lstlisting}
#definitions for atoms position and charges
#the angle the hydrogen atoms make
theta = 106.0/180.0*np.pi
#Length of the two branches
A = 1.0
# Hydrogen 1 (x0,y0,q)
# Hydrogen 2 
# Oxygen
water = ((-np.sin(theta/2)*A, 0					, 1),
         ( np.sin(theta/2)*A, 0				 	, 1),
         ( 0				, -np.cos(theta/2)*A,-2) )

def rho1(x,y,atom):
    return atom[2]*np.exp(-np.sqrt((x-atom[0])**2 + (y-atom[1])**2))

def rhoSum(x,y,atoms):
    return np.sum([rho1(x,y,atom) for atom in atoms],axis=0)

#Generate a color dictionary for use with LinearSegmentedColormap
#that places red and blue at the min and max values of data
#and white when data is zero
def genDict(data):
    zero = 1/(1 - np.max(data)/np.min(data))
    cdict = {'red':[(0.0,  1.0, 1.0),
                   	(zero,  1.0, 1.0),
                   	(1.0,  0.0, 0.0)],
         'green':  [(0.0,  0.0, 0.0),
                   	(zero,  1.0, 1.0),
                   	(1.0,  0.0, 0.0)],
         'blue':   [(0.0,  0.0, 0.0),
                   	(zero,  1.0, 1.0),
                   	(1.0,  1.0, 1.0)]}
    return cdict
X = np.linspace(-5,5,100)
X,Y = np.meshgrid(X,X)
#Generate the grid of rho values
Rho = rhoSum(X,Y,water)
plt.imshow(Rho,cmap =  mcolors.LinearSegmentedColormap('cmap', genDict(Rho)))
plt.colorbar()
plt.show()
\end{lstlisting}
The function \li{genDict} scales the color values to be white when the charge density is zero. This is mostly to help visualize where there are neutrally charged zones by forcing them to be white. You may find it useful to also apply it when you solve for the electric  potential.

\begin{figure}[ht]
\centering
\includegraphics[width=12cm]{waterRho.png}
\caption{Relative charge density of an H$_2$O molecule}
\label{figure3}
\end{figure}

With a function for $\rho$, we can solve Poisson's equation for the electric potential field.
\begin{problem}
Solve for the electric potential $V$
\[
		\Delta V = -\rho(x,y) 
\]
The electric potential is usually defined to go to zero at inifinity.
We can approximate this using $V=0$ at the boundry conditions on $[-5,5]\times [-5,5]$.
Solve and plot the electric potential of water, using the definition for the atoms above.

\end{problem}

\begin{problem}
Solve for the electric potential of a CO$_2$ molecule.

CO$_2$ can be modeled as two atoms of relative charge $-1$ placed at $x=-1$ and $x=1$ on the $x$ axis and a third atom with relative charge $2$ at the origin. Simply reuse your code from the last problem with new definitions for the atoms using the template for a water molecule.

If the molecules are moving slowly (ie. when the temperature is low) electrostatic forces will dominate molecular interactions. 
Molecules will want to align themselves in the lowest energy configuration, or the lowest electric potential. 
Positive potentials will overlap themselves with negative potentials. 
From the electric potential plots you obtain, how do you think the molecules will arrange themselves when the temperature is cold, when they form an ice?
\end{problem}

\begin{figure}[ht]
\centering
\includegraphics[width=12cm]{co2V.png}
\caption{Relative electric potential field of a CO$_2$ molecule}
\label{figure4}
\end{figure}
