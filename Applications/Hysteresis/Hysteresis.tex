\lab{Applications}{Hysteresis}{Bifurcations}
\label{lab:Bifurcations}

Recall that any ordinary differential equation can be written as a first order system of DEs, 
\begin{align}
\dot{x} = F(x), \quad \dot{x} := \frac{d}{dt}x(t).\label{fos}
\end{align}
Many interesting applications and physical phenemena can be modelled using ODEs. Given a mathematical model of the form \eqref{fos}, it is important to understand geometrically how its solutions behave. This information can then be conveyed in a phase portrait, a graph describing solutions of \eqref{fos} with differential initial conditions. The first step in constructing a phase portrait is to find the equilibrium solutions of the equation, i.e., the zeros of $F(x)$, and to determine their stability. 

It is often the case that the mathematical model we study depends on some parameter or set of parameters $\lambda$. Thus the ODE becomes 
\begin{align}
\dot{x} = F(x,\lambda).\label{fos2}
\end{align}
The parameter $\lambda$ can then be tuned to better fit the physical application. As $\lambda$ varies, the equilibrium solutions and other geometric features of \eqref{fos2} may suddenly change. A value of $\lambda$ where the phase portrait changes is called a \emph{bifurcation point}; the study of how these changes occur is called \emph{bifurcation theory}. The parameter values and corresponding equilibrium solutions are often graphed together in a bifurcation diagram. 

As an example, consider the scalar differential equation 
\begin{eqnarray}
\dot{x} &=& x^2 + \lambda. \label{snbifurcation}
\end{eqnarray}
For $\lambda > 0$ equation (\ref{snbifurcation}) has no equilibrium solutions. At $\lambda = 0$ the equilibrium point $x=0$ appears, and for $\lambda < 0$ it splits into two equilibrium points. For this system, a bifurcation occurs at $\lambda = 0$. This is an example of a saddle-node bifurcation. The bifurcation diagram is shown in Figure \ref{bifurcation:sn} 


\begin{figure}[ht]
\centering
\includegraphics[width=\textwidth]{SaddleNBifurcation.pdf}
\caption{Bifurcation diagram for the equation $\dot{x} = \lambda + x^2$.}
\label{bifurcation:sn}
\end{figure}

% The flow of a differential equation \[\dot{x} = f(x)\] is the family of all possible solutions $\phi(t,x_0)$, where $x_0$ represents the arbitrary initial value, $t \in \mathbb{R}.$ Here we will mainly consider scalar differential equations (so-called because $x$ is one dimensional). It is often necessary to study a family of differential equations. These will have the form 
% \[\dot{x}= F(x,c),\]
% where $c$ may be a single parameter or a vector of parameters. 
% 
% Bifurcation theory is the study of how the qualitative structure of the flow of a differential equation varies as parameters in the differential equation are varied. 
% A differential equation has a stable orbit structure if sufficiently small changes in the parameter value do not change the qualitative structure of the flow.
% A parameter value for which the flow does not have a stable orbit structure is called a bifurcation value.
% 
% Terminology
% bifurcation theory: 'the study of possible changes in the structure of orbits of a differential equation depending on variable parameters'
% 'the study of changes in the qualitative structure of the flow of a differential equation as parameters are varied'
% phase portrait: 
% bifurcation diagram: 
% number of orbits and their direction of flow of a differential equation = 'orbit structure of the differential equation' or 'the qualitiative structure of the flow'.
%  
% 
% Hyperbolic equilibrium points
% $F(x_0,c_0) = 0$ where $\frac{df}{dx}(x_0,c_0) \not = 0.$ In this case the stability of the equilibrium point $x_0$ for values of $c$ near $c_0$ is determined by the derivative $\frac{df}{dx}(x_0,c_0)$.
% Example: $\dot{x} = x-c.$ Plot $x$
% 
% Hyperbolic equilibrium - mainly to be used when discussing stability
% Saddle-Node Bifurcation - \[\dot{x} = c + x^2.\]
% Transcritical Bifurcation - \[\dot{x} = cx + x^2.\]
% Hysteresis Loop - \[\dot{x} = c + x-x^3.\]
% Pitchfork Bifurcation (Supercritical) - \[\dot{x} = cx-x^3.\]
% For an exercise, do a variation of \[\dot{x} = 1+cx-x^3.\]
% 


Suppose that $F(x_0,\lambda_0) = 0.$ We use a method called natural embedding to find zeros $(x,\lambda)$ of $F$ for nearby values of $\lambda$. Specifically, we step forward in $\lambda$ by letting $\lambda_1 = \lambda_0 + \triangle \lambda$, and use Newton's method to find the value $x_1$ that satisfies $F(x_1,\lambda_1) = 0.$ This method works well except when $\lambda$ is near a bifurcation point $\lambda^*$.

The following code implements the natural embedding algorithm, and then uses that algorithm to find the curves in the bifurcation diagram for (\ref{snbifurcation}). Notice that this algorithm needs a good initial guess for $x_0$ to get started. 

\begin{lstlisting}
import numpy as np
import matplotlib.pyplot as plt
from scipy.optimize import newton

def EmbeddingAlg(param_list,guess,F):
	X = []
	for param in param_list:
		try:
			# Fix the parameter value inside the function.
			g = lambda x, param=param: F(x,param)
			# Solve for x_value making F(x_value, param) = 0.
			x_value = newton(g, guess, fprime=None, 
								args=(), tol=1.0e-08, maxiter=50)
			
			# Record the solution and update guess for 
			# the next iteration
			X.append(x_value)
			guess = x_value 
		except:
			return param_list[:len(X)], X	
	return param_list[:len(X)], X   		
	# returns the (possibly truncated) list of parameter values 
	# and corresponding x values


def F(x,lmbda):
	return x**2. + lmbda

# Top curve shown in the bifurcation diagram
C1, X1 = EmbeddingAlg(np.linspace(-5,0,200),np.sqrt(5),F)
# The bottom curve
C2, X2 = EmbeddingAlg(np.linspace(-5,0,200),-np.sqrt(5),F)
\end{lstlisting}


\begin{problem}
Use the natural embedding algorithm to create a bifurcation diagram for the differential equation
\[\dot{x} = \lambda x-x^3.\]
This type of bifurcation is called a pitchfork bifurcation (you should see a pitchfork in your diagram).
\end{problem}

\begin{problem}
Create bifurcation diagrams for the differential equation
\[\dot{x} = \eta + \lambda x-x^3,\]
where $\eta = -1, -.2, .2$ and $1.$ Notice that when $\eta = 0$ you can see the pitchfork bifurcation of the previous problem.
\end{problem}

\section{Budworm Population Dynamics}
A mathematical model describing budworm population dynamics is given by 
\begin{align}
\dot{N} = RN\left(1 - \frac{N}{K}\right) - p(N). \label{budworm1}
\end{align}
This model was studied by TODO et al, and described well in Strogatz's text. Here $N(t)$ represents the budworm population at time $t$, $R$ is the growth rate of the budworm population and $K$ represents the carrying capacity of the environment. We could interpret $K$ to represent the amount of food available to the budworms. 
$p(N)$ represents the death rate of budworms due to predators (birds); we assume specifically that $p(N)$ has the form $P(N) = \frac{BN^2}{A^2 + N^2}$.

Before studying the equilibrium points of \eqref{budworm1} it is important to reduce the number of parameters in the system by nondimensionalizing. Thus, we make the coordinate change $x = N/A$, $\tau = Bt/A$, $r = RA/B$, and $k = K/A$, obtaining finally the system 
\begin{align}
	\frac{dx}{d \tau} &= rx(1-x/k) - \frac{x^2}{1+x^2}.
\end{align}




\begin{problem}[Hysteresis]
Reproduce the bifurcation diagram for the differential equation
	\[\dot{x} = \lambda + x - x^3.\]
\end{problem}






