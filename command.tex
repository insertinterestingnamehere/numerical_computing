%------------------------------------------------------------------------------%
% command.tex                                                                  %
% This file contains the various environments and other misc. commands         %
%------------------------------------------------------------------------------%

%counter for problems. reset each chapter
\newcounter{problemnum}[chapter]


\newcommand{\objective}[1]{\vspace{5mm}{\bf Lesson Objective: } \emph{#1} \vspace{5mm}}
\renewcommand{\chaptername}{Lab}

\newcommand{\lab}[3]{\chapter[#3]{#1: #2}}

\newcommand{\argmax}{\text{argmax}}
\newcommand{\indicator}[1]{\mathbbm{1}_{\left[ {#1} \right] }}
\newcommand\cyr{%
\renewcommand\rmdefault{wncyr}%
\renewcommand\sfdefault{wncyss}%
\renewcommand\encodingdefault{OT2}%
\normalfont
\selectfont}
\DeclareTextFontCommand{\textcyr}{\cyr}
\def\Eoborotnoye{\char3}
\def\eoborotnoye{\char11}
\def\cprime{\char126}
\def\cdprime{\char127}

% Various commands that make life easier
\providecommand{\abs}[1]{\left\lvert#1\right\rvert}
\providecommand{\norm}[1]{\left\lVert#1\right\rVert}
\providecommand{\set}[1]{\lbrace#1\rbrace}
\providecommand{\setconstruct}[2]{\lbrace#1:#2\rbrace}
\DeclareMathOperator{\res}{res}           % Residue
\DeclareMathOperator{\Res}{Res}           % Residue

\newenvironment{amatrix}[1]{%
\left(\begin{array}{@{}*{#1}{c}|c@{}}
}{%
\end{array}\right)
}

\newenvironment{dmatrix}[2]{%
\left(\begin{array}{@{}*{#1}{c}|*{#2}{c}@{}}
}{%
\end{array}\right)
}

\newenvironment{pseudo}[2]
    {\begin{pseudocode}[shadowbox]{#1}{#2}}
    {\end{pseudocode}}

\newenvironment{problem}{\begin{shaded}\begin{problemnum}}{\end{problemnum}\end{shaded}}

\newtheoremup{problemnum}{Problem}
\definecolor{shadecolor}{gray}{0.90}

\newcommand{\li}[1]{\lstinline[style=python]!#1!}

\newcommand{\ipt}[2]{\langle #1,#2 \rangle}
\newcommand{\ip}{\int_{-\infty}^{+\infty}}

\renewcommand{\ker}[1]{\mathcal{N}(#1)}
\newcommand{\ran}[1]{\mathcal{R}(#1)}
