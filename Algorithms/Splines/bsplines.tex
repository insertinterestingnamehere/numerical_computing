\lab{Algorithms}{B-Splines}{B-Splines}
Though B\'{e}zier curves are good for a variety of things, they have certain limitations.
\begin{itemize}
\item As the number of control points increases it becomes very expensive to compute the points on the curve.
\item Changes in the placement of a single control point affect the shape of the entire curve.
\item Since a change in any control point affects the entire curve, it is necessary to recompute the entire curve to account for any change in any control point.
\item As the number of points increases, individual points have progressively less affect on the portions of the curve that lie nearest to them.
\end{itemize}

B-Splines are an ideal way to answer these limitations.
A B-Spline is, roughly speaking, a piecewise B\'{e}zier curve.
In order to better explain what a Basis function is.

\section*{B-Spline Basis Functions}

