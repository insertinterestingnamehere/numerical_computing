\lab{Algorithms}{Bezier Curves and B-Splines}{Splines}

\objective{Understand the basics of Bezier Curves, B-Splines and deBoor's Algorithm}

There are many different ways of approximating functions. In computer graphics, spline curves are often used. 

One specific problem that any sort of approximation to a set of points must overcome is that it must be coordinate independent. This means that by rotating the coordinate axes, we do not actually change the curve. This is a problem with normal curve interpolation methods. The following is an example of two interpolating curves through a set of points in different coordinate systems.

\includegraphics[width=\textwidth]{bad_interpolation}

Another problem with traditional interpolation is that, though the curve may pass through each point, its behavior between the points we are interpolating may be unpredictable, as in the following example. This is called Runge's Phenomenon. When doing approximation of a function, this can be abated by using carefully chosen interpolation points; however, when we are trying to approximate a given set of points we will probably not be able to choose exactly what we are given. Carefully choosing interpolation points still leaves us with the rotation problem illustrated above.

\includegraphics[width=\textwidth]{bad_interpolation2}

There is clearly a need for a more stable curve that we can manipulate easily. This is why Bezier curves were invented. The following is a geometric way of tracing out a Bezier curve for an ordered set of $n$ points. These are called control points. We would like to define a smooth curve from the first point to the last point which can be manipulated according to the position of the other points. We will parameterize this curve between the first and last point with respect to $t$, letting $t$ go from $0$ to $1$. To evaluate the curve at a given $t=T_0$ do the following:
\begin{itemize}

\item

Parameterize the lines between each of the points letting $t$ go from $0$ to $1$. For the points $P_{n}$ and $P_{n+1}$ this parameterization is $P_{n} (1-t) + P_{n+1} t$.

\item

Evaluate each of these parameterizations at $T_0$. Note that this gives $n-1$ new points.

\item

Repeat this process on the new lists until only one point is left. This is the value of the Bezier curve for these control points at the parameter $T_0$.

\end{itemize}

This is what is known as De Casteljau's Algorithm. It can be illustrated as follows:

The following are illustrations of De Casteljau's Algorithm for $4$ points at $t= 0$, $.25$, $.5$, $.75$, and $1$ respectively.

\includegraphics[width=\textwidth]{decasteljau_1}

\includegraphics[width=\textwidth]{decasteljau_2}

\includegraphics[width=\textwidth]{decasteljau_3}

\includegraphics[width=\textwidth]{decasteljau_4}

\includegraphics[width=\textwidth]{decasteljau_5} 

This algorithm is also usable for large numbers of points, though evaluation time increases rapidly.

\includegraphics[width=\textwidth]{decasteljau_6}

\begin{problem}
Implement De Caseljau's algorithm.
\end{problem}

\begin{problem}
Use a slider bar to interactively show how the algorithm works. Use Matplotlib's \li{ginput} function to get the control points for the graph.
\end{problem}

It is important to note that these curves are only dependent upon the points themselves and not on the coordinate system with which we are viewing them. Since a bezier curve is defined only by its control points, it is much easier to rotate it without changing its shape.

\includegraphics[width=\textwidth]{bezier_rotation}

There is a way of generating these functions explicitly. It is by writing the parameters of $x$ and $y$ as sums of bernstein polynomials in $t$. The $k$'th bernstein polynomial of order $n$ is written as 

$$\theta_{i,n}=\left( \begin{smallmatrix} n\\ i \end{smallmatrix} \right) (1-t)^{n-i} t^i$$

where $\left( \begin{smallmatrix} n\\ i \end{smallmatrix} \right) = \frac{n!}{i!(n-i)!}$, i.e. the binomial coefficients.

The Bezier curve for a given set of control points (also called the control polygon) is given by the formula:

$$\gamma (t) = \sum_{i=0}^n P_i \theta_{i,n} (t)$$ 

where $P_i$ is the i'th point of the control polygon.

These functions are convenient for quick evaluation of low order bezier curves, but they are not stable for high numbers of points. This is because the coefficients $\left( \begin{smallmatrix} n\\ i \end{smallmatrix} \right)$ grow very rapidly for large $n$. For example, $\left( \begin{smallmatrix} 40\\ 20 \end{smallmatrix} \right)=137846528820$ The errors that can come from having these large coefficents used in each of these polynomials are not troublesome at for low degree polynomials, but they quickly become a concern for larger ones. 

\begin{problem}
Write a python function which, given the control polygon returns functions for the parameterizations of $x$ and $y$. You may want to use scipy polynomial objects to do this.
\end{problem}

\begin{problem}
Plot a bezier curve with 30 randomly generated control points using De Casteljau's Algorithm. Try doing it with Bernstein polynomials. What do you observe? How many control points can you use before the Bernstein polynomial approach breaks down?
\end{problem}

\begin{problem}
Compare computation time for computing different orders of Bezier curves using both methods. For differrent orders of Bezier curves, test running a single point through the algorithm and running a large number of points through the algorithm. What do you observe?
\end{problem}

\begin{problem}
Use De Casteljau's algorithm to evaluate a bezier curve of degree $10$ in 3-space using randomly generated control points.
\end{problem}

\begin{itemize}
\item We can build upon the foundation of bezier curves using a generalization known as b-splines
\item These are essentially piecewise functions, where we guarantee a certain degree of continuity.
\item The central difference is we define knot points, between which the function is a polynomial of specified degree. If we define all of our knots to be at the endpoints of our interval we actually get a bezier curve.
\item Like bezier curves, we have control points, which allow us to manipulate the curve in an intuitive way.
\item Talk about the definition, note, once again, the bernstein polynomials that would make direct evaluation difficult.
\item Talk about DeBoor's algorithm, which allows, again for easy, stable evaluation.
\end{itemize}

B-splines are a generalization of Bezier Curves. Two principal advantages of B-Splines are:

\begin{itemize}

\item Bezier curves increase in degree according to the number of points used. Actually evaluating the curve for large numbers of points may be very inefficent.

\item The effect of changing one control point on a Bezier Curve is global, i.e. that by changing any control point we change the entire curve. 

\end{itemize}

B-splines overcome these problems by piecing together bezier curves formed to approximate individual portions of the curve to form a new whole curve. The resulting piecewise curve allows for easy evaluation at any point along the curve and, since only some control points are used in evaluation at any given part of the curve, local changes in the control points only change the curve locally. Since the B-spline is a composite of Bezier curves, it maintains many of the useful properties of Bezier curves. 

\begin{problem}
Extend the problems from bezier curves to allow b spline input and output.
\end{problem}