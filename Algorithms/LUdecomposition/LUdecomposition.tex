\lab{Algorithms}{LU Decomposition}{LU Decomposition}
\label{lab:LUdecomp}
\objective{In this section we will find the REF and the LU decomposition for matrices.}

In linear algebra there are three elementary row operations on matrices: switching two rows, multiplying a row by a constant, and adding a multiple of one row to another row.
Each of these operations can, in theory, be done by left multiplication by a corresponding elementary matrix.
This approach is \emph{extremely} slow in practice.
It is much faster to perform these operations directly by modifying only the portions of an array that change as a result of the row operation.
The following code shows how these modifications can be made in-place to an array.
\lstinputlisting[style=fromfile]{row_opers.py}

\section*{Programming Row Reduction}
Solving a linear system can be done most efficiently by using elementary row operations to reduce a matrix to \emph{row echelon form} (REF), as opposed to \emph{reduced row echelon form} (RREF).
Consider the following matrix:
\[
\begin{pmatrix}
4&5&6&3 \\
2&4&6&4 \\
7&8&0&5
\end{pmatrix}
\]
Using elementary row operations, we can reduce $A$ to REF as follows:
\begin{lstlisting}
>>> import numpy as np
>>> A = np.array([[4., 5., 6., 3.],[2., 4., 6., 4.],[7., 8., 0., 5.]])
array([[ 4.,  5.,  6.,  3.],
       [ 2.,  4.,  6.,  4.],
       [ 7.,  8.,  0.,  5.]])
>>> A[1] -= (A[1,0]/A[0,0]) * A[0]
>>> A[2] -= (A[2,0]/A[0,0]) * A[0]
>>> A[2,1:] -= (A[2,1]/A[1,1]) * A[1,1:]
>>> A
array([[ 4. ,  5. ,  6. ,  3. ],
       [ 0. ,  1.5,  3. ,  2.5],
       [ 0. ,  0. , -9. ,  1. ]])
\end{lstlisting}
The additional requirement is often added that the first nonzero entry of each row be 1.
Do not worry about that requirement here.
Notice that in our third row operation we were able to operate on only a portion of the third row because we knew that the first value would still be 0.
In this case it made little difference, but it is good to watch for things like this because they can save a great deal of time when working with larger matrices. 

A brief discussion of some potential pitfalls is in order. 
Round-off error can lead to serious numerical issues. Consider the 
following variation on the above example. 
\begin{lstlisting}
>>> A = np.array([[4., 5., 6., 3.],[2., 2.5, 6., 4.],[7., 8., 0., 5.]])
array([[ 4.,  5.,  6.,  3.],
       [ 2.,  2.5,  6.,  4.],
       [ 7.,  8.,  0.,  5.]])
>>> A[1] -= (A[1,0]/A[0,0]) * A[0]
>>> A[2] -= (A[2,0]/A[0,0]) * A[0]
\end{lstlisting}
If we work this out by hand, we should currently have
\[
\begin{pmatrix}
4&5&6&3 \\
0&0&3&2.5 \\
0&-7.5&-10.5&-.25
\end{pmatrix}.
\]
All that is left is to swap the second and third rows, and the matrix is in 
row echelon form. However, suppose that due to numerical round-off error, 
the machine computes instead 
\[
\begin{pmatrix}
4&5&6&3 \\
0&10^{-15}&3&2.5 \\
0&-7.5&-10.5&-.25
\end{pmatrix}.
\]
The algorithm would then attempt to pivot on the \li{A[1,1]} entry:
\begin{lstlisting}
>>> A[2,1:] -= (A[2,1]/A[1,1]) * A[1,1:]
>>> A
array([[ 4. ,  5.0e+00 , 6.00e+00 ,  3.000e-00 ],
       [ 0. ,  1.0e-14 , 3.00e+00 ,  2.500e-00 ],
       [ 0. ,  0.0e+00 , 2.25e+14 ,  1.875e+14 ]])
\end{lstlisting}
The round-off error in the \li{A[1,1]} entry has affected the third
row, and the matrix is now much different than our first calculation. In 
larger matrices, such an error could potentially propagate through many 
steps in the calculation, resulting in garbage output.

This example also illustrates another issue to be aware of. Even if there 
were no round-off error in the \li{A[1,1]} entry, a naive implementation of 
the algorithm might still attempt to pivot on that entry, which would mean 
dividing by zero. As noted above, we would first need to swap rows 2 and 3 
in order to proceed. Dealing with leading zeros by permuting the rows is 
another consideration when designing a fool-proof REF solver. 

\begin{problem}
\label{prob:REF}
Write a Python function, which takes a matrix and reduces it to REF.
Assume that the matrix is invertible and ignore the possibility that a zero may appear on the main diagonal during row reduction. You are not 
responsible for dealing with potential round-off errors. 
\end{problem}

\section*{LU Decomposition}
The LU Decomposition refers to a process for factoring a square matrix into 
a product of a lower triangular matrix and an upper triangular matrix. Such 
a factorization exists only for certain classes of matrices. In the case 
of an invertible $n \times n$ matrix $A$, the LU decomposition exists if and only if all
of the leading principle minors are non-zero (that is, all of the
submatrices \li{A[:k,:k]} for $k = 1,\ldots,n-1$ are invertible). Even in 
this case, however, it may still be necessary to permute the rows of the 
matrix in order to prevent zeros from appearing on the main diagonal. Thus, 
row swap operations are in general necessary for the LU decomposition. 

Using row reduction we can reduce the matrix $A$ to upper triangular form.
Say this can be done in $k$ row operations.
Let $U$ be the upper triangular form of $A$, so we have:
Hence, we have
\[
U = E_k \dots E_2 E_1 A.
\]
Since the elementary matrices are invertible, we also have
\[
(E_k \dots E_2 E_1)^{-1} U =  A.
\]
Then we define $L$ to be
\[
L = (E_k \dots E_2 E_1)^{-1}
\]
which is the same as
\[
L = E_1^{-1} E_2^{-1} \dots E_k^{-1}
\]
In either case, we have $L U = A$.

The inverses of elementary matrices are also elementary matrices. $L$ can be computed by applying a series of simple operations to an identity matrix.
As it turns out, when we are only doing type 3 row operations, each of the operations represented by right multiplication by these inverse matrices results in a change in a single entry of $L$.

In practice, the LU decomposition of an array $A$ can be computed like this:
\begin{itemize}
\item Make a copy $U$ of $A$.
\item Make an identity matrix $L$ of the same shape as $A$.
\item Iterate through the entries below the diagonal of $U$.
For each entry below the main diagonal of $U$ do the following:
	\begin{itemize}
	\item Set the corresponding entry of $L$ to the quotient of the current entry of $U$ and the entry of the main diagonal of $U$ located above the current entry.
	\item Perform the type 3 row operation to set the current entry of $U$ to 0.
		Remember to avoid computation involving columns that have already been processed.
	\end{itemize}
\item Return $L$ and $U$
\end{itemize}
In this case, we have ignored the possibility that a 0 may appear along the main diagonal during computation.
A full implementation of the LU decomposition would have to account for this possibility as well by swapping rows appropriately.

\section*{Why This Matters}
The LU decomposition is more efficient for solving linear systems than traditional row reduction and also allows quick computation of inverses and determinants. 
For very large matrices, the LU decomposition can be performed without using any extra space as follows:
$L$ is be stored above the main diagonal of the array and $U$ is be stored below it.
There is no need to store the main diagonal of $L$ since all its entries are ones.

\begin{problem}
\label{prob:LU}
Write a Python function takes as input an $n\times n$ matrix, performs the LU decomposition and returns $L$ and $U$.
To verify that it works, multiply $L$ and $U$ together and compare to $A$.
Assume that the matrix is invertible and ignore the possibility that a zero may appear on the main diagonal during row reduction.

Write another version of the function that modifies its input in place, storing $L$ below the main diagonal and $U$ in the rest of the array.
\end{problem}

As noted earlier, in general one needs to consider row-swapping when 
calculating the LU decomposition. This means that it is not always possible 
to simply have 
\[
A = LU,
\] 
but rather 
\[
PA = LU,
\]
where $P$ is a permutation matrix representing the necessary row-swaps. Given such a decomposition, we can solve the linear system $Ax = b$ by 
first solving $Ly = Pb$ and then $Ux = y$. Since $L$ and $U$ are triangular, 
these systems can be solved easily with backward and forward substitution. 
We can use this technique to calculate $A^{-1}$ by solving the matrix 
equation $LUX = P$ column-by-column. Finally, we can calculate the 
determinant of $A$ via the formula
\[
\det(A) = (-1)^S\left(\displaystyle\prod_{i=1}^nu_{ii}\right),
\]
where $S$ is the number of row-swaps. 

SciPy includes a complete implementation of the LU decomposition in the \li{linalg} module. 

\begin{problem}
\label{prob:loopSolve}
Suppose there is a need to solve the system $Ax = b$ for fixed $A$ and many 
different values of $b$. It would be unwise to perform row-reduction for 
every single new value of $b$, and the LU decomposition helps us avoid this 
and save time. 

Write a function that takes in an invertible matrix $A$, the LU factorization
of $A$ (in the same form as the output of \li{scipy.linalg.lu_factor}),
a two-dimensional array $B$ whose rows represent the different values of $b$,
and a boolean argument indicating whether or not to use the LU factorization.
The function should then loop through the rows of the array of $b$ values
and solve the system $Ax = b$ at each iteration using either \li{scipy.linalg.lu_solve} or \li{scipy.linalg.solve}, depending on the value of
the boolean argument.
Set $A$ and $B$ to a random $500 \times 500$ arrays, and compute the LU
factorization of $A$. Time your function on these inputs for both values of
the boolean argument. Hopefully you see the utility of the LU decomposition
in this setting!
\end{problem}

%\begin{problem}
%\label{prob:lusolve}
%Write a function that takes the LU decomposition computed by the second function you made in Problem \ref{prob:LU} and another array representing the right hand side of a linear system and modifies the second array in place so that it represents the solution to the linear system.
%No changes to the array storing the LU decomposition are necessary.
%\end{problem}
%
%\begin{problem}
%\label{prob:det}
%Write a function which uses the solution to Problem \ref{prob:REF} to find the determinant of $A$.
%Notice that the solution to Problem \ref{prob:REF} computes $U$ when applied to a square matrix.
%\end{problem}

\section*{The Cholesky Decomposition (Optional)}

Under certain conditions, the Cholesky decomposition offers a more efficient alternative to the LU decomposition.
It requires half the number of calculations and half the memory that the standard LU decomposition needs.
Furthermore, it is a \emph{numerically stable} decomposition, which means that round-off and truncation errors are kept suitably under control, rather than growing and propagating throughout the computation.
Because of the efficiency and numerical stability, Cholesky decomposition is used in solving least squares, optimization, and state estimation problems.
The Cholesky decomposition, however, is only applicable to Hermitian (for real matrices, this means symmetric) positive definite matrices.
It can be thought of as the matrix equivalent to taking the square root of a positive real number.

The Cholesky decomposition of a $A$ is a lower-triangular matrix, $L$, such that
\begin{equation*}
 A = LL^*
\end{equation*}
Where $L^*$ is the conjugate transpose of $L$.
For real valued matrices, this is equivalent to $L^T$.

The entries of $L$ are calculated as follows.
\begin{align*}
&L_{i,j} = \frac{1}{L_{j,j}}\left(A_{i,j} -\sum_{k=1}^{j-1}{L_{i,k}L_{j,k}^*}\right) \mbox{ for $i>j$} \\ \\
&L_{i,i} = \sqrt{A_{i,i} - \sum_{k=1}^{i-1}{L_{i,k}L_{i,k}^*}}
\end{align*}
where $L^*$ denotes the conjugate transpose of $L$.

Notice that in this computation, current calculation will depend on previous calculations.
To calculate $L$ properly, you must start in the upper left corner and iterate down.

Note: when testing positive definite systems, an easy way to generate a random symmetric positive definite matrix is by generating a random array \li{A} and then computing \li{A.dot(A.T)}.

\begin{problem}
Write your own implementation of the Cholesky decomposition.
Test it using a random symmetric matrix (build a random square matrix $A$, then $A^TA$ will be positive definite).
Check the output of your function to ensure that it is functioning properly.
\end{problem}

%\begin{problem}
%Modify your previous answer so that it computes the Cholesky decomposition by modifying the array in place.
%Make sure you set the portion of the array above the main diagonal to 0.
%Then write a function that takes this reduced form of the array and uses it to solve a linear system by back substitution.
%This should be nearly the same as Problem \ref{prob:lusolve}.
%\end{problem}

The linalg module of SciPy also includes a Cholesky decomposition that should be much faster than the one you just implemented.
It works much like the LU decomposition, providing the methods
\li{cho_factor} and \li{cho_solve}. 

\begin{problem}
Repeat the steps in problem \ref{prob:loopSolve}, this time using the 
Cholesky decomposition. Make sure the input matrix to be factored is 
positive definite. 
\end{problem}