\lab{Algorithms}{The Shooting Method for Boundary Value Problems}{The Shooting Method for Boundary Value Problems}
\label{lab:Shooting}

Consider a boundary value problem of the form 
\begin{eqnarray*}
y'' &=& f(x,y,y'), \,\, a \leq x \leq b, \\
y(a) &=& \alpha, \\
y(b) &=& \beta. 
\end{eqnarray*}
One natural way to approach this problem is to study the initial value problem associated with this differential equation: 
\begin{eqnarray*}
y'' &=& f(x,y,y'), \,\, a \leq x \leq b, \\
y(a) &=& \alpha, \\
y'(a) &=& t. 
\end{eqnarray*}
The goal is to determine an  appropriate value $t$, so that the solution of the ivp is also a solution of the boundary value problem. 

Suppose $y(x,t)$ be the solution of the initial value problem given above. We wish to find the value of $t$ so that 
$y(b,t) - \beta = 0$. 
Applying Newton's method to the function $f(t) = y(b,t) - \beta$, we obtain the iterative method 
\[t_{n+1} = t_n - \frac{ y(b,t_n) - \beta}{\frac{d}{dt} \left.y(b,t)\right|_{t_n} }, \,\, n = 0,1,\hdots .\]

We recall that Newton's method requires a good initial guess $t_0$; a plausible initial guess would be the average rate of change of the solution across the entire interval, so that $t_0 =  (\beta - \alpha)/(b-a)$. If this initial guess is not sufficient, the initial guess may be refined by looking at the solution $y(x,t_0)$ of the initial value problem.

This method requires us to evaluate or approximate the function $\frac{d}{dt} \left.y(b,t)\right|_{t_n}$. This term may be approximated with a finite difference, giving us the iterative method
\[t_{n+1} = t_n - \frac{ (y(b,t_n) - \beta)(t_n - t_{n-1})}{y(b,t_n) - y(b,t_{n-1}) }, \,\, n = 1, 2,\hdots .\]
This variation of the shooting algorithm is called the secant method, and requires two initial values instead of one. Notice that finding $y(b,t_n)$ requires solving the initial value problem.

For example, consider the boundary value problem
\begin{equation}
\begin{split}
\label{bvp1}
y'' &= -4y -9\sin(x), \,\, x \in [0,3\pi/4],\\
y(0) &= 1, \\
y(3 \pi/4) &= -\frac{1+3\sqrt{2}}{2}.
\end{split}
\end{equation}


% The following code employs the secant method to find the initial slope $y'(0) = t$ of a solution of \ref{bvp1}. The numerical solution may then be computed and plotted using \li{dopri5} and \li{matplotlib}. % \ref{bvp1} has an exact solution $y(x) = \cos(2x) + (1/2)\sin(2x) -3\sin(x)$.

Recall that the necessary dependencies are \li{numpy} and the \li{ode} class from \newline \li{scipy.integrate}. The first step to finding the initial slope $y'(0) = t$ is to define 
the ode as a first order system of differential equations, and to define the necessary parameters:
\begin{lstlisting}
def ode_f(x,y): return numpy.array([y[1] , -4.*y[0] - 9.*numpy.sin(x)])

a, b = 0., 3*numpy.pi/4.
alpha, beta =  1., -(1.+3*numpy.sqrt(2))/2.

\end{lstlisting}
To calculate $y(b,t_0)$ we use the \li{ode} class: 
\begin{lstlisting}
reltol, abstol = 1e-9,1e-8
example = ode(ode_f).set_integrator('dopri5',atol=abstol,rtol=reltol) 
example.set_initial_value(np.array([alpha,t0]),a) 
y0 = example.integrate(b)[0]
\end{lstlisting}


\begin{figure}[ht]
\centering
\includegraphics[width=\textwidth]{Fig1.pdf}
\caption{Two solutions of $y'' = -4y -9\sin(x),$ both satisfying the boundary conditions $y(0) = y(\pi) = 1.$}
\label{shooting:prob2}
\end{figure}



\begin{problem} Use the secant method to solve the bvp
\begin{equation*}
\begin{split}
y'' &= -4y -9\sin(x), \,\, x \in [0,3\pi/4],\\
y(0) &= 1, \\
y(3\pi/4) &=-\frac{1+3\sqrt{2}}{2}.
\end{split}
\end{equation*}
When you code the secant method, be sure to include a maximum number of iterations, as well as an appropriate stopping criteria. For example, if $|y(b,t_n)-\beta|<10^{-8},$ you have likely found a solution. 
\end{problem}


\begin{problem} Appropriately defined initial value problems will usually have a unique solution. Boundary value problems are not so straightforward; they may have no solution or they may have several. You may have to determine which solution is physically interesting. The following bvp has at least two solutions. Using the secant method, find both numerical solutions and their initial slopes. (Their plots are given in Figure \eqref{shooting:prob2}.) What initial values $t_0, t_1$ did you use to find them?
\begin{equation*}
\begin{split}
y'' &= -4y -9\sin(x), \,\, x \in [0,\pi],\\
y(0) &= 1, \\
y(\pi) &=1.
\end{split}
\end{equation*}

\end{problem}


\begin{problem}[The Cannon Problem]
Consider the following scenario: suppose a projectile is fired from a cannon with velocity $35\, m/s^2.$ At what angle $\theta(0)$ should it be fired to land at a distance of $120\, m$? (These differential equations do not take into account air resistance.)

There should be two values $\theta(0)$ that produce a solution for this bvp. Use the secant method to numerically compute and then plot both trajectories.
\begin{eqnarray*}
\frac{dy}{dx} &=& \tan {\theta} ,\\
\frac{dv}{dx} &=& -\frac{g \tan{\theta}}{v } ,\\
\frac{d\theta}{dx} &=& -\frac{g}{v^2},\\
y(0)&=& y(120) = 0,\\
v(0) &=& 35 \text{ }m/s^2.\\
\end{eqnarray*}
($g = 9.8067$ $m/s^2.$) Hint: This is a system of three first order differential equations, and so our secant method requires a slight modification. Define an appropriate function $f(t),$ and then use Newton's method.
\end{problem}


Let us consider how to solve for $\frac{d}{dt} y(b,t)$. We will assume that the function $y(x,t)$ can be differentiated with respect to $x$ and $t$ in any order, and let  $z(x,t) = \frac{d}{dt} y(x,t).$ Using the chain rule, we obtain 
\begin{eqnarray*}
z'' = \frac{d}{dt} y''(x,t) &=& \frac{\partial f}{\partial y} (x,y(x,t),y'(x,t)) \cdot \frac{dy}{dt}(x,t) ,\\
&+& \frac{\partial f}{\partial y'} (x,y(x,t),y'(x,t)) \cdot \frac{dy'}{dt}(x,t),
\end{eqnarray*}
Using the initial conditions associated with $y(x,t)$, we obtain the following initial value problem for $z(x,t)$: 
\begin{eqnarray*}
z'' &=& \frac{\partial f}{\partial y} (x,y,y') z + \frac{\partial f}{\partial y'} (x,y,y') z'
,\,\,a \leq x \leq b, \\
 z(a) &=& 0, z'(a) = 1.
\end{eqnarray*}

To use Newton's method, the (coupled) IVPs for $y$ and $z$ must be solved simultaneously. The iterative method then becomes 
\[
t_{n+1} = t_n - \frac{ y(b,t_n) - \beta}{z(b,t_n)}, \,\, n = 0,1,\hdots
.\]
% The following code solves the BVP using Newton's method: 
% \begin{lstlisting}
% import numpy as np
% from scipy.integrate import ode
% 
% a, b = 0., 3*np.pi/4.
% alpha, beta =  1., -(1.+3*np.sqrt(2))/2.
% dim, iterate = 4,10
% reltol, abstol,TOL = 1e-9,1e-8,1e-9
% t = (beta-alpha)/(b-a)  # Initial guess for the slope y'(1)
% def ode_f(x,y): 
% 	return np.array([y[1], -4.*y[0] - 9.*np.sin(x), 
% 				 y[3],  -4.*y[2]                  ])
% 	
% 	
% for j in range(1,iterate):
% 	print '\nj = ', j,'\nt = ', t
% 		
% 	example = ode(ode_f).set_integrator('dopri5',atol=abstol,rtol=reltol,nsteps=5000) 
% 	example.set_initial_value(np.array([alpha,t,0.,1.]),a) 
% 	X = example.integrate(b)
% 	y0, z0 = X[0],X[2]
% 
% 	if abs(y0-beta)<TOL: 
% 		print '\n--Initial slope t computed successfully--',\
% 		'\n|y(b) - beta| = ',np.abs(beta-y0),'\n'
% 		break
% 	t = t - (y0 - beta)/z0   # Update guess y'(1) = t
% \end{lstlisting}


\begin{problem}
Use Newton's method to solve the bvp
\begin{equation*}
\begin{split}
y'' &= 6y^2-6x^4-10, \,\, x \in [1,2],\\
y(1) &= 2, \\
y(2) &= 4.25.
\end{split}
\end{equation*}
Plot your solution. What is an appropriate initial guess? 
\end{problem}

\begin{problem}
Use Newton's method to solve the bvp
\begin{equation*}
\begin{split}
y'' &= 3 + \frac{2y}{x^2}, \,\, x \in [1,e],\\
y(1) &= 6, \\
y(e) &= e^2 + 6/e.
\end{split}
\end{equation*}
Plot your solution. What is an appropriate initial guess? 
\end{problem}












% 
% \section*{The Cannon Problem}
% 
% Consider the problem of tracking a projectile launched at given angle and velocity. Let its coordinates be given by $\vec{r}(t) = \langle x(t), y(t) \rangle.$ If $\theta(t)$ represents the angle of the velocity vector from the positive $x$-axis and $v(t) = |\vec{v}(t) |$, then we have 
% \begin{eqnarray*}
% \dot{x} &=& v\cos{\theta},\\
% \dot{y} &=& v\sin{\theta}.\\
% \end{eqnarray*}
% The tangent vector to the path traced by the projectile is the unit vector in the direction of the projectile's velocity, so $\vec{T}(t) = \langle \cos{\theta}, \sin{\theta} \rangle.$ The unit normal vector $\vec{N} (t)$ is given by $\vec{N} (t)= \langle -\sin{\theta}, \cos{\theta} \rangle.$ Thus the relationship between basis vectors $\vec{i}, \vec{j}$, and $\vec{T(t)}, \vec{N}(t)$ is given by 
% 
% \[
% \left[\begin{array}{cc}\cos{\theta} & \sin{\theta} \\-\sin{\theta} & \cos{\theta}\end{array}\right] \left[\begin{array}{c}\vec{i} \\\vec{j}\end{array}\right] = \left[\begin{array}{c}\vec{T(t)} \\\vec{N(t)}\end{array}\right]
% .\]
% 
% 
% 
% 
% Let $F_g$ represent the force on the projectile due to gravity, and $F_d$ represent the force on the projectile due to air resistance. (We assume the air is still.) From Newton's law we have
% 
% \begin{eqnarray*}
% m \dot{\vec{v}} &=& F_g + F_d , \\
% &=& -mg \vec{j} - C_Dv^2 \vec{t},\\
% &=& -mg( \sin{\theta} \vec{T} + \cos{\theta} \vec{N} ) - C_D v^2 \vec{T},\\
% &=& (-mg \sin{\theta} - C_D v^2 ) \vec{T} - mg \cos{\theta} \vec{N}.
% \end{eqnarray*}
% 
% From the identity 
% $\vec{v} = \langle \dot{x}, \dot{y} \rangle = \langle v \cos{\theta}, v \sin{\theta} \rangle$ 
% we have 
% \begin{eqnarray*}
% m \dot{\vec{v}} &=& m\langle v' \cos{\theta} - v\sin{\theta} \cdot \theta' ,v'\sin{\theta} + v\cos{\theta} \cdot \theta' \rangle \\
% &=& m(v'\cos{\theta} - v\sin{\theta} \cdot \theta')(\cos{\theta} \vec{T} - \sin{\theta}\vec{N}) \\
% &+& m(v' \sin{\theta} + v\cos{\theta} \cdot \theta')( \sin{\theta} \vec{T} + \cos{\theta} \vec{N})  ,\\
% &=& m(\vec{T} \cdot v' + \vec{N} \cdot v \cdot \theta') .
% \end{eqnarray*}
% From these two equations we have 
% \begin{eqnarray*}
% mv' &=& -mg\sin{\theta} - C_D v^2,\\
% mv\theta' &=& -mg \cos{\theta}.
% \end{eqnarray*}
% 
% 
% Thus we have the coupled system of differential equations
% 
% \begin{eqnarray*}
% \dot{x} &=& v\cos{\theta},\\
% \dot{y} &=& v\sin{\theta},\\
% v' &=& -g\sin{\theta} - C_D v^2/m,\\
% \theta' &=& -g \cos{\theta}/v.
% \end{eqnarray*}
% 
% Suppose we wish to fire a projectile (which travels at a predetermined initial velocity) at a target. We need to find an appropriate angle at which the projectile should be fired. In this case the independent variable $t$ used above is unessential to our problem. If we assume that $t$ is an smooth invertible function of $x$ ($t = t(x)$), then we obtain 
% \begin{eqnarray*}
% \frac{dy}{dx} &=& \frac{dy}{dt}\frac{dt}{dx} ,\\
% &=& \frac{dy}{dt} \frac{1}{v\cos{\theta}}, \text{  etc.}
% \end{eqnarray*}
% Thus our system of differential equations becomes 
% 
% \begin{eqnarray*}
% \frac{dy}{dx} &=& \tan {\theta} ,\\
% \frac{dv}{dx} &=& -\frac{g \sin{\theta} + (C_D/m)v^2}{v \cos{\theta}} ,\\
% \frac{d\theta}{dx} &=& -\frac{g}{v^2}.
% \end{eqnarray*}
% 
% 
% 
% \begin{problem}
% Use the secant method to solve the boundary value problem 
% 
% \end{problem}
% 


%\li{ode} 
%
%\begin{lstlisting}

%\end{lstlisting}
%