\lab{Algorithms}{Jordan Forms and Schur Decomposition}{Jordan Forms and Schur Decomposition}
\objective{Learn about Jordan forms, what we would like to use them for, and why we use Schur decomposition instead.}
\label{lab:Jordan}



\textbf{Outline}
\begin{itemize}
	\item Recall that Jordan forms are (blah blah blah)
	\item Jordan forms are cool because it's good to know the eigenvalues and eigenvectors. 
	\item Unfortunately the Jordan form is really hard to compute. Abel's Impossibility Theorem suggests we might have to iterate. Most eigenvalue solvers these days are iterative algorithms that find the Schur decomposition. Like the QR algorithm, which you've already done a lab about.

\end{itemize}

\section*{Schur Decomposition}
Schur decomposition relies on the following theorem.


\begin{theorem}
{\bf Schur's Theorem:} For any complex $n \times n$ matrix $A$, there exists a unitary matrix $Q$ such that $A = Q^\ast U Q$, where $U$ is upper triangular.
\end{theorem}

%\footnotetext{The requirement that $A$ be complex doesn't mean we're in trouble if $A$ only has real entries. It just means we have to allow complex entries in the Schur decomposition for the theorem to hold in general. If $A$ is real and has all real eigenvalues, then the Schur decomposition of $A$ will also be real.}

The eigenvalues of $A$ are the diagonal entries of $U$. (Why?) But it follows from Abel's Impossibility Theorem (Theorem \ref{Theorem:Abel}) that no finite algorithm can compute the eigenvalues of $A$ exactly. (How does this follow?) Therefore, no finite algorithm can directly compute the Schur factorization of A.
