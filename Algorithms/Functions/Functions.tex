\lab{Algorithms}{Functions and Logic}{Functions and Logic}

\objective{This section will teach how to build functions. It will also introduce logical statements used in programming.}

Up to this point we have only run scripts. We now introduce a much more powerful and versatile method of problem solving: writing functions.

A function, in the programming sense, is a block of code that accepts input (arguments) and returns output.  When you need a function, you simply define one.  Python functions begin with the keyword \li{def}, followed by the function name.  Input arguments are listed inside parentheses separated by commas.  Finally the function definition ends with a colon.  In Python, every function returns a value. This value can be explicitely returned using a \li{return} statement.  If no \li{return} is defined, the function will always return \li{None}.  Below is a simple function that returns what it receives
\begin{lstlisting}[style=python]
: def returnInput(x):
....: return x
\end{lstlisting}

Note that all code inside the function is indented.  This is very important in Python.  Any code not indented under the function definition is not part of the function.  To execute a function, you call the function by name and pass input parameters.  Try calling the function and passing different arguments.
\begin{lstlisting}[style=python]
: returnInput("This is a string")
'This is a string'
: returnInput(37)
37
: returnInput(3**2)
9
\end{lstlisting}

This function is defined for as long as we have IPython running.  If we happen to close IPython, our function definition will be lost.  To avoid losing our function definitions, we can place them in a Python script file (a *.py file).

\begin{problem}
In a file, functions.py, define functions that will do the following:
\begin{itemize}
\item Accept anything, but always return the number 42
\item Accept two numbers and return their product
\item Accepts no arguments and prints ``You called!"
\end{itemize}
\end{problem}

Restart IPython.  Now, how do we use our functions defined in functions.py without having to redefine each function inside IPython.  Fortunately, Python allows us to import functions.  Each *.py is really a Python module which can be imported in the same fashion as the SciPy library.
\begin{lstlisting}[style=python]
: import functions
\end{lstlisting}

Now our function are accessible as \li{functions.<function_name>}.  Using the \li{as} keyword, we can assign an alias to our module.  Let us define a function in \li{functions.py} that will compute the roots of a quadratic equation.  This function will return a tuple containing multiple values.  Note that before using the \li{sqrt} function, we have to define it, which we do by importing \li{sqrt} from Python's \li{math} library.
\begin{lstlisting}[style=python]
from math import sqrt
def quadrForm(a,b,c):
    descr = sqrt(b**2-4*a*c)
    x1 = (-b+descr)/2.0*a
    x2 = (-b-descr)/2.0*a
    return (x1, x2)
\end{lstlisting}

This function takes as inputs the coefficients of a quadratic function and returns the output of the quadratic formula.
\begin{lstlisting}[style=python]
: functions.quadrForm(1,-2,1)
(1.0, 1.0)
\end{lstlisting}

We note that this function is susceptible to floating point error. For example, try to find the roots of the polynomial $x^2 - (10^7 + 10^-7)x + 1$ using our function:

\begin{lstlisting}[style=python]
>> [x1 x2] = quadrForm(1,-(1e7 + 1e-7),1)
(10000000.0, 9.96515154838562e-08)
\end{lstlisting}

Clearly the first root is correct, but the second root is clearly in error (admittedly the error is small, but in this case it is easily fixed). The second root shows the error because it is calculated by subtracting two numbers that are very close together.

We can solve this problem by using slightly different approach. We first calculate the root that is farther from zero using the formula.  Note that we will have to write our own \li{sign()} function in Python.  The \li{sign()} function should return $-1.0$ if the input is negative, $0.0$ if the input is zero, or $1.0$ if the input is positive.
\[
x_1 = \frac{-b - \text{sign}(b)*\text{descr}}{2a}
\]
We can then use a formula known as Viete's formula to calculate the other root (solving for x2).
\[
x_1 x_2 = c/a
\]

\begin{problem}
Write a function that implements this second approach to finding the roots. Call it \li{quadrForm2}.  Note the improved accuracy.
\end{problem}

\section*{Logic, Conditionals and Loops}
Three basic operators in logic are \li{and} (\&), \li{or} (\textbar), and \li{not} (!). We can use relational operators from the~\ref{tbl:relops} to build logical statements.

\begin{table}[h!]
\begin{center}
\begin{tabular}{|c|c|}
	\hline
	\li{==} & equal to\\
        \li{<} & less than\\
	\li{<=} & less than or equal to\\
	\li{\!=} & not equal to\\
	\li{>} & greater than\\
	\li{>=} & greater than or equal to\\
	\hline
\end{tabular}
\caption{Some relational operators}
\label{tbl:relops}
\end{center}
\end{table}

Using these building blocks we can build complicated statements. For example, consider a vector $x$ and the statement \li{(x>1) & (x<10)}.  The statement will evaulate \li{True} if $x_i$ is between one and ten, and \li{False} otherwise.  This technique is called \emph{masking} and is useful for operating only on parts of a vector or an array that meet certain conditions.  Applying the mask to our vector $x$ will yield a boolean vector (a vector with only \li{True} or \li{False} values).  We can assign an operation for when we encounter a \li{True} value.  For example, suppose we want to add $5.5$ to any number that is less than $0.6$ in our vector $x$.  We can do this simply with the following statement:
\begin{lstlisting}[style=python]
: a = sp.rand(10); a
array([ 0.90492936,  0.42251211,  0.80463372,  0.45087988,  0.94395439,
        0.11372628,  0.17177908,  0.42053736,  0.94759312,  0.59030686])
: b = (a<.6); b
array([False,  True, False,  True, False,  True,  True,  True, False,  True], dtype=bool)
: a[b]  #the values that correspond to the True values of our mask
array([ 0.42251211,  0.45087988,  0.11372628,  0.17177908,  0.42053736,
        0.59030686])
: a[b] += 5.5; a   # a += 1 is equivalent to a = a + 1. a[b] makes it so that the operation is only applied to values which correspond to a 'True' value in b.
array([ 0.90492936,  5.92251211,  0.80463372,  5.95087988,  0.94395439,
        5.61372628,  5.67177908,  5.92053736,  0.94759312,  6.09030686])
\end{lstlisting}

\begin{problem}
Create logical statements for the following:
\begin{itemize}
%\item True if x is prime and less than one-thousand, false otherwise (use the function \li{isprime}).
\item True if $x^2$ is greater than 10 or x is positive and smaller than 2.
\item True if the bessel function of the second kind evaluated at x, with nu = 1, has magnitude greater than 1 (you will need to import \li{special.yn()}).
\end{itemize}
\end{problem}

\begin{problem}
Create a function that takes an input vector x and shuffles it like a deck of cards. You may assume that x has even length. The key is to create a mask using some random vector and then assign specific cards to even or odd slots of an output vector using that mask. Then see how many shuffles it takes to make the output $y$ ``random." (you can do this by looking at the offdiagonal entry of corrcoef(x,y), which should be close to zero if the output is random).
\end{problem}

