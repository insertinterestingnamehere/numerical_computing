\lab{Algorithms}{IVP Solvers}{IVP Solvers}
\label{lab:IVP}

\objective{Implement several basic numerical solvers for initial value problems, and use them to study harmonic oscillators.}

\section{Solvers for initial value problems}

Consider the initial value problem 
\begin{eqnarray*}
y' &=& f(x,y),\,\, a \leq x \leq b, \\
y(a) &=& y_0
\end{eqnarray*}
where $f$ is a continuous function. A solution is a continuously differentiable function $y(x)$ that satisfies the equation $y' = f(x,y)$ on the interval $[a,b]$ and for which $y(a) = y_0$.  


There are many initial value problems (IVPs) where it is impossible to find a closed form (analytic) expression for the solution. 
For other IVPs there is a closed form expression for the solution, but it may be difficult to interpret. 
In either case, we can usually make good use of various methods of numerical approximation to study the solution. (It is still important to know mathematically that a solution exists, even when we cannot find an explicit solution!)

Most methods require us to approximate the solution on a set of grid points $a = x_0< x_1<\hdots< x_n = b$ in our interval.  For simplicity we will assume that each of the $n$ subintervals $[x_{i-1},x_i]$ has equal length $h = (b-a)/n$. $h$ is called the \textit{step size}. We then look for values $y_0,y_1, \hdots, y_n$ that approximate our solution ( so $y_i \approx y(x_i)$).  

Consider the following use of Taylor's theorem: For each $i, 1 \leq i \leq n$, we have
\[
y(x_{i+1}) = y(x_{i}) + h y'(x_i) + \frac{h^2}{2} y''(\xi_i)\text{ for some }\xi_i \in [x_i,x_{i+1}].
\]
For small values of $h$, the quantity $\frac{h^2}{2} y''(\xi_i)$ will be negligible ($h^2$ is much smaller than $h$), and so we will have
\begin{eqnarray*}
y(x_{i+1}) &\approx& y(x_{i}) + h y'(x_i)  ,\\
&\approx & y(x_{i}) + h f(x_i,y(x_i)).
\end{eqnarray*}
This approximation leads to a first order ($\mathcal{O}(h^1)$) method called Euler's method: Let $y_0 = y(a)$, and for $i = 0, 1, \hdots, n-1$, let $y_{i+1} = y_i +hf(x_i,y_i)$. 
% \begin{enumerate}
% \item Let $y_0 = y(a)$. 
% \item For $i = 0, 1, \hdots, n-1$, let $y_{i+1} = y_i +hf(x_i,y_i)$. 
% \end{enumerate}

We can also use Taylor's theorem as follows: 
\begin{eqnarray*}
y(x_{i}) &=& y(x_{i+1}) - h y'(x_{i+1}) + \frac{h^2}{2} y''(\xi_i) \text{ for some } \xi_i \in [x_i,x_{i+1}], \\
y(x_{i+1}) &\approx & y(x_{i}) + h f(x_{i+1},y(x_{i+1}))  \text{ for small } h .
\end{eqnarray*}
This approximation leads to the backwards Euler method, another first order method: Let $y_0 = y(a)$ and for $i = 0, 1, \hdots, n-1$, solve  $y_{i} = y_{i+1}-hf(x_{i+1},y_{i+1})$ for $y_{i+1}$.
% \begin{enumerate}
% \item Let $y_0 = y(a)$. 
% \item For $i = 0, 1, \hdots, n-1$, solve  $y_{i} = y_{i+1}-hf(x_{i+1},y_{i+1})$ for $y_{i+1}$. 
% \end{enumerate}

Note that for both the Euler and backwards Euler methods, only $y_i, f, $ and other points in the interval $[x_i, x_{i+1}]$ are needed to find $y_{i+1}$. Because of this these are called \textit{one-step methods}. 



%\clearpage

\begin{figure}[ht]
\centering
\includegraphics[width=\textwidth]{Fig1.pdf}
\caption{The solution of $y' -y= -2x+4, y(0) = 0$, is $y(x) = -2+2x + 2e^x.$ This is a plot of the solution, alongside approximations with Euler's method for several stepsizes.}
\label{ivp:euler}
\end{figure}



\begin{problem}
The solution of the IVP
\begin{eqnarray*}
y' + y &=& 2-2x,\,\, 0 \leq x \leq 2, \\
y(0) &=& 0,
\end{eqnarray*}
is given by $y(x) = 4-2x -4e^{-x}$. Use Euler's method to numerically approximate the solution
with step sizes $h = 0.4, 0.2$, and $0.1.$ Plot your results using \li{matplotlib}.
\end{problem}



So how do we come up with numerical methods with higher order accuracy? Using Taylor's theorem (as we did for Euler's method) to create higher-order one-step methods would lead to numerically approximating derivatives of $f(t,y)$ - not necessarily desirable. 

Let us look for a second order method of the form 
\begin{enumerate}
\item $y_0 = y(a)$,
\item $y_{i+1} = y_i + a f(x_i+b, y_i+c)$.
\end{enumerate}
By expanding $a f(x+b, y+c)$ with Taylor's theorem and matching constants in the equation
\begin{eqnarray*}
f(x,y) + \frac{h}{2}f'(x,y) &=& f(x,y) + \frac{h}{2}\frac{\partial f}{\partial x}(x,y) +  + \frac{h}{2}\frac{\partial f}{\partial y}(x,y) \cdot f(x,y),
\end{eqnarray*}
we find that $a = 1, b = h/2,$ and $c = h/2$. This method is called the Midpoint method. IVP solvers with this general form are called \textit{Runge-Kutta methods}. Another second order (Runge-Kutta) method is the modified Euler method: 
\begin{enumerate}
\item $y_0 = y(a)$,
\item $y_{i+1} = y_i + \frac{h}{2}[ f(x_i, y_i) + f(x_{i+1}, y_i+ hf(x_i, y_i))]$ for $i = 0,1,\hdots, n-1$. 
\end{enumerate}

There are many Runge-Kutta methods with varying orders of accuracy. In practice, the methods of order four, such as the following, are most commonly used: 
\begin{enumerate}
\item $y_0 = y(a)$, 
\item $K_1 = hf(x_i,y_i)$,
\item $K_2 = hf(x_i + \frac{h}{2}, y_i + \frac{1}{2} K_1),$
\item $K_3 = hf(x_i + \frac{h}{2} , y_i + \frac{1}{2} K_2),$
\item $K_4 = hf(x_{i+1} , y_i +  K_3),$
\item $y_{i+1} = y_i + \frac{1}{6}(K_1 + K_2 + K_3 + K_4)$ for $i = 0,1,\hdots,n-1$.
\end{enumerate}




\begin{problem}
Suppose a differential equation is given by
\[ y' = f(t).\]
Which quadrature formulas correspond to Euler's method, backward Euler's method, modified Euler's method, the Midpoint method, and the fourth order Runge-Kutta method (RK4)? 
\end{problem}


\begin{problem}
Consider the IVP given by 
\begin{eqnarray*}
y' + y &=& 2-2x,\,\, 0 \leq x \leq 2, \\
y(0) &=& 0.
\end{eqnarray*}
Use Euler's method, the Midpoint method, and RK4 to approximate the value of the solution at $x = 2$, with a stepsize of $h = 0.2, 0.1,$ and $0.05 $. Find the relative error of each approximation.
\end{problem}

For many differential equations $y' = f(x,y)$ there is no closed form expression. Even when there is a closed form expression, it may be difficult to interpret. Consider the initial value problem 
\begin{eqnarray*}
y'(x) &=& \sin y(x), \\
y(0) &=& y_0.
\end{eqnarray*}
This IVP does have solution, which is given implicitly by 
\[x = \ln \left|\frac{\cos y_0 + \cot y_0}{\csc y + \cot y} \right|.\]
In this case, and many others, to understand the general solution of a ordinary differential equation it is often easier to use an IVP solver to plot several integral curves for various initial values. For example, it is easy to show that this differential equation has constant solutions $y_n(x) = n \pi, n \in \mathbb{N}$. Knowing this, and after plotting several integral curves as in Figure \ref{ivp:int_curves}, it is easy to see how solutions of this IVP will behave.


\begin{figure}
\centering
\includegraphics[width=\textwidth]{Fig4.pdf}
\caption{Several integral curves for the differential equation $y' =\sin y$, using the python solver \li{dopri5}. }
\label{ivp:int_curves}
\end{figure}

\begin{problem}
Plot the solutions of 
\[ y' + y = 2-2x\,\, 0 \leq x \leq 2, \] 
with initial conditions $y(0) = 0, 2, 4, 6, $ and $8$. Use RK4 to compute the solutions. 
\end{problem}

\pagebreak
\section{Harmonic Oscillators and Resonance}

Harmonic oscillators show up often in classical mechanics. A few examples include the pendulum (with small displacement), spring-mass systems, and the flow of electric current through various types of circuits. We will begin studying the mathematical model in the context of a spring-mass system.
%
%Mathematically,the motion of a  harmonic oscillator is described by an ordinary differential equation of the form \[ay'' + by' + cy = f(t) .\] Usually this equation is accompanied by two initial conditions, $y(0) = y_0$ and $y'(0) = y'_0.$ We will describe the construction of this differential equation in the context of a spring-mass sytem; we will then analyze its mathematical properties using the RK4 algorithm constructed in the last lab. 

Suppose an object with mass $m$ is hung at the end of a vertical spring. The position at which the object hangs is called the \textit{equilibrium position} for the system. Mathematically, a spring-mass system describes the motion of the object for time $t > 0$  if it is displaced from its equilibrium position and given an initial velocity. 

The principal property of a harmonic oscillator is that once the position $y(t)$ leaves its equilibrium position $y = 0$, it experiences a restoring force $F_r = -ky.$ This force pushes $y$ back towards its equilibrium position. For a spring-mass system this is called Hooke's law. Hooke's law describes the system well only if the displacement $y$ is small. The resulting differential equation is given by Newton's law, $F = ma$, giving us 
\[my'' = -ky.\]

Often there is an additional damping force $F_d$, often due to some type of friction. This force is usually proportional to the velocity of the mass, is always in the opposite direction of velocity, and represents energy leaving the spring-mass system. Thus we have $F_d = -\gamma y', $ where $ \gamma > 0$ is constant. 

We may also need to consider an additional, external force $f(t)$ that is interacting with our spring-mass system.

At this stage, we have a fairly general mathematical model: 
\begin{eqnarray*}
F &=& F_r + F_d + f(t),\\
my'' &=& -ky -\gamma y' + f(t),
\end{eqnarray*}
or 
\[my''+\gamma y' + ky = f(t). \]

We say that the harmonic oscillator is damped if $\gamma > 0$, and undamped if $\gamma =0$. Likewise the harmonic oscillator is forced if $f(t) \not = 0$, and free if $f(t) \equiv 0$. A simple harmonic oscillator is one that is undamped and free. 

\section*{Simple harmonic oscillators}
An undamped free harmonic oscillator is described by an IVP of the form
\begin{eqnarray*}
my'' + ky &=& 0,\\
y(0) &=& y_0,\\
y'(0) &=& y_0'.
\end{eqnarray*} 
The solution of this IVP is $y = c_1\cos (\omega_0 t) + c_2 \sin (\omega_0 t)$ where $\omega_0 = \sqrt{k/m}$ is the natural frequency of the oscillator and $c_1$ and $c_2$ are determined by applying the initial conditions. This in turn can be written in the form 
\[y = A\sin (\omega_0 t + \delta) .\]

To solve this IVP using the fourth order Runge Kutta method (RK4), we need to write this system in the form 
\[z'(t) = f(t,z(t)) \]
We can do this by letting $z_1 = y, z_2 = y'$. Then we have 
\[     z'=  \left[\begin{array}{c}z_1 \\z_2\end{array}\right]'  =  \left[\begin{array}{c}z_2 \\\frac{k}{m}z_1\end{array}\right]= f(z).\]

Consider the IVP 
\begin{eqnarray*}
y'' + .1y ' +y &=& 0,\,\, 0 \leq x  \leq 20, \\
y(0) &=&2,\\
y'(0) &=& 0.
\end{eqnarray*}
This can be written as the first order system
\[ \left[\begin{array}{c}y_1 \\y_2\end{array}\right]' = \left[\begin{array}{c}y_2 \\.1y_2-y_1\end{array}\right],
\]
and solved with the following code: 
\begin{lstlisting}
import numpy as np
import matplotlib.pyplot as plt

# Implementation of the Runge Kutta fourth order method
def Runge_Kutta_4(func,a,b,n,y0,dim):
	x = np.linspace(a,b,n+1); 
	Y = np.zeros((len(x),dim)); 
	Y[0,:] = y0
	
	h = 1.*(b-a)/n
	for j in range(0,len(x)-1): 
		k1 = h*func(x[j],Y[j,:])
		k2 = h*func(x[j]+h/2.,Y[j,:]+(1/2.)*k1)
		k3 = h*func(x[j]+h/2.,Y[j,:]+(1/2.)*k2)
		k4 = h*func(x[j+1],Y[j,:]+k3)
		Y[j+1,:] = Y[j,:] + (1/6.)*(k1 + 2*k2 + 2*k3 + k4)
	return Y

# Definition of f(y) in the 2 dimensional ode y' = f(y) for the equation y'' + .1y' + y = 0.
def ode_func(x,y): 
	out = np.array([ y[1] , .1*y[1]-y[0] ])
	return out

a, b, n, ya = 0.0, 20.0, 100, np.array([2., 0.])
Y = Runge_Kutta_4(ode_func,a,b,n,ya,2)
plt.plot(np.linspace(a,b,n+1), Y[:,0], 'k-')

plt.xlabel('x')
plt.ylabel('y')
plt.show()
\end{lstlisting}


\begin{problem}
Use RK4 to solve the IVP
\begin{eqnarray*}
my'' + ky &=& 0,\,\, 0 \leq x \leq 20, \\
y(0) &=& 2, \\
y'(0) &=& -1,
\end{eqnarray*} 
for $m = 1$ and $k =1$.  Plot your solution $y(t)$.  Compare this with the solution of the IVP if  $m = 3$ and $k =1$. Consider: Why does the difference in solutions make sense physically?
\end{problem}

\section*{Damped free harmonic oscillators}
We now consider damped free harmonic oscillators. These systems are described by the differential equation
\[my''(t) +\gamma y'(t) + ky(t) = 0.\]
For fixed values of $m$ and $k$, it is interesting to study the effect of the damping coefficient $\gamma$. 

The roots of the characteristic equation are \[r_1,r_2 = \frac{-\gamma \pm \sqrt{\gamma^2 -4km}}{2m} .\]
Note that the real parts of $r_1$ and $r_2$ are always negative, and so any solution $y(t)$ will decay over time due to a dissipation of the system energy. There are several cases to consider for the general solution of this equation: 
\begin{enumerate}
\item If $\gamma^2 > 4km$, then the general solution is $y(t) = c_1 e^{r_1t} + c_2e^{r_2t}$. Here the system is said to be $\textit{overdamped}$. Notice from the general solution that there is no oscillation in this case.
\item If $\gamma^2 = 4km$, then the general solution is $y(t) = c_1 e^{\gamma t/2m} + c_2 te^{\gamma t/2m}$. Here the system is said to be $\textit{critically damped}$. 
\item If $\gamma^2 < 4km$, then the general solution is 
\begin{eqnarray*}
y(t) &=& e^{-\gamma t/2m} \left[c_1\cos(\mu t) + c_2 \sin (\mu t)\right],\\
&=& R e^{-\gamma t/2m}  \sin (\mu t + \delta),
\end{eqnarray*}
where $R$ and $\delta$ are fixed, and $\mu = \sqrt{4km-\gamma^2}/2m.$ This system does oscillate.
\end{enumerate}



\begin{problem}
Use RK4 to solve the IVP
\begin{eqnarray*}
y'' +\frac{1}{2} y'+ y &=& 0, \,\, 0 \leq x \leq 20,\\
y(0) &=& 1, \\
y'(0) &=& -1,
\end{eqnarray*} 
on the interval $[0,20]$. Plot your solution $y(t)$, and find $y(20)$ correct to four decimal places. (Check that the relative error is less than $5e^{-5}$.)  How many subintervals do you need?
\end{problem}



\begin{problem}
Use RK4 to solve the IVP
\begin{eqnarray*}
y'' +2.1 y'+ y &=& 0,\,\, 0 \leq x \leq 20,\\
y(0) &=& 2, \\
y'(0) &=& -4.
\end{eqnarray*} 
Plot your solution $y(t)$, and find $y(20)$ correct to four decimal places. 
\end{problem}


\section*{Forced harmonic oscillators without damping}
Let's look at the systems described by the differential equation
\begin{eqnarray}
my''(t)  + ky(t) &=& F(t). \label{Forced_harm_osc}
\end{eqnarray}
In many instances the external force $F(t)$ is periodic, so let us assume that $F(t) = F_0 \cos(\omega t)$. If $\omega_0 = \sqrt{k/m} \not = \omega,$ then the  general solution of \ref{Forced_harm_osc} is given by 
\[y(t) = c_1 \cos (\omega_0 t) + c_2\sin (\omega_0 t) + \frac{F_0}{m(\omega_0^2 - \omega^2)} \cos (\omega t).\]
If $\omega_0 = \omega$, then the general solution is 
\[y(t) = c_1 \cos (\omega_0 t) + c_2\sin (\omega_0 t) + \frac{F_0}{2m\omega_0} t \sin (\omega_0 t).\]
This last solution contains a term that grows arbitrarily large as $t \to \infty$. If we included damping then the solution would be bounded, but will still be large for small $\gamma$ and $\omega$ close to $\omega_0$. Let us consider physical spring-mass system. \ref{Forced_harm_osc} holds only for small oscillations (this is where Hooke's law is applicable). For larger oscillations, this equation will not hold. However, the fact that the equation predicts large oscillations suggests the spring-mass system could fall apart as a result of the external force.

\begin{problem}
Use RK4 to solve the IVP
\begin{eqnarray*}
2y'' + 2y &=& 2 \cos (x), \,\, 0 \leq x \leq 200,\\
y(0) &=& 2, \\
y'(0) &=& -1.
\end{eqnarray*} 
Plot your solution $y(t)$, and find $y(200)$ correct to four decimal places. 
\end{problem}

\begin{problem}
Use RK4 to solve the IVP
\begin{eqnarray*}
2y'' + \gamma y' + 2y &=& 2 \cos (\omega x), \,\, 0 \leq x \leq 200,\\
y(0) &=& 2, \\
y'(0) &=& -1.
\end{eqnarray*} 
Plot your solution $y(t)$ for $\gamma = .1$ and $\omega =1.1$, and find $y(200)$ correct to four decimal places. 
Do the same for  $\gamma = .01$ and $\omega =1.01$.
\end{problem}

%References:
%
%Perona, Pietro, and Jitendra Malik. "Scale-Space and Edge Detection Using Anisotropic Diffusion." IEE Transactions on Pattern Analysis and Machine Intelligence 12 (1990): 629-39. Web.
%
%Kim, Seongjai. "Edge-Preserving Noise Removal, Part I: Second Order Anisotropic Diffusion." University of Kentucky Department of Mathematics Technical Report (2009): n. pag. Web.
%
