\lab{Algorithms}{Modified Gram-Schmidt (QR)}{QR decomposition}
\label{lab:QRdecomp}
\objective{Use Gram-Schmidt algorithm and orthonormal transformations to perform QR decomposition.}

The QR decomposition is used to represent any invertible square matrix as the matrix product of an orthogonal ($A^T A = I$) matrix, $Q$, and an upper triangular matrix, $R$.
This decomposition is useful in computing least squares and is part of a common method for finding eigenvalues (the QR algorithm).
We can succinctly state the QR decomposition in the following theorem.
\begin{theorem}
Let $A$ be an $m\times n$ matrix of rank $n$.  Then $A$ can be
factored into a product $Q R$, where $Q$ is an $m\times n$ matrix
with orthonormal columns and $R$ is a nonsingular $n \times n$ upper
triangular matrix.
\end{theorem}

\section*{Computing the QR Decomposition}
There are many methods for computing the QR factorization.
This lab will focus on the method that uses the Gram-Schmidt algorithm for orthogonalizing 
a full-rank matrix.
We use this algorithm to create $Q$.
Let $\{\x_i\}_{i=1}^n$ be a basis for the inner product space $V$.
Let \[ \q_1 = \frac{\x_1}{\norm{\x_1}}\] and define $\q_2,\q_3,\ldots,\q_n$ recursively by
$$ 
\q_{k} = \frac{\x_k - \p_{k-1}}{\|\x_k - \p_{k-1}\|}, \,\,\,\, k=2,\ldots,n,
$$
where
$$
\p_{k-1} = \sum_{i=1}^{k-1} \langle \q_i, \x_k\rangle \q_i,
$$
and $\p_0 = 0$. 
Then the set $\{\q_i\}_{i=1}^n$ is an orthonormal basis for $V$.

For the above algorithm, let $r_{j k} = \langle \q_j, \x_k\rangle$ when $j < k$ and
$r_{kk} = \|\x_k-\p_{k-1}\|$.
This can be written as
\begin{align*}
\x_1 &= r_{1 1} \q_1\\
\x_2 &= r_{1 2} \q_1 + r_{2 2} \q_2\\
\vdots \:\: &= \quad \vdots\\
\x_n &= r_{1 n} \q_1 + r_{2 n} \q_2 + \ldots + r_{n n} \q_{n},
\end{align*}
or in matrix form as
\[
[\x_1 \hspace{5mm} \x_2 \hspace{5mm} \cdots \hspace{5mm} \x_n]
=
[\q_1 \hspace{5mm} \q_2 \hspace{5mm} \cdots \hspace{5mm} \q_n]
\begin{bmatrix}
r_{1 1} & r_{1 2} & \cdots & r_{1 n}\\
0 & r_{2 2} & \cdots & r_{2 n}\\
\vdots & \vdots & \ddots & \vdots\\
0 & 0 & \cdots & r_{n n}
\end{bmatrix}.
\]
Hence if our original basis vectors $\{\x_i\}_{i=1}^n$ correspond to column
vectors of a matrix $A$, we can likewise write the resulting
orthonormal basis $\{\q_i\}_{i=1}^n$ as a matrix $Q$ of column
vectors.  Then we have that $A = Q R$, where $R$ is the above
nonsingular upper-triangular $n\times n$ matrix.
Numerically, the Gram Schmidt process can have problems due to
finite precision arithmetic. 

In some cases, rounding errors may cause the resulting basis to fail to be orthonormal. To combat this, we consider the Modified Gram-Schmidt which can be used to carry out a slightly revised algorithm.  To do this, we first compute $\q_1$ as before.  We then project it out of each of the remaining original vectors
$\x_2,\x_3,\ldots,\x_n$ via
\[
\x_k := \x_k - \langle \q_1,\x_{k} \rangle \q_1,\quad k=2,\ldots,n.
\]
Then we compute $\q_2$ to be the unit vector of $\x_2$, that is,
\[
\q_2 = \frac{\x_2}{\|\x_2\|}.
\]
We repeat by projecting out $\q_2$ from the remaining vectors
$\x_3,\x_4,\ldots,\x_n$, and then continuing this process until
all $\q_i$ are obtained.

\begin{algorithm}
\caption{Modified Gram-Schmidt}
\label{Alg:MGS}
\begin{algorithmic}[1]
\Procedure{Modified Gram-Schmidt}{$A$}
\State $m, n \gets \text{shape} \left( A \right)$
\State $Q \gets \text{copy} \left( A \right)$
\State $Q \gets \text{zeros}((n,n))$
\For{$0 \leq i < n$}
    \State $R_{i,i} \gets \norm{Q_{:,i}}$
    \State $Q_{:,i} \gets Q_{:,i}/R_{i,i}$
    \For{$i+1 \leq j < n$}
        \State $R_{i,j} \gets Q_{:,j}Q_{:,i}$
        \State $Q_{:,j} \gets Q_{:,j}-R_{i,j}Q_{:,i}$
	\EndFor
\EndFor
\State \pseudoli{return} $Q, R$
\EndProcedure
\end{algorithmic}
\end{algorithm}

\begin{problem}
\label{prob:QR}
Write your own implementation of the QR decomposition. Write a function \li{QR} that accepts as input a square matrix $A$ of full rank, and computes the QR decomposition, returning the matrices $Q$ and $R$ (which should be the same shape as $A$). Be sure to use the numerically stable Modified Gram-Schmidt algorithm. Also assume there are no zeros on the main diagonal.

You can test that you have the right decomposition by verifying that $QR=A$ and $Q^T Q = I$. While this is true most of the time, it may not always be true due to roundoff.
\end{problem}

\begin{problem}
The QR decomposition gives a really nice way to calculate the magnitude of the determinant of a square matrix of full rank. Write a function \li{QRDet} that accepts a square matrix of full rank as input, and returns the magnitude of the determinant of that matrix. Use your QR decomposition to perform this
calculation.

You may check your work by comparing your results to those produced by \li{scipy.linalg.det}. 
\end{problem}

\section*{QR Decomposition in SciPy}
The linear algebra library in SciPy wraps the very efficient algorithms of LAPACK to calculate the QR decomposition.
In addition to being much faster, SciPy's QR decomposition is also much more general and can decompose non-square matrices.

\begin{lstlisting}
>>> import numpy as np
>>> from scipy import linalg as la
>>> A = np.random.rand(4,3)
>>> Q, R = la.qr(A)
>>> Q.dot(R) == A                      # there are False entries
>>> np.allclose(Q.dot(R), A)           # A = QR
>>> np.allclose(Q.T.dot(Q), np.eye(4)) # Q is indeed, orthogonal
\end{lstlisting}

In order to interpret the results correctly, we need to understand that the computer has limited precision (especially with floating point numbers).
This is why \li{Q.dot(R)} is not exactly equal to \li{A}. However, we can see that the matrix product of $QR$ is very close to $A$ using the \li{allclose} method. This verifies that the product of $Q$ and $R$ is indeed $A$. Also note that $Q^T Q = I$, which implies that the column vectors of $Q$ are orthonormal.

\section*{Solving Least Squares Problems}
The QR decomposition can only be used to solve the linear least squares problem if $A$ is full rank. If $A$ is less than full rank, then we have to calculate the least squares solution more creatively.

%MATLAB's least squares backslash operator is based off the QR decomposition.
%SciPy uses the SVD to solve least squares problems because, although it is slower, the algorithm is more numerically stable.


For large or ill-conditioned problems, the QR decomposition provides a nice method for computing least squares solutions of over-determined matrices.
Consider the least squares problem $Ax=b$. We can approximate the solution with $\widehat x = (A^T A)^{-1}A^T b$.
Alternatively, we write the linear system as
\[ Q R x = b. \]
We then multiply both sides by $Q^T$, yielding
\[ R x = Q^T b. \]
Then $\widehat x = R^{-1} Q^T b$.

However, we can avoid calculating the inverse of $R$ (inverting a matrix is \emph{very expensive} computationally).
Since $R$ is a triangular matrix, we have a triangular system that we can solve much more efficiently.
SciPy includes a solver for triangular systems, \li{linalg.solve_triangular()}.
We approximate $x$ by solving the triangular system $Rx = Q^T b$.

\begin{problem}
Write a function \li{LeastSquares} that will accept a linear system (a matrix $A$ and a vector $b$) and solve the least squares problem.
Your function should rely on the SciPy's QR decomposition and triangular system solver.  Assume that $A$ is full rank.
You may test your function against the output of SciPy's least squares function, \li{linalg.lstsq()}.
\end{problem}

\section*{Orthonormal transformations}
Recall that a matrix $Q$ is \emph{unitary} if $Q^\mathsf{H} Q = I$ or, for real matrices, $Q^T Q = I$.
For the real case we say that such a matrix is \emph{orthonormal}.

Unitary transformations have the very desirable property of being numerically stable. The number $\kappa(A) = \norm{A} \norm{A^{-1}}$ is called the \emph{condition number} of $A$. We'll discuss condition number more in another lab; for now, all you need to know is that if $\kappa(A)$ is small, then calculations involving $A$ are less susceptible to numerical errors.


For the induced 2-norm, it holds that $\norm{Q}=1$ when $Q$ is unitary.
The Cauchy-Schwarz inequality 

\centerline{$\norm{AB} \leq \norm{A} \norm{B}$}
also holds for this norm, and so it follows that 

\centerline{$\kappa(A) = \norm{A} \norm{A^{-1}} \geq \norm{A A^{-1}} = \norm{I} = 1$.}


Note that if $Q$ is unitary, $Q^{-1} = Q^\mathsf{H}$ and $Q^\mathsf{H}$ is also unitary, so $\kappa(Q) = \norm{Q} \norm{Q^\mathsf{H}} = 1$. This means that orthonormal matrices have the smallest possible condition number.

Any orthogonal matrix $Q$ can be described as a reflection, a rotation, or some combination of the two.
If $det(Q) = 1$, then $Q$ is a rotation.
If $det(Q) = -1$, then $Q$  is a reflection or a composition of a reflection and a rotation. Let's explore these two types of unitary transformations and some of their applications. We will focus on the real case to simplify matters.

\section*{Householder reflections}
A Householder reflection is a linear transformation $P: \mathbb{R}^n \rightarrow \mathbb{R}^n$ that reflects a vector $x$ about a hyperplane.
See figure \ref{fig:Householder_reflector}.
Recall that a hyperplane can be defined by a unit vector $v$ which is orthogonal to the hyperplane. 
As shown in figure \ref{fig:Householder_reflector}, $x - \langle v,x \rangle v$ is the projection of $x$ onto the hyperplane orthogonal to $v$.
However, to reflect \emph{across} the hyperplane, we must move twice as far; that is, $Px = x - 2\langle v,x \rangle v$.
This can be written $Px = x - 2v(v^\mathsf{H} x)$, so $P$ has matrix representation $P = I - 2v v^\mathsf{H}$.
Note that $P^\mathsf{H} P = I$; thus $P$ is orthonormal.

\begin{figure}
\includegraphics[width= \textwidth]{fig1}
\caption{Householder reflector}
\label{fig:Householder_reflector}
\end{figure}

\subsection*{Householder triangularization}
Consider the problem of computing the $QR$ decomposition of a matrix $A$.
You've already learned the Gram-Schmidt and the Modified Gram-Schmidt algorithms for this problem.
The $QR$ decomposition can also be computed by applying a series of Householder reflections.
Gram-Schmidt and Modified Gram-Schmidt make $A$ \emph{orthonormal} using a series of transformations stored in an \emph{upper triangular} matrix.
On the other hand, we can use Householder reflections to make $A$ \emph{triangular} by a series of \emph{orthonormal} transformations.

Let's demonstrate this method on a $4 \times 3$ matrix $A$.
First we find an orthonormal transformation $Q_1$ that maps the first column of A into the span of $e_1$
(where $e_1$ is the vector where the first element is one and the remainder of the elements are zeros).

\def\mc#1{\multicolumn{1}{c|}{#1}}
\begin{equation*}
\begin{pmatrix}
* & * & * \\
* & * & * \\
* & * & * \\
* & * & *
\end{pmatrix}
\underrightarrow{Q_1}
\begin{pmatrix}

* & * & * & \\ \cline{2-3}
\mc{0} & * & \mc{*}& \\
\mc{0} & * & \mc{*} & \\
\mc{0}& * & \mc{*} & \\ \cline{2-3}
\end{pmatrix}
\end{equation*}
Let $A_2$ be the boxed submatrix of $A$.
Now find an orthonormal transformation $Q_2$ that maps the first column of $A_2$ into the span of $e_2$.

\begin{equation*}
\begin{pmatrix}
* & * \\
* & * \\
* & *
\end{pmatrix}
\underrightarrow{Q_2}
\begin{pmatrix}
* & * \\
0 & * \\
0 & *
\end{pmatrix}
\end{equation*}
Similarly, $ \begin{pmatrix} * \\ * \end{pmatrix} \underrightarrow{Q_3} \begin{pmatrix} * \\ 0 \end{pmatrix} $.
(Technically $Q_2$ and $Q_3$ act on the whole matrix and not just on the submatrices, so that $Q_i: \mathbb{R}^n \rightarrow \mathbb{R}^n$ for all $i$.
$Q_2$ leaves the first row and the first column alone, and $Q_3$ leaves the first two rows and the first two columns alone.)
Then $Q_3 Q_2 Q_1 A =$

\begin{equation*}
Q_3 Q_2 Q_1
\begin{pmatrix}
* & * & * \\
* & * & * \\
* & * & * \\
* & * & *
\end{pmatrix}
= Q_3 Q_2
\begin{pmatrix}
* & * & * \\
0 & * & * \\
0 & * & * \\
0 & * & *
\end{pmatrix}
= Q_3
\begin{pmatrix}
* & * & * \\
0 & * & * \\
0 & 0 & * \\
0 & 0 & *
\end{pmatrix}
=
\begin{pmatrix}
* & * & * \\
0 & * & * \\
0 & 0 & * \\
0 & 0 & 0
\end{pmatrix}
\end{equation*}

We've accomplished our goal, which was to triangularize $A$ using orthonormal transformations.
But how do we find the $Q_i$ that do what we want? The answer lies in using Householder reflections.

To find $Q_1$, we first identify an appropriate hyperplane to reflect $x$ into the span of $e_1$.
It turns out there are two hyperplanes that will work, as shown in figure \ref{fig:two reflectors}.
(In the complex case, there are infinitely many such hyperplanes.)
Between the two, the one that reflects $x$ further will be more numerically stable.
This is the hyperplane perpendicular to $v = sign(x_1)\norm{x}_2 e_1 + x$.

To see how this works, let $x$ be the first column of the submatrix that we want to project onto the span of $e_1$.
In order for this to be a unitary operation, this will need to preserve the norm of $x$.
This means that $\left( I - 2 v v^\mathsf{H} \right) x = \pm \norm{x} e_1$, or, in other words,

\[ 2 v v^\mathsf{H} x =
\begin{pmatrix}
x_1 \pm \norm{x} \\
x_2 \\
x_3 \\
\vdots \\
x_n
\end{pmatrix}\]

Let $u$ be the vector on the right hand side of this expression.
It can be shown that the vector  $\frac{u}{\norm{u}}$ is the proper choice for $v$.
%We will show that the vector $\frac{u}{\norm{u}}$ is the proper choice for $v$.
%Notice that:
%
%\[\norm{u}^2 = \norm{x}^2 \pm 2 \norm{x} x_1 + x_1^2 + x_2 + \dots + x_n^2 = 2 \norm{x}^2 \pm 2 \norm{x} x_1 \]
%
%and that
%
%\[\norm{x}^2 \pm \norm{x} x_1 = u^\mathsf{H} x \]
%
%So we have
%
%\begin{align*}
%2 v v^\mathsf{H} x &= 2 u \frac{\norm{x}^2 \pm x_1 \norm{x}}{\norm{u}^2} \\
%		&= 2 u \frac{u^\mathsf{H} x}{\norm{u}^2} \\
%		&= 2 \frac{u}{\norm{u}} \left( \frac{u}{\norm{u}} \right)^\mathsf{H} x
%\end{align*}
%
%So $\frac{u}{\norm{u}}$ is a proper choice of $v$ that will project $x$ into the span of $e_1$.

This whole process is summarized in Algorithm \ref{Alg:Householder}.

\begin{figure}
\includegraphics[width= \textwidth]{fig2}
\caption{two reflectors}
\label{fig:two reflectors}
\end{figure}

\begin{algorithm}
\caption{Householder triangularization}
\label{Alg:Householder}
\begin{algorithmic}[1]
\Procedure{Householder}{$A$}
\State $m, n \gets \text{shape} \left( A \right)$
\State $R \gets \text{copy} \left( A \right)$
\State $Q \gets I_m$
\For{$0 \leq k < n-1$}
    \State $v_k \gets \text{copy} \left( R_{k:,k} \right)$
    \State $v_{k_0} \gets v_{k_0} + \text{sign} \left( v_{k_0} \right) \norm{v_k}$
    \State $v_k \gets v_k / \norm{v_k}$
    \State $R_{k:,k:} \gets R_{k:,k:} - 2 v_k \left( v_k^\mathsf{H} R_{k:,k:} \right)$
    \State $Q_{k:} \gets Q_{k:} - 2 v_k \left( v_k^\mathsf{H} Q_{k:} \right)$
\EndFor
\State \pseudoli{return} $Q^\mathsf{H}, R$
\EndProcedure
\end{algorithmic}
\end{algorithm}

To see how we are operating on the matrices $A$ and $Q$, consider the way each orthonormal transformation defined by the $v_k$ operates blockwise on each matrix.
The matrix form of each operation on $A$ and $Q$ can be represented in block form like this:

\[
\begin{pmatrix}
I & 0 \\
0 & I - 2 v_k v_k^\mathsf{H}
\end{pmatrix}
\]

Notice that a block matrix of this form operates only on entries that lie in the rows from $k$ onward.
Consider what happens when we left-multiply a $m \times n$ matrix by a block matrix of this form.
We obtain the following:

\[
\begin{pmatrix}
I & 0 \\
0 & I - 2 v_k v_k^\mathsf{H}
\end{pmatrix}
\cdot
\begin{pmatrix}
A[:k,:k] & A[:k,k:] \\
A[k:,:k] & A[k:,k:]
\end{pmatrix}
\]
\[
=
\begin{pmatrix}
A[:k,:k] & A[:k,k:] \\
A[k:,:k] - 2 v_k v_k^\mathsf{H} A[k:,:k] & A[k:,k:] - 2 v_k v_k^\mathsf{H} A[k:,k:]
\end{pmatrix}
\]

And, when we consider right multiplication by the same block matrix, we see that it fixes the first $k-1$ columns as below.

\[
\begin{pmatrix}
A[:k,:k] & A[:k,k:] \\
A[k:,:k] & A[k:,k:]
\end{pmatrix}
\cdot
\begin{pmatrix}
I & 0 \\
0 & I - 2 v_k v_k^\mathsf{H}
\end{pmatrix}
=
\begin{pmatrix}
A[:k,:k] & A[:k,k:] - 2  A[:k,k:] v_k v_k^\mathsf{H} \\
A[k:,:k] & A[k:,k:] - 2 A[k:,k:] v_k v_k^\mathsf{H}
\end{pmatrix}
\]

When we are iterating through the columns of $R$ and zeroing out the entries below the main diagonal we are able to safely ignore all the entries that lie in columns we have already processed because they are already zero.

This algorithm returns orthonormal $Q$ and upper triangular $R$ satisfying $A = QR$.nNotice that we did not explicitly construct each orthonormal reflector matrix. We applied the changes we needed to each portion of the array that needed to be changed. Doing the operations in this way allows us to avoid unnecessarily increasing the computational complexity of the algorithm.
A few other clever optimizations can still be applied, but they will not change the overall complexity of the algorithm.

%It should now be clear how it was that we computed $R$ using this algorithm.
%$Q$ is computed in much the same way.
%Since each of the orthonormal operations is self-inverse (i.e. idempotent), $Q$ can be computed by applying the these operations to the identity in reverse order.
%In other words, you could make an identity matrix and then for $k$ such that $n-2 \geq k > -1$ , do $I[k:,k:] -= 2 v_k v_k^\mathsf{H}$
%In our computation, it may be more convenient to simply apply the operations to an identity matrix as we go, just like we are doing to $R$, then take the transpose at the end to invert $Q$.
%This way we do not have to store the $v_k$ as we go.
%There is one key difference, when applying these operations to $Q$ we cannot ignore columns we have already processed because they are not necessarily zero.
%It is interesting to note that we can use the $v_k$ to behave like $Q$ or $Q^{-1}$ depending on the order in which we apply them.
%Such an approach does not require the computation of $Q$ or $Q^{-1}$ at all.

Another important thing to notice is that an outer product is needed to compute $v_k \left( v_k^\mathsf{H} A[k:,k:] \right)$, not an inner product.
Make sure this is accounted for when you write the code to run this algorithm. You can either make the vectors $v_k$ column vectors (two dimensional with a single column) instead of just one-dimensional arrays, or you can use the built in function \li{np.outer} in the appropriate location.

\begin{problem}
\label{prob:HouseholderQR}
Write a function \li{householder} that accepts an array $A$ as input, and performs
the algorithm described above to compute the QR decomposition of $A$. Return the
matrices $Q$ and $R$.

It is simple to check that your code works: multiply the two output matrices
of your function, and check that the result matches the original input matrix.
\end{problem}

\subsection*{Stability of the Householder QR algorithm}
We will now examine the stability of the Householder QR algorithm.
We will use SciPy's built in QR factorization which uses Householder reflections internally.

Try the following.

\begin{lstlisting}
>>> Q, X = la.qr(np.random.rand(500,500)) # create a random orthonormal matrix:
>>> R = np.triu(np.random.rand(500,500)) # create a random upper triangular matrix
>>> A = np.dot(Q,R) # Q and R are the exact QR decomposition of A
>>> Q1, R1 = la.qr(A) # compute QR decomposition of A
\end{lstlisting}

Observe:

\begin{lstlisting}
>>> la.norm(Q1-Q)/la.norm(Q) # check error in Q
0.282842955725
>>> la.norm(R1-R)/la.norm(R) # check error in R
0.0428922016647
\end{lstlisting}

This is terrible!
This algorithm works in $16$ decimal points of precision, but $Q_1$ and $R_1$ are only accurate to $0$ and $1$ decimal points, respectively.
We've lost $16$ decimal points of precision!

Don't lose hope.
Check how close the product $Q_1 R_1$ is to $A$.
\begin{lstlisting}
>>> A1 = Q1.dot(R1)
>>> np.absolute(A1 - A).max()
3.9968028886505635e-15
\end{lstlisting}
We've now recovered $15$ digits of accuracy.
Considering the error relative to the norm of $A$ (using the 2-norm for matrices), we see that this relative error is even smaller.
\begin{lstlisting}
>>> la.norm(A1 - A, ord=2) / la.norm(A, ord=2)
8.8655568331889288e-16
\end{lstlisting}
The errors in $Q_1$ and $R_1$ were somehow ``correlated," so that they canceled out in the product.
The errors in $Q_1$ and $R_1$ are called \emph{forward errors}.
The error in $A_1$ is the \emph{backward error}.

In fact, the large errors in \li{Q1} and \li{R1} were not because the algorithm was bad, it was because $A$ was poorly conditioned.
The condition number for randomly generated upper triangular matrices is generally very high, and this was the case here.
This has, in turn, made the condition number of $A$ extremely large.

Try the following to compute the condition number of $A$.
In this case the condition number of $A$ and $R$ are computed to be different, though, in theory, they should be exactly the same.
\begin{lstlisting}
>>> from numpy.linalg import cond
>>> cond(A)
4.1426075832870472e+18
>>> cond(R)
3.1767577244363792e+19
\end{lstlisting}

Householder QR factorization is more numerically stable than Gram-Schmidt or even Modified Gram-Schmidt (MGS).
However, MGS is still useful for some types of iterative methods because it finds the orthonormal basis one vector at a time instead of all at once (for an example see Lab \ref{lab:EigSolve}).

\subsection*{Upper Hessenberg Form}
An upper Hessenberg matrix is a square matrix with zeros below the first subdiagonal.
Every  $n \times n$ matrix $A$ can be written $A = Q^THQ$ where $Q$ is orthonormal and $H$ is an upper Hessenberg matrix, called the Hessenberg form of $A$.

The Hessenberg decomposition can be computed using Householder reflections in a process very similar to Householder triangularization.
Let's demonstrate this process on a $5 \times 5$ matrix $A$.
Note that $A=Q^THQ$ is equivalent to $QAQ^T = H$. Our strategy is to multiply $A$ on the right and left by a series of orthonormal matrices until it is in Hessenberg form.
If we use the same $Q_1$ as in the first step of the Householder algorithm, then with $Q_1 A$ we introduce zeros in the first column of $A$.
However, since we now have to multiply $Q_1 A$ on the left by $Q_1^T$, all those zeros are destroyed, as demonstrated below.
In order to zero out the entire first column we must choose $Q_1$ appropriately so it does not fix the first row. When we apply the same operation on the right, this ruins the column that we just zeroed out.
(Although this process may seem futile now, it actually does tend to decrease the size of the subdiagonal entries.
If we repeat over and over again, the subdiagonal entries will often converge to zero. That's the idea behind the $QR$ algorithm in Lab \ref{lab:EigSolve}.)
\[
\begin{array}{ccccc}
\begin{pmatrix}
* & * & * & * & * \\
* & * & * & * & * \\
* & * & * & * & * \\
* & * & * & * & * \\
* & * & * & * & *
\end{pmatrix}
&\underrightarrow{Q_1 \cdot }&
\begin{pmatrix}
* & * & * & * & * \\
0 & * & * & * & * \\
0 & * & * & * & * \\
0 & * & * & * & * \\
0 & * & * & * & *
\end{pmatrix}
&\underrightarrow{\cdot Q_1^T }&
\begin{pmatrix}
* & * & * & * & * \\
* & * & * & * & * \\
* & * & * & * & * \\
* & * & * & * & * \\
* & * & * & * & *
\end{pmatrix}
\\
A & & Q_1A & & Q_1 A Q_1^T
  \end{array}
\]
Instead, let's try starting with a different $Q_1$ that leaves the \emph{first} row alone and reflects the \emph{rest} of the rows into the span of $e_2$. This means that $Q_1^T$ leaves the first column alone.
\[
\begin{array}{ccccc}
\begin{pmatrix}
* & * & * & * & * \\
* & * & * & * & * \\
* & * & * & * & * \\
* & * & * & * & * \\
* & * & * & * & *
\end{pmatrix}
&\underrightarrow{Q_1 \cdot }&
\begin{pmatrix}
* & * & * & * & * \\
* & * & * & * & * \\
0 & * & * & * & * \\
0 & * & * & * & * \\
0 & * & * & * & *
\end{pmatrix}
&\underrightarrow{\cdot Q_1^T }&
\begin{pmatrix}
* & * & * & * & * \\
* & * & * & * & * \\
0 & * & * & * & * \\
0 & * & * & * & * \\
0 & * & * & * & *
\end{pmatrix}
\\
A & & Q_1A & & Q_1 A Q_1^T
  \end{array}
\]
We now iterate through the matrix until we obtain
\begin{equation*}
Q_3 Q_2 Q_1 A Q_1^T Q_2 ^T Q_3^T =
\begin{pmatrix}
* & * & * & * & * \\
* & * & * & * & * \\
0 & * & * & * & * \\
0 & 0 & * & * & * \\
0 & 0 & 0 & * & *
\end{pmatrix}
\end{equation*}

This is even more convenient when we are working with Hermitian matrices.
In that case, the matrices applied on the left zero out everything below the first subdiagonal and the matrices applied on the right zero out everything above the first superdiagonal, leaving us with a tridiagonal matrix.
There are remarkably efficient ways to solve systems involving tridiagonal matrices, so this is especially convenient.

The pseudocode for computing the Hessenberg form of a matrix is shown in Algorithm \ref{Alg:Hessenberg}.
The exact inner workings of this algorithm are similar to the inner workings of Algorithm \ref{Alg:Householder}.

\begin{algorithm}
\caption{Reduction to Hessenberg Form}
\label{Alg:Hessenberg}
\begin{algorithmic}[1]
\Procedure{Hessenberg}{$G,u,l,p$}
\State $m, n \gets \text{shape}(A)$
\State $H \gets \text{copy}(A)$
\State $Q \gets I_m$
\For{$0 \leq k < n-2$}
    \State $v_k \gets H_{k+1:, k}$
    \State $v_{k_0} \gets v_{k_0} + \text{sign}(v_{k_0}) \norm{v_k}$
    \State $v_k \gets v_k/norm{v_k}$
    \State $H_{k+1:,k:} \gets H_{k+1:,k:} - 2v_k(v_k^\mathsf{H} H_{k+1:,k:})$
    \State $H_{:,k+1:} \gets H_{:,k+1:} - 2(H_{:,k+1:} v_k) v_k^\mathsf{H}$
    \State $Q_{k+1:} \gets Q_{k+1:} - 2v_k(v_k^\mathsf{H} Q_{k+1:})$
\EndFor
\State \pseudoli{return} $Q, R$
\EndProcedure
\end{algorithmic}
\end{algorithm}

\begin{problem}
\label{prob:hessenberg}
Write a function \li{hessenberg} that computes the Hessenberg form of a real-valued
input matrix $A$. The function should return $Q$ and $H$ satisfying $A = Q^THQ$,
where $Q$ is orthonormal and $H$ has zeros below the first subdiagonal.

The code for this algorithm will be fairly similar to the code for the QR factorization using Householder reflections.
This factorization technique will be used later on in Lab \ref{lab:EigSolve}.
Notice what happens when you compute the Hessenberg factorization of a Hermitian matrix.
\end{problem}

%Sources: http://www.cs.unc.edu/~krishnas/eigen/node5.html
% http://en.wikipedia.org/wiki/Givens_rotation
%http://en.wikipedia.org/wiki/QR_decomposition
%	Note the Operation count: Householder is 2/3 n^3, MGS is 2 n^3
%http://en.wikipedia.org/wiki/QR_algorithm
%Applied Numerical methods using MATLAB by Yang has some code written for this
%http://www.math.kent.edu/~reichel/courses/intr.num.comp.2/lecture21/evmeth.pdf
%	These are eigenvalue algorithms explained carefully
%http://en.wikipedia.org/wiki/Householder_transformation
%Numerical Linear Algebra, by Lloyd N. Trefethen and David Bau III, Chapters 10 and 16 

