cd \lab{Algorithms}{The Dormand-Prince Method}{The Dormand-Prince Method}
\label{lab:DormandPrince}

Many industry grade ode solvers are similar to the RK4 method already described. A common, reliable method that is a good choice for initially studying most problems is the Dormand-Prince method. This method is implemented in Python's \li{scipy.integrate} module as \li{dopri5}, and in MatLab as \li{ode45}. A similar method is the Runge-Kutta-Fehlberg method (RKF45). 

The Dormand-Prince method is a Runge-Kutta method that computes a fourth order accurate solution, followed by a fifth order accurate solution. These solutions are used to estimate the error in the fourth order solution. In turn, the estimated error is used to help determine the size of each step $h_i = t_i-t_{i-1}$ used by the method, instead of used a fixed stepsize $h = (b-a)/n$. 

We will demonstrate how to solve the initial value problem
\begin{eqnarray*}
y'(t) &=& 6+2t-y, \\
y(0) &=& 2,
\end{eqnarray*}
using \li{dopri5}. We start with importing several useful modules and defining the ode. 
The \li{ode} class is imported from the \li{scipy.integrate} module. 
This class functions as an interface to several numerical ode methods, one of which is \li{dopri5}. 
These other methods can be useful in certain situations; however, \li{dopri5} is a good solver to try on new problems. 



We create an instance of the \li{ode} class and initialize it using the \li{set_integrator} and \li{set_initial_value} methods. 
Useful parameters are \li{atol} and \li{rtol}, which set the maximum allowed absolute and relative tolerances for the solution. 
Other parameters and methods are explained in the documentation for \li{scipy}.  
Here the method solves for $y(1.6)$:

\begin{lstlisting}
from scipy.integrate import ode
import numpy as np
import matplotlib.pyplot as plt

a, ya, b = 0., 2., 1.6
def ode_f(t,y): return np.array([-1.*y+6.+2.*t])

ode_object = ode(ode_f)
ode_object.set_integrator('dopri5',atol=1e-5) 
ode_object.set_initial_value(ya,a) 
print ode_object.integrate(b)
\end{lstlisting}

Alternatively, let us solve for $y$ on a evenly spaced mesh, and then plot the results.
%# The output of this function must have the shape (dim,), where dim
%# is the dimension of the system.
\begin{lstlisting}
ode_object = ode(ode_f).set_integrator('dopri5',atol=1e-5) 
ode_object.set_initial_value(ya,a) 

dim, t = 1, np.linspace(a,b,51)
Y = np.zeros((len(t),dim))
Y[0,:] = ya
for j in range(1,len(t)): Y[j,:] = ode_object.integrate(t[j])  

plt.plot(t,Y[:,0],'-k')
plt.show()
\end{lstlisting}


\begin{problem}
Using \li{dopri5}, solve the IVP
\begin{eqnarray*}
5y''' + y'+2y &=& 0, \,\, 0 \leq x \leq 16,\\
y(0) &=&0,\\
y'(0) &=& 1,\\
y''(0) &=& -2.
\end{eqnarray*}
\end{problem}




\section*{The SIR Model}
The SIR model describes the spread of an epidemic throughout a large population. It does this by describing the movement of the population through three phases of the disease: those individuals who are Susceptible, those who are Infectious, and those who have been Removed from the disease. Those individuals in the Removed class have either died, or have recovered from the disease and are now immune to it. If the outbreak occurs over a short period of time, we may reasonably assume that the total population is fixed, so that $S'(t) + I'(t) + R'(t) = 0$.  We may also assume that $S(t) + I(t) + R(t) = 1$, so that $S(t)$ represents the \textit{fraction} of the population that is susceptible, etc. 

Individuals may move from one class to another as described by the flow 
\[S \to I \to R.\] Let us consider the transition rate between $S$ and $I $. Let $\beta$ represent the average number of contacts made per day that could spread the disease. The proportion of these contacts that are with a susceptible individual is $S(t)$. Thus, one infectious individual will on average infect $\beta S(t)$ others per day. Let $N$ represent the total population size. Then we obtain the differential equation
 \[\frac{d}{dt}(S(t) N) = -\beta S(t) (I(t) N).\]
 
 Now consider the transition rate between $I$ and $R$. We assume that there is a fixed proportion $\gamma$ of the infectious group who will recover on a given day, so that 
 \[\frac{d}{dt}R(t) = -\gamma I(t).\]
 Note that the proportion who will recover on a given day is the reciprocal of the average length of time spent in the infectious phase. 
 
 Since we have $I'(t) = - S'(t) - R'(t)$, the  differential equations are given by
\begin{eqnarray*}
\frac{dS}{dt} &=&-\beta IS ,\\
\frac{dI}{dt} &=& \beta I S-\gamma I, \\
\frac{dR}{dt} &=&\gamma I.
\end{eqnarray*}


\begin{problem}
Solve the IVP
\begin{eqnarray*}
\frac{dS}{dt} &=&-\frac{1}{2} IS ,\\
\frac{dI}{dt} &=& \frac{1}{2} I S-\frac{1}{4} I, \\
\frac{dR}{dt} &=&\frac{1}{4} I,\\
S(0) &=& 1-6.25\cdot10^{-7},\\
I(0) &=& 6.25\cdot10^{-7},\\
R(0) &=&0,
\end{eqnarray*}
on the interval $[0,100]$.
\end{problem}


\begin{problem}
Suppose that, in a city of approximately three million, five have recently entered the city carrying a certain disease. (Suppose they have just entered the infectious state.)

Each of those individuals has a contact each day that will spread the disease, and average of three days is spent in the infectious state. Find the solution of the corresponding SIR equations for the next fifty days. 

Assume for simplicity that those who are in the infectious state either cannot go to work or are unproductive, etc. At the peak of the infection, how many in the city will still be able to work? 
Answer the same question if instead of three days, an average of seven days is spent in the infectious state.
\end{problem}


\begin{problem}
Suppose that, in a city of approximately three million, five have recently entered the city carrying a certain disease. (Suppose they have just entered the infectious state.) 

Each of those individuals will make three contacts every ten days that will spread the disease, and average of four days is spent in the infectious state. Find the solution of the corresponding SIR equations. Choose an appropriate time interval and plot your results.
\end{problem}


\section*{A Weightloss Model}
Here we will look at a model of weight change based on basic thermodynamics and kinematics.
The main idea behind weight change is simple.  If a persons energy intake is more than their energy expended, then they gain weight.  If their intake is less, then they lose weight.
Let \emph{energy balance}, denoted $EB$, be the difference between \emph{energy intake} $EI$ and \emph{energy expenditure} $EE$, that is,
\begin{equation}
\label{EB}
EB = EI - EE.
\end{equation}
When the energy intake exceeds the energy expended, the balance is positive and weight is gained.  Correspondingly, if the amount of energy expended exceeds the energy intake, the balance is negative and weight is lost.


Ones body weight at time $t$ is the sum of the weight of their fat and lean tissue; that is,  $BW(t) = F(t) + L(t).$ These quantities can be described by the compartmental model
\begin{subequations}
\label{compartment}
\begin{align}
\rho_F \dfrac{dF(t)}{dt} &= (1-p(t)) EB(t),\label{compartment:a}\\
\rho_L \dfrac{dL(t)}{dt} &= p(t) EB(t),\label{compartment:b}
\end{align}
\end{subequations}
where $p(t)$ and $1-p(t)$ represent the proportion of the energy balance ($EB(t)$) that results in a change in the quantity of lean or fatty tissue, respectively. Constants $\rho_L$ and $\rho_F$ represent the energy density of lean and fatty tissue (about $1800$ and $9400$ kcal/kg).

Next we need to find expressions for $p(t)$ and $EB(t)$ in terms of $L$ and $F$ (the dependent variables), $PAL$ and $EI$ (possibly varying parameters), and other constant parameters.

 The proportion $p(t)$ will vary with $F$ and $L$; from Forbes' Law \cite{Fo.2} we have that
\begin{equation}
\label{Forbes}
\dfrac{dF}{dL} = \dfrac{F}{10.4}.
\end{equation}
Hence,
\[
\dfrac{F}{10.4} = \dfrac{dF}{dL} = \dfrac{dF/dt}{dL/dt} = \dfrac{\dfrac{(1-p(t)) EB(t)}{\rho_F}}{\dfrac{p(t) EB(t)}{\rho_L}} = \dfrac{\rho_L}{\rho_F} \dfrac{1-p(t)}{p(t)}.
\]
Solving for $p(t)$ gives Forbes' equation
\begin{equation}
\label{Forbes2}
p(t) = \dfrac{C}{C+F(t)}\quad\mbox{where}\quad C=10.4\dfrac{\rho_L}{\rho_F}.
\end{equation}


We will use two expressions for energy expenditure (EE). First, we have the formula
\begin{equation}
\label{EE0}
EE = PAL \times RMR,
\end{equation}
where $PAL$ is your physical activity level and $RMR$ your resting metabolic rate.
Your resting metabolic rate can be determined by using the Mifflin equation. This equation is an estimate based on a population study and is widely used in the literature. It takes into account your gender, age (A) in years, and height (H) in meters:
\begin{equation}
\label{RMR}
RMR = \begin{cases} 9.99 W + 625 H + 5 A + 5 & \mbox{if male}\\ 9.99 W + 625 H + 5 A -161 & \mbox{if female.}\end{cases}
\end{equation}  
Your physical activity level can be determined by using the table below.
\begin{table}[h]
\begin{center}
\begin{tabular}{|l|l|}
\hline
1.40--1.69 & People who are sedentary and do not exercise regularly, spend \\
& most of their time sitting, standing, with little body displacement
\\
\hline
1.70--1.99 & People who are active, with frequent body displacement throughout  \\
& the day or who exercise frequently\\
\hline
2.00--2.40 & People who engage regularly in strenuous work or exercise for \\
& several hours each day\\
\hline
\end{tabular}
\caption{This is a rough guide for physical activity level (PAL).  
% For more detailed estimates, see \cite{Heym} and Appendix \ref{PAL_appendix}.
}
\end{center}\label{PAL_table}
\end{table}




The second expression for energy expenditure comes from decomposing more precisely the different ways that energy is expended:
\begin{equation}
\label{EE}
EE = \underbrace{\delta BW}_\text{\parbox{1cm}{physical\\activity}} + \underbrace{\beta_{tef} EI}_\text{\parbox{1cm}{thermic\\effect of\\eating}} + \underbrace{\beta_{at} EI + \gamma_F F + \gamma_L L + \eta_F \dfrac{dF}{dt} + \eta_L \dfrac{dL}{dt}  + K}_\text{resting metabolic rate (RMR)},
\end{equation}
where $\gamma_F = 22$ kcal/kg/d, $\gamma_L = 3.2$ kcal/kg/d, $\eta_F = 180$ kcal/kg, and $\eta_L = 230$ kcal/kg; see \cite{Hall.2, Hall.4}.  Further, we let $\beta_{tef}=0.10$ and $\beta_{at}=0.14$ denote the coefficients for the thermic effect of feeding and adaptive thermogenesis, respectively.  The parameter $\delta$ is the coefficient representing the amount of energy expended from physical activity per kilogram of body mass.  Notice that $\gamma_L$ is significantly larger than $\gamma_F$.  This means that lean tissue metabolizes energy much faster than fatty tissue.  As a result, there are instances where one may want to increase their lean body mass through resistance training so that they are better able to support a higher caloric intake without significant weight gain.  Finally, we remark that the constant $K$ can be tuned to an individual's body type directly through RMR and fat measurement, and is assumed to remain constant over time.

% Assumptions made/Areas to improve: 
% include more accurate approximation of PAL (given in appendix), BMI (show to vary with race), account for variation in body type.

Thus, since the input $EI$ is assumed to be known, we can use \eqref{EE} and \eqref{Forbes2} to write \eqref{compartment} in terms of $F$ and $L$, thus allowing us to close the system of ordinary differential equations (ODEs).  

Specifically, we have
\begin{align*}
RMR(t) = \frac{EE}{PAL}&= K + \gamma_F F(t) + \gamma_L L(t) + \eta_F \dfrac{dF}{dt} + \eta_L \dfrac{dL}{dt}  + \beta_{at} EI\\
% &= K + \gamma_F F(t) + \gamma_L L(t) + \dfrac{\eta_F}{\rho_F} (1-p(t)) EB(t) + \dfrac{\eta_L}{\rho_L} p(t) EB(t)  + \beta_{at} EI,\\
\dfrac{1}{PAL}\left(EE - EI + EI \right) &= K + \gamma_F F(t) + \gamma_L L(t) \\
&+ \left(\dfrac{\eta_F}{\rho_F} (1-p(t)) + \dfrac{\eta_L}{\rho_L} p(t) \right) EB(t) + \beta_{at} EI.\\
\left(\dfrac{1}{PAL}-\beta_{at}\right) EI &= K + \gamma_F F(t) + \gamma_L L(t) \\
&+ \left(\dfrac{\eta_F}{\rho_F} (1-p(t)) + \dfrac{\eta_L}{\rho_L} p(t) + \dfrac{1}{PAL}\right) EB(t).
\end{align*}
% Thus,
% \[
% \left(\dfrac{1}{PAL}-\beta_{at}\right) EI = K + \gamma_F F(t) + \gamma_L L(t) + \left(\dfrac{\eta_F}{\rho_F} (1-p(t)) + \dfrac{\eta_L}{\rho_L} p(t) + \dfrac{1}{PAL}\right) EB(t).
% \]
Solving for $EB(t)$ in the last equation yields
\begin{equation}
\label{EB2}
EB(t) = \dfrac{\left( \dfrac{1}{PAL} - \beta_{at} \right) EI - K - \gamma_F F(t) - \gamma_L L(t)}{\dfrac{\eta_F}{\rho_F} (1-p(t)) + \dfrac{\eta_L}{\rho_L} p(t) + \dfrac{1}{PAL}}.
\end{equation}
% To find $K$, we note that
% \begin{equation}
% \label{K}
% K = \left(\dfrac{1}{PAL}-\beta_{at}\right) EI - \gamma_F F(t) - \gamma_L L(t) - \eta_F \dfrac{dF}{dt} - \eta_L \dfrac{dL}{dt}  - \dfrac{1}{PAL} EB.
% \end{equation}
In equilibrium ($EB = 0$), this gives us
\begin{equation}
\label{K2}
K = \left(\dfrac{1}{PAL}-\beta_{at}\right) EI - \gamma_F F - \gamma_L L.
\end{equation}
Thus, for a subject who has maintained the same weight for a while, one can determine $K$ by using \eqref{K2}, if they know their average caloric intake and amount of fat (assume $L=BW-F$). The function \li{weight_odesystem} in the following code implements \eqref{compartment}.


\begin{lstlisting}
class c: pass

# 
# Fixed Constants
# 
c.rho_F		= 9400.   # 
c.rho_L		= 1800.   #
c.gamma_F	= 3.2     # 
c.gamma_L	= 22.     # 
c.eta_F		= 180.    # 
c.eta_L		= 230.    # 
c.C			= 10.4    # Forbes constant
c.beta_AT	= 0.14    # Adaptive Thermogenesis
c.beta_TEF	= 0.1     # Thermic Effect of Feeding

K 			= 0

def weight_odesystem(t,y,EI,PAL):
	F, L = y[0], y[1]
	p, EB = Forbes(F), EnergyBalance(F,L,EI,PAL)
	return np.array([(1-p)*EB/c.rho_F,p*EB/c.rho_L]) 

def EnergyBalance(F,L,EI,PAL):
	p = Forbes(F)
	a1 = (1./PAL-c.beta_AT)*EI - K - c.gamma_F*F - c.gamma_L*L
	a2 = (1-p)*c.eta_F/c.rho_F + p*c.eta_L/c.rho_L+1./PAL
	return a1/a2

def Forbes(F):
	C1 = c.C*c.rho_L/c.rho_F
	return 1.*C1/(C1+F)

def fat_mass(BW,age,H,sex):
	BMI = BW/H**2.
	if sex=='male': 
		return BW*(-103.91 + 37.31*np.log(BMI)+0.14*age)/100
	else: 
		return BW*(-102.01 + 39.96*np.log(BMI)+0.14*age)/100
	
\end{lstlisting}



\begin{problem}
Consider the initial value problem
\begin{subequations}
% \label{compartment}
\begin{align*}
\rho_F \dfrac{dF(t)}{dt} &= (1-p(t)) EB(t),\\ %\label{compartment:a}\\
\rho_L \dfrac{dL(t)}{dt} &= p(t) EB(t),\\%\label{compartment:b}
F(0) &= F_0, \\
L(0) &= L_0.
\end{align*}
\end{subequations}

To solve this IVP for a specific individual we need initial conditions $F_0$ and $L_0.$ The function \li{fat_mass} given earlier calculates these values based on an individual's body weight (kg), age, height (meters), and gender. ($L = BW - F$.)

\end{problem}











