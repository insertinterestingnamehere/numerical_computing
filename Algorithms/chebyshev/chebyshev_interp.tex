\lab{Python}{Polynomial Interpolation Using Chebyshev Polynomials}{Polynomial Interpolation Using Chebyshev Polynomials}
\label{lab:cheb_interp}

\objective{Explore basic uses of polynomial interpolation using Chebyshev polynomials.}

\section*{Chebyshev nodes}

In previous labs we have explored ways to compute interpolating polynomials given sets of points.
We have also noted that such interpolation may still be a very poor approximation to a function depending on the function itself and the choice of interpolating points.
The question then arises, "when can polynomial interpolation be a good approximation to a given function?"
You may recall that equally spaced points provided a very poor approximation to a given function around the edge of the interval where we are performing the interpolation.
This weakness in interpolation using equispaced points is called Runge's Phenomenon.
We can avoid Runge's Phenomenon by choosing to interpolate our function with a different set of points.
Since there is an increased amount of instability toward the edges of the interval where we are interpolating we will choose more points toward the edges of the interval so that the approximation remains accurate throughout the whole interval.
It can be shown that, when forming an interpolating polynomial of degree $n$, the ideal points to use for interpolation on the interval $[-1, 1]$ are the points $\cos{\frac{\pi k}{n}}$ for $k\in\mathbb{Z}$, $0 \leq k \leq n$.
These points are called the Chebyshev nodes.
These points can also be viewed as the projection of equispaced points along the half-circle in the complex plane projected onto the real axis.
This interpretation is shown in Figure [Add Figure].
They can be shifted and scaled appropriately for use on any interval.
As we have presented them here they are on the interval $[-1, 1]$.
These nodes provide for much more stable interpolation of functions as can be seen in Figure [Add Figure].

\section*{Chebyshev Polynomials}
This sort of interpolation is also related to what are called the Chebyshev polynomials.
Chebyshev Polynomials can be used 