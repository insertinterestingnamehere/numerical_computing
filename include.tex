\newenvironment{amatrix}[1]{%
\left(\begin{array}{@{}*{#1}{c}|c@{}}
}{%
\end{array}\right)
}

\newenvironment{dmatrix}[2]{%
\left(\begin{array}{@{}*{#1}{c}|*{#2}{c}@{}}
}{%
\end{array}\right)
}

\newenvironment{pseudo}[2]
    {\begin{pseudocode}[shadowbox]{#1}{#2}}
    {\end{pseudocode}}

\newenvironment{problem}{\begin{shaded}\begin{problemnum}}{\end{problemnum}\end{shaded}}

\newtheoremup{problemnum}{Problem}
\definecolor{shadecolor}{gray}{0.90}

\newcommand{\li}[1]{\lstinline[style=python]!#1!}

\newcommand{\ipt}[2]{\langle #1,#2 \rangle}
\newcommand{\ip}{\int_{-\infty}^{+\infty}}

\renewcommand{\ker}[1]{\mathcal{N}(#1)}
\newcommand{\ran}[1]{\mathcal{R}(#1)}

\makeatletter
\g@addto@macro\@floatboxreset\centering
\makeatother

\DeclareMathOperator{\res}{res}           % Residue
\DeclareMathOperator{\Res}{Res}           % Residue


\def\0{{\bf 0}}
\def\a{{\bf a}}
\def\b{{\bf b}}
\def\e{{\bf e}}
\def\p{{\bf p}}
\def\q{{\bf q}}
\def\u{{\bf u}}
\def\v{{\bf v}}
\def\w{{\bf w}}
\def\x{{\bf x}}
\def\y{{\bf y}}
\def\z{{\bf z}}
\def\subspace{\lhd}

\def\CalL{\mathcal{L}}
\def\CalO{\mathcal{O}}
\def\CalV{\mathcal{V}}
\def\CalU{\mathcal{U}}
\def\bU{{\bar{u}}}
\def\R{\Re e}
\def\I{\Im m}
\def\M{M_n}

\lstset{basicstyle=\footnotesize\ttfamily,
        keywordstyle=\color{blue}\bfseries,
        tabsize=4,
        frame=tb,
        captionpos=b,
        title=\lstname,
        abovecaptionskip=-5pt,
        belowcaptionskip=-5pt,
        breaklines=true,
        breakatwhitespace=false,
        showstringspaces=false}

\lstdefinestyle{fromfile}{frame=single
                          numbers=left,
                          numberstyle=\tiny,
                          stepnumber=2,
                          numbersep=7pt,
                          numberfirstline=true,
                          abovecaptionskip=10pt,
                          belowcaptionskip=10pt}

\lstset{basicstyle=\footnotesize\ttfamily,
		keywordstyle=\color{blue}\bfseries\ttfamily,
		tabsize=4,
		frame=tb,
		captionpos=b,
		breaklines=true,
		breakatwhitespace=false,
		title=\lstname,
		showstringspaces=false}



%----PYTHON STYLES----                          
\lstdefinestyle{python}{language=Python}
\lstdefinestyle{pythonnums}{language=Python,
                            numbers=left,
                            numberstyle=\tiny,
                            stepnumber=2,
                            numbersep=7pt,
                            numberfirstline=true}